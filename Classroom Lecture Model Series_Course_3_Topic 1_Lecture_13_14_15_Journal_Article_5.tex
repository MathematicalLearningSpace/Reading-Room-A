%-------------------------------------------------------------------------------------------------------------------%
%--------------------Classroom Lecture Model Series-----------------------------------------------------------------%
%-------------------------------------------------------------------------------------------------------------------%
%-----------------Sample Article in Progress for the Classroom and Journal Publication------------------------------%
\begin{document}
\twocolumn
\scriptsize
\begin{frontmatter}
		\title{DRAFT Sample: A QSAR Feature Matrix Design For Protein-Compound Interaction}
		\author{The Mathematical Learning Space, \corref{cor1}\fnref{fn1}}
		\cortext[cor1]{Corresponding author}
		\address{The Mathematical Learning Space}
		\ead{http://mathlearningspace.weebly.com}	
\end{frontmatter}	

Introduction:
\begin{enumerate}
\item Objective 1:
\item Objective 2:
\item Objective 3:
\end{enumerate}
Conclusion:

Keywords: Feature Matrix Design, Molecular Machine Learning, Cell Cycle Proteins, Gastric and Stomach Cancer, Pennsylvania cancer survival rates, Statistical learning, Supervised Classification, Random Forest, Variable Importance 
Vocabulary Words:

\section{Introduction}

\begin{table}[H]\centering
	\begin{tabular}{p{1cm}p{4cm}p{3cm}}
		Article ID & Summary & Comments\\
		\hline
		\hline
	\end{tabular}
\end{table}


\subsection{Gastric/Stomach Cancer}

\begin{table}[H]\tiny
	\caption{Genes/proteins that are basic to the analysis of gastric cancer \cite{key400},\cite{key401},\cite{key402}}	
	\begin{tabular}{p{3cm}p{3cm}p{1cm}}
		\hline
		Stage & List of Gene & \\
		\hline 
		Drug Resistance & CDX2, MUC2, REG4, CDH17, MDR1, SHH \\
		Genomic Instability & p53, p21, BAX, p48, GADD45, BAK, POLK \\
		Tumor Progression & Retinoic.Acid, RAR.Beta, RXR \\
		Intestinal Metaplasia & DV1, GSK.3Beta, Beta.Catenin, Axin, APC, CK1.alpha,GBP \\
		Dysplasia Path 1 & EFG, ERBB2, SHC, GRB2, SOS, RAS, RAF, MEK, ERK.1 \\
		Dysplasia Path 2 & PI3K, PIP3, AKT, mTOR, p53, S6K, BCL2 \\
		Dysplasia Path 3 & TGF.Beta,TGF.BetaRI,TGF.BetaRII, SMAD.2, SMAD.4, p15, p21 \\
		Normal Gastic Muscosa 1 & HGF, c.MET, GRB2, SOS, RAS, RAF, MEK, ERK.1 \\
		Normal Gastic Muscosa Survival Path 1 & FGF, FGFR2, GAB1, PI3K, PIP3, AKT, mTOR, GSK.3Beta \\
		\hline  
	\end{tabular}
\end{table} 

\begin{table}[H]\tiny
	\caption{State Changes in Signal Transduction for Gastric Cancer Pathways \cite{key400},\cite{key401},\cite{key402}}	
	\begin{tabular}{p{1cm}p{2cm}p{4cm}p{0.1cm}}
		\hline	
		Category &State Change Category & Pathway &  \\
		\hline
		A & Amplified & FGFR to RAS-ERK signaling pathway &  \\
		A & Amplified & FGFR to PI3K signaling pathway &  \\
		A & Amplified & ERBB2 to RAS-ERK signaling pathway &  \\
		A & Amplified & ERBB2 to PI3K signaling pathway &  \\
		A & Amplified & MET to RAS-ERK signaling pathway &  \\
		A & Amplified & MET to PI3K signaling pathway &  \\
		A & Amplified & CCNE to cell cycle G1/S &  \\
		\hline
		B & Mutation-activated & KRAS/NRAS to ERK signaling pathway &  \\
		B & Mutation-inactivated & APC to Wnt signaling pathway &  \\
		B & Mutation-inactivated & TP53 to transcription &  \\
		B & Mutation-inactivated & CDH1 to beta-catenin signaling pathway &  \\
		\hline
		C & Over expression & TERT to telomerase activity &  \\
		C & Over expression & CDX2 to transcriptional activation &  \\
		C & Over expression & CDX2 to transcriptional repression &  \\
		\hline
		D & Reduced expression & CDKN1B to p27-cell cycle G1/S &  \\
		D & Reduced expression & TGFBR1 to TGF-beta signaling pathway &  \\
		E & Loss & of CDH1 to beta-catenin signaling pathway &  \\
		\hline 
	\end{tabular}
\end{table}

\subsection{Plan of the Article}

\begin{enumerate}
\end{enumerate}


\section{Cell Cycle Proteins}

\begin{table}[H]\tiny
\caption{Sample Cell Cycle Proteins Group 1 from Fabregat et al. 2018 PMID: 29145629}	
\begin{tabular}{p{0.5cm}p{1.5cm}p{1.5cm}p{1.5cm}p{1.5cm}}
\hline
ID &&&&\\ 
\hline   
  1 & Q9Y6D9.MD1L1N &  Q2NKX8.ERC6LN &  Q7L7X3.TAOK1N &  P43034.LIS1\\   
  5 & O95229.ZWINTN &  P46060.RAGP1N &  P53350.PLK1N &  O75122.CLAP2\\  
  9 & Q562F6.SGO2N &  Q15691.MARE1N &  P36873.PP1GN &  Q8WYP5.ELYS\\   
 13 & Q02224.CENPEN &  Q9P258.RCC2N &  O00139.KIF2AN &  P30153.2AAA\\   
 17 & P30154.2AABN &  Q15173.2A5BN &  Q13362.2A5GN &  Q15172.2A5A\\   
 21 & Q14738.2A5DN &  Q16537.2A5EN &  P67775.PP2AAN &  P62714.PP2AB\\  
 25 & Q8NI77.KI18AN &  P42677.RS27N &  Q12834.CDC20N &  Q5FBB7.SGO1\\   
 29 & O43683.BUB1N &  Q7Z460.CLAP1N &  Q8WVK7.SKA2N &  O14980.XPO1\\   
 33 & Q96BD8.SKA1N &  P49454.CENPFN &  Q13257.MD2L1N &  O43684.BUB3\\   
 37 & O60566.BUB1BN &  P52948.5.NUP98 & Q96EE3.1.SEH1 & P57740.NU107\\  
 41 & Q8NFH3.NUP43N &  Q12769.NU160N &  Q8NFH4.NUP37N &  Q9BW27.NUP85\\  
 45 & Q8WUM0.NU133N &  P55735.SEC13N &  P50748.KNTC1N &  O43264.ZW10\\   
 49 & Q9H900.ZWILCN &  Q9BZD4.NUF2N &  Q8NBT2.SPC24N &  O14777.NDC80\\  
 53 & Q9HBM1.SPC25N &  Q9H081.MIS12N &  Q96IY1.NSL1N &  Q9H410.DSN1\\   
 57 & Q6P1K2.PMF1N &  Q8NG31.KNL1N &  O15392.BIRC5N &  Q96GD4.AURKB\\  
 61 & Q9NQS7.INCEN &  Q53HL2.BOREAN &  P49450.CENPAN &  Q96BT3.CENPT\\  
 65 & Q03188.CENPCN &  Q9BS16.CENPKN &  Q92674.CENPIN &  Q9H3R5.CENPH\\  
 69 & Q8N0S6.CENPLN &  Q8N2Z9.CENPSN &  Q9BU64.CENPON &  Q6IPU0.CENPP\\  
 73 & Q71F23.CENPUN &  Q7L2Z9.CENPQN &  Q13352.CENPRN &  Q9NSP4.CENPM\\  
 77 & Q96H22.CENPNN &  Q9BPU9.B9D2N &  Q13409.DC1I2N &  O14576.DC1I1\\  
 81 & Q14204.DYHC1N &  Q9Y6G9.DC1L1N &  O43237.DC1L2N &  P63167.DYL1\\   
 85 & Q96FJ2.DYL2N &  Q9GZM8.NDEL1N &  Q9Y266.NUDCN &  Q8N4N8.KIF2B\\  
 89 & P49792.RBP2N &  Q99661.KIF2CN &  Q14008.CKAP5N &  Q9NXR1.NDE1\\   
 93 & P30622.CLIP1N &  Q96EA4.SPDLYN &  Q9UJX3.APC7N &  Q9UJX4.APC5\\   
 97 & Q8NHZ8.CDC26N &  O00762.UBE2CN &  Q9NYG5.APC11N &  Q13042.CDC16\\  
\hline 
\end{tabular}
\end{table}

\section{Feature Matrix Design}

\begin{table}[H]\tiny
 \caption{Examples of Molecular Descriptors and Categories }
\begin{tabular}{p{1cm}p{1cm}p{6cm}}
		\hline	
Letter & Category & Description \\
\hline
& sequence & sequences of Uniprot Peptides \\
A & group  & group code "0" for A and "1" and not A \\
B & length & length of the amino acid sequence \\
C & mw & molecular weight of the amino acid sequence \\
D & tinyAA & fraction percent of tiny amino acids for the sequence \\
E & smallAA & fraction percent of small amino acids for the sequence \\
F & aliphaticAA & fraction percent of aliphatic amino acids for sequence \\
G & aromaticAA & fraction percent of aromatic amino acids for sequence \\
H & nonpolarAA & fraction percent of non-polar amino acids for sequence \\
I & polarAA & fraction percent of polar amino acids for sequence \\
J & chargedAA & fraction percent of charged amino acids for sequence \\
K & basicAA & fraction percent of basic amino acids for sequence \\
L & acidicAA & fraction percent of acid amino acids for sequence \\
M & charge & charge of the amino acid sequence \\
N & pI & isoelectric point of the amino acid sequence \\
O & aindex & aliphatic index of the amino acid sequence \\
P & instaindex & instability index of the amino acid sequence \\
Q & boman & potential peptide-interaction index of the amino acid sequence \\
R & hydrophobicity & hydrophobicity index of the amino acid sequence \\
S & hmoment & hydrophobic moment of the amino acid sequence \\
T & transmembrane & fraction of Transmembrane windows of N amino acids for sequence \\
U & surface & fraction of Surface windows of N1 amino acids for sequence \\
V & globular & fraction of Globular windows of N2 amino acids for sequence \\
  \hline
\end{tabular}
\end{table}

Based on this collection the following were selected for a feature matrix in Table 1.

Response.Vector=ifelse(Feature.1>= 40, 1, 0)

\begin{table}[H]\tiny
  \caption{1D Index Function Design for Proteins }
\begin{tabular}{p{1cm}p{1cm}p{5cm}}
\hline 
Feature & Scalar Relation & Parameters \\
\hline 
Feature.1 & instaIndex & x \\
Feature.2 & lengthpep & x \\
Feature.3 & boman & x \\
\hline 
Feature.4 & charge & x,pH=5 \\
Feature.5 & charge & x,pH=7 \\
Feature.6 & charge & x,pH=9 \\
\hline 
Feature.7 & aIndex & x \\
Feature.8 & pI & x \\
Feature.9 & mw & x \\
\hline 
Feature.10 & hmoment & x angle=100 window=11 \\
Feature.11 & hmoment & x angle=160 window=11 \\
\hline
\end{tabular}
\end{table}

\section{Molecular Machine/Statistical Learning}

\section{Algorithms}

The following algorithm presented the methodology for the classical statistical learning.
\footnotesize
\begin{algorithm}[H]
\begin{algorithmic}[1]
\State Categorize the Proteins Based On Residue Functional Equilvalence Relations
\State Assemble the Collection into A Feature Matrix for Machine Learning Models
\State Fit a Multivariate Mixture Model to the Feature Matrix
\State Choose a Dependent Vector from the Columns for Multinomial Classification 
\State Select a Machine Learning Model for Prediction from a Group of Classification Models
\State Based on Predictions, Examine the Accuracy and Specificity Relationship with ROC Analysis
\State Compute the AUC and Variable Importance
\end{algorithmic}
\caption{ Mulitvariate Mixture Distributions }
\label{Multivariate_Distrib}
\end{algorithm}

\section{Supervised Classification Models}

\begin{table}[H]\tiny
 \caption{Model Design Patterns for the AUC Maximization}
\begin{tabular}{p{1cm}p{3cm}p{1cm}p{1cm}p{1cm}}
\hline 
Model ID & Description & Parameters & AUC & Reference \\ 
\hline
\hline 
\end{tabular}
\end{table}

\section{Results}

\begin{figure}[H]
	\centering
	\begin{minipage}[b]{0.3\linewidth}
		\includegraphics[scale=0.15]{Example_1_Figure_1.png}
	\end{minipage}\hfill
	\begin{minipage}[b]{0.3\linewidth}
		\includegraphics[scale=0.15]{Example_1_Figure_2.png}
	\end{minipage}\hfill	
	\begin{minipage}[b]{0.3\linewidth}
		\includegraphics[scale=0.15]{Example_1_Figure_3.png}
	\end{minipage}\hfill
	\begin{minipage}[b]{0.3\linewidth}
		\includegraphics[scale=0.15]{Example_1_Figure_4.png}
	\end{minipage}\hfill
	\begin{minipage}[b]{0.3\linewidth}
		\includegraphics[scale=0.15]{Example_1_Figure_5.png}
	\end{minipage}\hfill	
	\begin{minipage}[b]{0.3\linewidth}
		\includegraphics[scale=0.15]{Example_1_Figure_6.png}
	\end{minipage}\hfill
	\begin{minipage}[b]{0.3\linewidth}
		\includegraphics[scale=0.15]{Example_1_Figure_7.png}
	\end{minipage}\hfill
	\begin{minipage}[b]{0.3\linewidth}
		\includegraphics[scale=0.15]{Example_1_Figure_8.png}
	\end{minipage}\hfill	
	\begin{minipage}[b]{0.3\linewidth}
		\includegraphics[scale=0.15]{Example_1_Figure_9.png}
	\end{minipage}\hfill
	\caption{(a) Receiver Operating Characteristic Curve (b) Marginal distribution Feature 1, (c) Marginal distribution Feature 2, (d) Marginal distribution Feature 3, (e) Marginal distribution Feature 4,
(f) Marginal distribution Feature 5, (g) Marginal distribution Feature 6, (h) Marginal distribution Feature 7, and (i) Marginal distribution Feature 8}
	\label{fig:Figure1}
\end{figure} 

\subsection{Variable Importance}

\begin{table}[H]\tiny
  \caption{Feature Matrix Combinations of Variable Importance}
\begin{tabular}{p{1cm}p{2cm}p{1cm}p{2cm}p{1cm}}
\hline 
Feature & Model 1 & Model 2 & Model 3 & Model 4\\
\hline 
\hline 
\end{tabular}
\end{table}

Figures 1 - 3 present the variable importance for each of the models. 

\begin{figure}[H]
	\begin{minipage}[b]{0.7\linewidth}
		\includegraphics[scale=0.5]{Example_2_Figure_1.png}
	\end{minipage}\hfill
	\caption{(a) Variable Importance Model 1 }
	\label{fig:Figure1}
\end{figure} 


\section{Conclusion}

\bibliographystyle{plain}
\begin{thebibliography}{00}

\bibitem[1]{key1}U.S. Cancer Statistics Working Group. U.S. Cancer Statistics Data Visualizations Tool, based on November 2018 submission data (1999-2016): 
\newblock U.S. Department of Health and Human Services, Centers for Disease Control and Prevention and National Cancer Institute; 
\newblock www.cdc.gov/cancer/dataviz, June 2019.

\bibitem[400]{key400} Kanehisa, Furumichi, M., Tanabe, M., Sato, Y., and Morishima, K.; 
\newblock KEGG: new perspectives on genomes, pathways, diseases and drugs. 
\newblock Nucleic Acids Res. 45, D353-D361 (2017).

\bibitem[401]{key401} Kanehisa, M., Sato, Y., Kawashima, M., Furumichi, M., and Tanabe, M.; 
\newblock KEGG as a reference resource for gene and protein annotation. 
\newblock Nucleic Acids Res. 44, D457-D462 (2016).

\bibitem[402]{key402} Kanehisa, M. and Goto, S.; 
\newblock KEGG: Kyoto Encyclopedia of Genes and Genomes. 
\newblock Nucleic Acids Res. 28, 27-30 (2000). 

\subsection{Machine Learning}

\subsection{Machine Learning Models : Classifiers}
\bibitem[1]{key1}Breiman, L. (2001). 
\newblock Random forests
\newblock Machine Learning, 45:5-32.

\bibitem[1]{key1}\href{}Multinomial Logistic Regression Algorithm
\bibitem[1]{key1}\href{}Fast Estimation of Multinomial Logit Models:R Package mnlogit
\bibitem[1]{key1}\href{}Variational Multinomial Logit Gaussian Process
\bibitem[1]{key1}\href{}Estimating Mixtures of Discrete Choice Model
\bibitem[1]{key1}\href{}Diagonal Orthant Multinomial Probit Models
\bibitem[1]{key1}\href{}A modified score function estimatorfor multinomial logistic regression in small samples
\bibitem[1]{key1}\href{}MaximumLikelihood Estimation of Logistic Regression Models:Theory and Implementation
\bibitem[1]{key1}\href{}A Short Introduction into Multinomial Probit Models
\bibitem[1]{key1}\href{}Multiple Choice Models--MNL and Nested Logit
\bibitem[1]{key1}\href{}Mixed Graphical Models via Exponential Families
\bibitem[1]{key1}\href{}Graphical Models and Exponential Families
\bibitem[1]{key1}\href{}Estimation of High-Dimensional Graphical Models Using Regularized Score Matching
\bibitem[1]{key1}\href{}Graphical Models, Exponential Families, and Variational Inference
\bibitem[1]{key1}\href{}Square Root Graphical Models: Multivariate Generalizations ofUnivariate Exponential Families that Permit Positive Dependencies
\bibitem[1]{key1}\href{}A Note on the Lasso for Gaussian Graphical Model Selection

\subsection{Machine Learning Models : Clusters}

\subsection{Resampling Strategies}

\subsection{Support Vector Models}

\subsection{Deep Learning Models}

\section{Machine Learning Algorithms}

\section{Variable Importance}

\bibitem[12]{key12}Ishwaran, H. (2007). 
\newblock Variable importance in binary regression trees and forests. 
\newblock Electronic J. Statist., 1, 519-537.

\bibitem[1000]{key1000}R Core Team (2015). 
\newblock R: A language and environment for statistical computing. R Foundation for Statistical Computing, Vienna, Austria.
\newblock URL https://www.R-project.org/.

\end{thebibliography}
\end{document}


