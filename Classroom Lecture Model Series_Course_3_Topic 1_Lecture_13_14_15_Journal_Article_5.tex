%-------------------------------------------------------------------------%
%--------------------Classroom Lecture Model Series-----------------------%
%-------------------------------------------------------------------------%

\begin{document}
\twocolumn
\scriptsize
\begin{frontmatter}
		\title{}
		\author{\corref{cor1}\fnref{fn1}}
		\cortext[cor1]{Corresponding author}
		\address{The Mathematical Learning Space}
		\ead{http://mathlearningspace.weebly.com}	
\end{frontmatter}	

Introduction:
\begin{enumerate}
\item Objective 1:
\item Objective 2:
\item Objective 3:
\end{enumerate}
Conclusion:

Keywords: Machine Learning, Deep learning, Parallel Programming, ROC Analysis 
Vocabulary Words:

\section{Introduction}

\subsection{Plan of the Article}

\section{Topic Review: Machine Learning}

\section{Machine Learning Models : Classifiers}

\section{Machine Learning Models : Clusters}

\subsection{Resampling Strategies}

\subsection{Support Vector Models}

\subsection{Deep Learning Models}

\section{Machine Learning Algorithms}

\subsection{Algorithm A}

\subsection{Parallel Programming}

\subsection{Tables}

\subsubsection{Prediction Probability Matrix}

\subsection{Figures}

\subsubsection{Performance Measures}

\subsubsection{Learning Curves}

\subsubsection{ROC Curves}

\section{Model Comparisons}

\section{Conclusions}


\section{R Application Programming Interfaces (APIs)}




\bibliographystyle{plain}
\begin{thebibliography}{00}

\bibitem[1000]{key1000}R Core Team (2015). 
\newblock R: A language and environment for statistical computing. R Foundation for Statistical Computing, Vienna, Austria.
\newblock URL https://www.R-project.org/.

\end{thebibliography}
\end{document}
