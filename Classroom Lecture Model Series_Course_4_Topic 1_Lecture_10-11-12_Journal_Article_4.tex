%-------------------------------------------------------------------------%
%--------------------Classroom Lecture Model Series-----------------------%
%-------------------------------------------------------------------------%


\begin{document}
\twocolumn
\scriptsize
\begin{frontmatter}
		\title{DRAFT Sample: Categorical Temporal Transformation Designs for Smoothing Models of Jazz Compositional Waveforms}
		\author{\corref{cor1}\fnref{fn1}}
		\cortext[cor1]{Corresponding author}
		\address{The Mathematical Learning Space}
		\ead{http://mathlearningspace.weebly.com}	
\end{frontmatter}	

Introduction:
\begin{enumerate}
\item Objective 1:
\item Objective 2:
\item Objective 3:
\end{enumerate}
Conclusion:

keywords: Wavlet Transforms, Music Theory, Arrangements, Harmmonic Classification, Sinc function

\section{Introduction}


\subsection{Plan of the Article}

\begin{enumerate}
\item Review of Music Theory and Literature Review
\item Mathematical Review of Wavelet Models of Compositions
\item Multi-resolution Analysis 
\item Frequency Modeling
\item Presentation of the Results
\end{enumerate}

\section{Music Theory Review}

\begin{table}[H]
	\caption{Composition Motifs in Album, Duration and Acoustic Complexity}	
	\begin{tabular}{p{1cm}p{4cm}p{2cm}p{2cm}}
	\hline
	Track No & Name & Duration (seconds) & Acoustic Complexity\\
	\hline
	\hline 
	\end{tabular}
\end{table}

\section{Motif Collection 1}

\section{Motif Collection 2}

\section{Motif Collection 3}

\section{Composition Example}

%--------------Measure 1- 4------------------------------------------
\subsection{Measures 1-4}

\begin{music}
\parindent10mm\instrumentnumber{1}
\setname1{Piano}                  
\setstaffs1{2}                     
\generalmeter{\meterfrac44}        
\startextract
%--------Measure 1
\Notes\qa{cccc} | \qa{cdcd}\en
\bar
%--------Measure 2
\Notes \qa{cccc} | \qa{cdcd} \en
\bar
%--------Measure 3
\Notes \qa{cccc} | \qa{aaaa} \en
\bar
%--------Measure 4
\Notes\qa{cdcd} | \qa{cccc}\en
\endextract
\end{music}


%--------------Measure 5- 8------------------------------------------
\subsection{Measures 5-8}

%--------------Measure 9- 12------------------------------------------
\subsection{Measures 9-12}

%--------------Measure 13- 16------------------------------------------
\subsection{Measures 13-16}

%--------------Measure 17- 20------------------------------------------
\subsection{Measures 17-20}

%--------------Measure 21- 24------------------------------------------
\subsection{Measures 21-24}

%--------------Measure 25- 28------------------------------------------
\subsection{Measures 25-28}

%--------------Measure 29- 32------------------------------------------
\subsection{Measures 29-32}


\subsection{Topic A: Wavelets In Music}

\centering	
\begin{table}[H]\tiny
	\caption{}	
	\begin{tabular}{r|p{4cm}|l}
		\hline	
		Topic & Description & Reference \\
		\hline 
		\hline 
	\end{tabular}
\end{table}

\section{Mathematical Review}

\subsection{Topic Review: Wavelet Models}

\centering	
\begin{table}[H]\tiny
	\caption{}	
	\begin{tabular}{r|p{4cm}|ll}
		\hline	
		Model & Description & Results & Reference \\
		\hline 
		\hline 
	\end{tabular}
\end{table}

\begin{figure}[H]
	\centering
	\begin{minipage}[b]{0.5\linewidth}
	%\includegraphics[scale=0.25]{Example_1_Figure_1.png}
	\end{minipage}\hfill
	\begin{minipage}[b]{0.5\linewidth}
	%\includegraphics[scale=0.25]{Example_1_Figure_2.png}
	\end{minipage}\hfill	
	\begin{minipage}[b]{0.5\linewidth}
	%\includegraphics[scale=0.25]{Example_1_Figure_3.png}
	\end{minipage}\hfill
	\begin{minipage}[b]{0.5\linewidth}
	%\includegraphics[scale=0.25]{Example_1_Figure_4.png}
	\end{minipage}\hfill
	\caption{1, 2, 3 and 4}
	\label{fig:Figure1}
\end{figure} 

\section{Results}

\subsection{Multiresolution Analysis}

\centering	
\begin{table}[H]\tiny
	\caption{}	
	\begin{tabular}{r|p{4cm}|l}
		\hline	
		Scale & Description & Results \\
		\hline 
		\hline 
	\end{tabular}
\end{table}

\begin{figure}[H]
	\centering
	\begin{minipage}[b]{0.5\linewidth}
	%\includegraphics[scale=0.25]{Example_2_Figure_1.png}
	\end{minipage}\hfill
	\begin{minipage}[b]{0.5\linewidth}
	%\includegraphics[scale=0.25]{Example_2_Figure_2.png}
	\end{minipage}\hfill	
	\begin{minipage}[b]{0.5\linewidth}
	%\includegraphics[scale=0.25]{Example_2_Figure_3.png}
	\end{minipage}\hfill
	\begin{minipage}[b]{0.5\linewidth}
	%\includegraphics[scale=0.25]{Example_2_Figure_4.png}
	\end{minipage}\hfill
	\caption{1, 2, 3 and 4}
	\label{fig:Figure1}
\end{figure} 


\subsubsection{Tables}

\centering	
\begin{table}[H]\tiny
	\caption{}	
	\begin{tabular}{r|p{4cm}p{1cm}}
		\hline	
		ID & Model & Description \\
		\hline 
		\hline 
	\end{tabular}
\end{table}

\subsubsection{Figures}

\begin{figure}[H]
	\centering
	\begin{minipage}[b]{0.5\linewidth}
	%\includegraphics[scale=0.25]{Example_3_Figure_1.png}
	\end{minipage}\hfill
	\begin{minipage}[b]{0.5\linewidth}
	%\includegraphics[scale=0.25]{Example_3_Figure_2.png}
	\end{minipage}\hfill	
	\begin{minipage}[b]{0.5\linewidth}
	%\includegraphics[scale=0.25]{Example_3_Figure_3.png}
	\end{minipage}\hfill
	\begin{minipage}[b]{0.5\linewidth}
	%\includegraphics[scale=0.25]{Example_3_Figure_4.png}
	\end{minipage}\hfill
	\caption{1, 2, 3 and 4}
	\label{fig:Figure1}
\end{figure} 


\section{Additional Topics for the Classroom}

\begin{enumerate}
\item Additional Wavelet Design Models
\end{enumerate}

\section{Appendix: Complete Composition}


\bibliographystyle{plain}
\begin{thebibliography}{00}
\bibitem[1]{key100}Sueur J., Aubin T., Simonis C. (2008). 
\newblock Seewave: a free modular tool for sound analysis and synthesis. 
\newblock Bioacoustics, 18: 213-226.

\bibitem[2]{key101}Uwe Ligges, Sebastian Krey, Olaf Mersmann, and Sarah Schnackenberg (2016). 
\newblock tuneR: Analysis of music. 
\newblock URL: http://r-forge.r-project.org/projects/tuner/.

\bibitem[3]{key102}A. Anikin (2017). 
\newblock soundgen: Parametric Voice Synthesis. 
\newblock R package version 1.1.1.

\bibitem[4]{key103}Pieretti N, Farina A, Morri FD (2011) 
\newblock A new methodology to infer the singing activity of an avian community: the Acoustic Complexity Index (ACI). 
\newblock Ecological Indicators, 11, 868-873.
\end{thebibliography}

\end{thebibliography}

\end{document}
