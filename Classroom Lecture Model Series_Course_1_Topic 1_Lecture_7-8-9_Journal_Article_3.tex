%-------------------------------------------------------------------------%
%--------------------Classroom Lecture Model Series-----------------------%
%-------------------------------------------------------------------------%

\begin{document}
\twocolumn
\scriptsize
\begin{frontmatter}
		\title{}
		\author{\corref{cor1}\fnref{fn1}}
		\cortext[cor1]{Corresponding author}
		\address{The Mathematical Learning Space}
		\ead{http://mathlearningspace.weebly.com}	
\end{frontmatter}	

Introduction:
\begin{enumerate}
\item Objective 1:
\item Objective 2:
\item Objective 3:
\end{enumerate}
Conclusion:

Keywords: Stochastic Differential Equations, Bayesian Markov Chain Monte Carlo, Gene Ontology, VEGFC, Lymphangiogenesis
Vocabulary Words:

\section{Introduction}

\subsection{Plan of the Article}


\section{Topic Review}

\begin{table}[H]
\centering
\begin{tabular}{r|p{12cm}|l}
\hline
PubmedID  & Title & Summary \\
\hline	
\hline
	\end{tabular}
	\caption{Summary of Literature Review for Topic}
\end{table}		

\subsection{GO: Biological Process}

\subsection{GO: Molecular Function}

\subsection{GO: Cellular Component}

\section{Mathematical Background}

\subsection{Bayesian inference}

\subsubsection{Uniform Distribution}

\subsubsection{Normal, Beta, and Gamma Distributions}

\subsection{Bayesian MCMC}

\section{Equation Systems}

\subsection{Drift and Diffusion Parameter Specifications}

\subsection{Error Term Specifications}

\section{Results}

\begin{enumerate}
	\item Model A
	\item Model B
	\item Model C
\end{enumerate}

\subsection{Maximum Likelihood Estimation}

\begin{table}[H]
	\centering
	\small
	\begin{tabular}{rrrr}
		\hline
		& Model A & Model B & Model C \\ 
		\hline
		\hline
	\end{tabular}
\end{table}

\subsection{Tables}

\begin{table}[h]\footnotesize
	\small
	\caption{Parameter Description and Value}
	\begin{tabular}{rp{2cm}p{2cm}p{2cm}}
\hline
Parameter & Prior 1 & Prior 2 & Prior 3 \\
\hline
\hline
\end{tabular}	
\end{table}

\subsubsection{Lead-Lag Relationships}


\subsection{Figures}

\begin{figure}[H]
	\centering
	\begin{minipage}[b]{0.5\linewidth}
	%\includegraphics[scale=0.25]{Example_1_Figure_1.png}
	\end{minipage}\hfill
	\begin{minipage}[b]{0.5\linewidth}
	%\includegraphics[scale=0.25]{Example_1_Figure_2.png}
	\end{minipage}\hfill	
	\begin{minipage}[b]{0.5\linewidth}
	%\includegraphics[scale=0.25]{Example_1_Figure_3.png}
	\end{minipage}\hfill
	\begin{minipage}[b]{0.5\linewidth}
	%\includegraphics[scale=0.25]{Example_1_Figure_4.png}
	\end{minipage}\hfill
	\caption{1, 2, 3 and 4}
	\label{fig:Figure1}
\end{figure} 


\section{Conclusions and Discussions}

\begin{table}[H]
	\centering
	\begin{tabular}{rp{1cm}p{1cm}p{1cm}p{1cm}p{1cm}p{1cm}}
		\hline
		& Model A-MLE & Model A-Bayes & Model B-MLE & Model B-Bayes & Model C-MLE & Model C-Bayes \\ 
		\hline
		\hline
	\end{tabular}
\end{table}


\section{R Application Programming Interfaces (APIs)}


\bibliographystyle{plain}
\begin{thebibliography}{00}
		
\bibitem[101]{key101}Yoshida, N. (2011). 
\newblock Polynomial type large deviation inequalities and quasi-likelihood analysis for stochastic differential equations. 
\newblock Annals of the Institute of Statistical Mathematics, 63(3), 431-479.

\bibitem[102]{key102}Uchida, M., and Yoshida, N. (2014). 
\newblock Adaptive Bayes type estimators of ergodic diffusion processes from discrete observations. 
\newblock Statistical Inference for Stochastic Processes, 17(2), 181-219.

\bibitem[103]{key103}Kamatani, K. (2017).
\newblock Ergodicity of Markov chain Monte Carlo with reversible proposal. 
\newblock Journal of Applied Probability, 54(2).
	
\bibitem[104]{key104}Barndorff-Nielsen, O. E. and Shephard, N. (2004) 
\newblock Power and bipower variation with stochastic volatility and jumps, 
\newblock Journal of Financial Econometrics, 2, no. 1, 1–37.
	
\bibitem[105]{key105}Barndorff-Nielsen, O. E. and Shephard, N. (2006) 
\newblock Econometrics of testing for jumps in financial economics using bipower variation, 
\newblock Journal of Financial Econometrics, 4, no. 1, 1–30.
	
\bibitem[106]{key106}Huang, X. and Tauchen, G. (2005) 
\newblock The relative contribution of jumps to total price variance, 
\newblock Journal of Financial Econometrics, 3, no. 4, 456–499.

\bibitem[201]{key201}Wikipedia contributors. 
\newblock Michaelis–Menten kinetics. 
\newblock Wikipedia, The Free Encyclopedia. Wikipedia.

\bibitem[202]{key202}Wikipedia contributors. 
\newblock Hill equation (biochemistry).
\newblock Wikipedia, The Free Encyclopedia. Wikipedia. 

\bibitem[203]{key203}Wikipedia contributors. 
\newblock Stochastic differential equation.
\newblock Wikipedia, The Free Encyclopedia. Wikipedia.

\bibitem[204]{key204}Wikipedia contributors. 
\newblock Markov chain Monte Carlo.
\newblock Wikipedia, The Free Encyclopedia. Wikipedia.

\bibitem[205]{key205}Wikipedia contributors. 
\newblock Generalised logistic function." 
\newblock Wikipedia, The Free Encyclopedia. Wikipedia. 

\bibitem[206]{key206}Wikipedia contributors. 
\newblock Sigmoid function." 
\newblock Wikipedia, The Free Encyclopedia. Wikipedia. 

\bibitem[207]{key207}Wikipedia contributors. 
\newblock Heaviside step function.
\newblock Wikipedia, The Free Encyclopedia. Wikipedia.
\end{thebibliography}
\end{document}
