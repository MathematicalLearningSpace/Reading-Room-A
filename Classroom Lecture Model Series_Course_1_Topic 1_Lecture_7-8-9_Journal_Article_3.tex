%-------------------------------------------------------------------------%
%--------------------Classroom Lecture Model Series-----------------------%
%-------------------------------------------------------------------------%

\begin{document}
\twocolumn
\scriptsize
\begin{frontmatter}
		\title{}
		\author{\corref{cor1}\fnref{fn1}}
		\cortext[cor1]{Corresponding author}
		\address{The Mathematical Learning Space}
		\ead{http://mathlearningspace.weebly.com}	
\end{frontmatter}	

Introduction:
\begin{enumerate}
\item Objective 1:
\item Objective 2:
\item Objective 3:
\end{enumerate}
Conclusion:

Keywords: Stochastic Differential Equations, Bayesian Markov Chain Monte Carlo, Gene Ontology, VEGFC, Lymphangiogenesis
Vocabulary Words:

\section{Introduction}

\begin{table}[H]\centering
	\begin{tabular}{p{1cm}p{4cm}p{3cm}}
		Article ID & Summary & Comments\\
		\hline
		\hline
	\end{tabular}
\end{table}

\subsection{Plan of the Article}

\begin{enumerate}
\end{enumerate}

\section{Topic Review}

\begin{table}[H]
\centering
\begin{tabular}{r|p{12cm}|l}
\hline
PubmedID  & Title & Summary \\
\hline	
\hline
	\end{tabular}
	\caption{Summary of Literature Review for Topic}
\end{table}	

\centering	
\begin{table}[H]\tiny
\caption{Gene/Protein/Enzymes Description and References}	
\begin{tabular}{r|p{3cm}|l}
\hline	
Gene/Protein & Description & Reference \\
\hline 
\hline 
\end{tabular}
\end{table}


\subsection{Gene Ontology: Biological Process}

\begin{table}[H]\centering
	\begin{tabular}{p{1cm}p{4cm}p{3cm}}
		Article ID & Summary & Comments\\
		\hline
		\hline
	\end{tabular}
\end{table}

\subsection{Gene Ontology: Molecular Function}

\begin{table}[H]\centering
	\begin{tabular}{p{1cm}p{4cm}p{3cm}}
		Article ID & Summary & Comments\\
		\hline
		\hline
	\end{tabular}
\end{table}

\subsection{Gene Ontology: Cellular Component}

\begin{table}[H]\centering
	\begin{tabular}{p{1cm}p{4cm}p{3cm}}
		Article ID & Summary & Comments\\
		\hline
		\hline
	\end{tabular}
\end{table}

\section{Mathematical Background}

\begin{enumerate}
\end{enumerate}

\subsection{Bayesian inference}

\begin{equation}
\end{equation}

\subsection{Stochastic Integration}

\begin{equation}
\end{equation}

\subsubsection{Uniform Distribution}

\subsubsection{Normal, Beta, and Gamma Distributions}

\begin{table}[H]\centering
	\begin{tabular}{p{1cm}p{6cm}p{2cm}}
		CDF ID & Equation Summary & Comments\\
		\hline
		\hline
	\end{tabular}
\end{table}

\subsection{Bayesian MCMC}

\begin{equation}
\end{equation}

\section{Equation Systems}

\centering	
\begin{table}[H]\tiny
  \caption{Stochastic Processes and Paths }	
\begin{tabular}{r|p{0.5cm}p{0.5cm}p{2.5cm}|p{1cm}|p{1cm}}
\hline	
Stochastic Process & Start & Time & Description & Reference \\
\hline 
\hline 
\end{tabular}
\end{table}


\subsection{Drift and Diffusion Parameter Specifications}

\begin{table}[H]\centering
	\begin{tabular}{p{1cm}p{4cm}p{3cm}}
		EquationID & Description & Comments\\
		\hline
		\hline
	\end{tabular}
\end{table}

\subsection{Error Term Specifications}

\begin{table}[H]\centering
	\begin{tabular}{p{1cm}p{4cm}p{3cm}}
		Error ID & Summary & Comments\\
		\hline
		\hline
	\end{tabular}
\end{table}

\section{Results}

\begin{enumerate}
	\item Model A
	\item Model B
	\item Model C
\end{enumerate}

\subsection{Maximum Likelihood Estimation}

\begin{table}[H]
	\centering
	\small
	\begin{tabular}{rrrr}
		\hline
		& Model A & Model B & Model C \\ 
		\hline
		\hline
	\end{tabular}
\end{table}

\subsection{Tables}

\begin{table}[h]\footnotesize
	\small
	\caption{Parameter Description and Value}
	\begin{tabular}{rp{2cm}p{2cm}p{2cm}}
\hline
Parameter & Prior 1 & Prior 2 & Prior 3 \\
\hline
\hline
\end{tabular}	
\end{table}

\subsubsection{Lead-Lag Relationships}

\begin{table}[H]\centering
	\begin{tabular}{p{1cm}p{4cm}p{3cm}}
		Lead/Lag & Summary & Comments\\
		\hline
		\hline
	\end{tabular}
\end{table}

\subsection{Figures}

\subsubsection{Group A}

\begin{figure}[H]
	\centering
	\begin{minipage}[b]{0.5\linewidth}
	%\includegraphics[scale=0.25]{Example_1_Figure_1.png}
	\end{minipage}\hfill
	\begin{minipage}[b]{0.5\linewidth}
	%\includegraphics[scale=0.25]{Example_1_Figure_2.png}
	\end{minipage}\hfill	
	\begin{minipage}[b]{0.5\linewidth}
	%\includegraphics[scale=0.25]{Example_1_Figure_3.png}
	\end{minipage}\hfill
	\begin{minipage}[b]{0.5\linewidth}
	%\includegraphics[scale=0.25]{Example_1_Figure_4.png}
	\end{minipage}\hfill
	\caption{1, 2, 3 and 4}
	\label{fig:Figure1}
\end{figure} 

\subsubsection{Group B}

\begin{figure}[H]
	\centering
	\begin{minipage}[b]{0.5\linewidth}
	%\includegraphics[scale=0.25]{Example_1_Figure_5.png}
	\end{minipage}\hfill
	\begin{minipage}[b]{0.5\linewidth}
	%\includegraphics[scale=0.25]{Example_1_Figure_6.png}
	\end{minipage}\hfill	
	\begin{minipage}[b]{0.5\linewidth}
	%\includegraphics[scale=0.25]{Example_1_Figure_7.png}
	\end{minipage}\hfill
	\begin{minipage}[b]{0.5\linewidth}
	%\includegraphics[scale=0.25]{Example_1_Figure_8.png}
	\end{minipage}\hfill
	\caption{5, 6, 7 and 8}
	\label{fig:Figure1}
\end{figure} 


\section{Conclusions and Discussions}

\begin{table}[H]
	\centering
	\begin{tabular}{rp{1cm}p{1cm}p{1cm}p{1cm}p{1cm}p{1cm}}
		\hline
		& Model A-MLE & Model A-Bayes & Model B-MLE & Model B-Bayes & Model C-MLE & Model C-Bayes \\ 
		\hline
		\hline
	\end{tabular}
\end{table}

\begin{table}[H]\centering
	\begin{tabular}{p{1cm}p{4cm}p{3cm}}
		Topic ID & Summary & Comments\\
		\hline
		\hline
	\end{tabular}
\end{table}


\section{R Application Programming Interfaces (APIs)}


\bibliographystyle{plain}
\begin{thebibliography}{00}

\bibitem[1]{key1}Arsalane Chouaib Guidoum and Kamal Boukhetala (2017). 
\newblock Sim.DiffProc: Simulation of Diffusion Processes.
\newblock R package version 3.8.https://cran.r-project.org/package=Sim.DiffProc

\bibitem[2]{key2}Allen, E. (2007). 
\newblock Modeling with Ito stochastic differential equations.
\newblock Springer-Verlag, New York.

\bibitem[3]{key3}Jedrzejewski, F. (2009). 
\newblock Modeles aleatoires et physique probabiliste. 
\newblock Springer-Verlag, New York.

\bibitem[4]{key4}Henderson, D and Plaschko, P. (2006). 
\newblock Stochastic differential equations in science and engineering. 
\newblock World Scientific.

\bibitem[5]{key5}Bladt, M. and Sorensen, M. (2007). 
\newblock Simple simulation of diffusion bridges with application to likelihood inference for diffusions. 
Working Paper, University of Copenhagen.

\bibitem[6]{key6}Ito, K. (1944). 
\newblock Stochastic integral. 
\newblock Proc. Jap. Acad, Tokyo, 20, 19–529.

\bibitem[7]{key7}Stratonovich RL (1966). 
\newblock New Representation for Stochastic Integrals and Equations. 
\newblock SIAM Journal on Control, 4(2), 362–371.

\bibitem[8]{key8}Kloeden, P.E, and Platen, E. (1995). 
\newblock Numerical Solution of Stochastic Differential Equations. 
\newblock Springer-Verlag, New York.

\bibitem[9]{key9}Oksendal, B. (2000). 
\newblock Stochastic Differential Equations: An Introduction with Applications. 
\newblock 5th edn. Springer-Verlag, Berlin.

\bibitem[10]{key10}Boukhetala K (1996). 
\newblock Modelling and Simulation of a Dispersion Pollutant with Attractive Centre, volume 3, pp. 245-252. 
\newblock Computer Methods and Water Resources, Computational Mechanics Publications, Boston, USA.

\bibitem[11]{key11}Boukhetala K (1998). 
\newblock Estimation of the first passage time distribution for a simulated diffusion process. 
\newblock Maghreb Mathematical Review, 7, pp. 1-25.

\bibitem[12]{key12}Boukhetala K (1998). 
\newblock Kernel density of the exit time in a simulated diffusion. 
\newblock The Annals of The Engineer Maghrebian, 12, pp. 587-589.

\bibitem[13]{key13}Guidoum AC, Boukhetala K (2017). 
\newblock Performing Parallel Monte Carlo and Moment Equations Methods for Itô and Stratonovich Stochastic Differential Systems: R Package Sim.DiffProc. 
\newblock Preprint submitted to Journal of Statistical Software.

\bibitem[14]{key14}Guidoum AC, Boukhetala K (2017). 
\newblock Sim.DiffProc: Simulation of Diffusion Processes. 
\newblock R package version 3.8, URL https://cran.r-project.org/package=Sim.DiffProc.

\bibitem[15]{key15}Pienaar EAD, Varughese MM (2016). 
\newblock DiffusionRgqd: An R Package for Performing Inference and Analysis on Time-Inhomogeneous Quadratic Diffusion Processes. 
\newblock R package version 0.1.3, URL https://CRAN.R-project.org/package=DiffusionRgqd.

\bibitem[16]{key16}Roman, R.P., Serrano, J. J., Torres, F. (2008). 
\newblock First-passage-time location function: Application to determine first-passage-time densities in diffusion processes. 
\newblock Computational Statistics and Data Analysis. 52, 4132-4146.

\bibitem[17]{key17}Roman, R.P., Serrano, J. J., Torres, F. (2012). 
\newblock An R package for an efficient approximation of first-passage-time densities for diffusion processes based on the FPTL function. 
\newblock Applied Mathematics and Computation, 218, 8408-8428.

\bibitem[18]{key18}Fox, J. and Weisberg, S. (2011) 
\newblock An R Companion to Applied Regression
\newblock Second Edition, Sage.

\bibitem[19]{key19}Iacus, S.M. (2008). 
\newblock Simulation and inference for stochastic differential equations: with R examples. 
\newblock Springer-Verlag, New York.
		
\bibitem[101]{key101}Yoshida, N. (2011). 
\newblock Polynomial type large deviation inequalities and quasi-likelihood analysis for stochastic differential equations. 
\newblock Annals of the Institute of Statistical Mathematics, 63(3), 431-479.

\bibitem[102]{key102}Uchida, M., and Yoshida, N. (2014). 
\newblock Adaptive Bayes type estimators of ergodic diffusion processes from discrete observations. 
\newblock Statistical Inference for Stochastic Processes, 17(2), 181-219.

\bibitem[103]{key103}Kamatani, K. (2017).
\newblock Ergodicity of Markov chain Monte Carlo with reversible proposal. 
\newblock Journal of Applied Probability, 54(2).
	
\bibitem[104]{key104}Barndorff-Nielsen, O. E. and Shephard, N. (2004) 
\newblock Power and bipower variation with stochastic volatility and jumps, 
\newblock Journal of Financial Econometrics, 2, no. 1, 1–37.
	
\bibitem[105]{key105}Barndorff-Nielsen, O. E. and Shephard, N. (2006) 
\newblock Econometrics of testing for jumps in financial economics using bipower variation, 
\newblock Journal of Financial Econometrics, 4, no. 1, 1–30.
	
\bibitem[106]{key106}Huang, X. and Tauchen, G. (2005) 
\newblock The relative contribution of jumps to total price variance, 
\newblock Journal of Financial Econometrics, 3, no. 4, 456–499.

\bibitem[201]{key201}Wikipedia contributors. 
\newblock Michaelis–Menten kinetics. 
\newblock Wikipedia, The Free Encyclopedia. Wikipedia.

\bibitem[202]{key202}Wikipedia contributors. 
\newblock Hill equation (biochemistry).
\newblock Wikipedia, The Free Encyclopedia. Wikipedia. 

\bibitem[203]{key203}Wikipedia contributors. 
\newblock Stochastic differential equation.
\newblock Wikipedia, The Free Encyclopedia. Wikipedia.

\bibitem[204]{key204}Wikipedia contributors. 
\newblock Markov chain Monte Carlo.
\newblock Wikipedia, The Free Encyclopedia. Wikipedia.

\bibitem[205]{key205}Wikipedia contributors. 
\newblock Generalised logistic function." 
\newblock Wikipedia, The Free Encyclopedia. Wikipedia. 

\bibitem[206]{key206}Wikipedia contributors. 
\newblock Sigmoid function." 
\newblock Wikipedia, The Free Encyclopedia. Wikipedia. 

\bibitem[207]{key207}Wikipedia contributors. 
\newblock Heaviside step function.
\newblock Wikipedia, The Free Encyclopedia. Wikipedia.
\end{thebibliography}

\end{document}
