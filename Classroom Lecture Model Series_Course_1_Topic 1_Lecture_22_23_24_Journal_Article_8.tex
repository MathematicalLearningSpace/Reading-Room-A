%-------------------------------------------------------------------------%
%--------------------Classroom Lecture Model Series-----------------------%
%-------------------------------------------------------------------------%

\begin{document}
\twocolumn
\scriptsize
\begin{frontmatter}
		\title{DRAFT Sample: Stability Classification Designs for Differential Equation Systems of DNA, Protein and Compound Interaction Combinatorics}
		\author{\corref{cor1}\fnref{fn1}}
		\cortext[cor1]{Corresponding author}
		\address{The Mathematical Learning Space}
		\ead{http://mathlearningspace.weebly.com}	
\end{frontmatter}	

Introduction:
\begin{enumerate}
\item Objective 1:
\item Objective 2:
\item Objective 3:
\end{enumerate}
Conclusion:

Keywords: Global Asymptotic Stability, Stochastic Differential Equations, Transition Probability
Vocabulary Words:

\section{Introduction}

\centering
\begin{table}[H]\footnotesize
	\caption{}
	\begin{tabular}{rp{1cm}p{2cm}p{3cm}p{1cm}}
		\hline
		ID & A & B & C & Reference \\
		\hline
		\hline
	\end{tabular}
\end{table}
\raggedright


\subsection{Plan of the Article}

\begin{enumerate}
\item Biological Background of Different BioMathematical Functions
\item Mathematical Background of Ordinary and Stochastic Differential Equation Systems
\item Transition Probability
\item Global Asymptotic Stability
\item V Ergodic Stability
\item Convergence to Stationary Distribution with Density
\end{enumerate}

\section{Protein Interaction Diagram and Annotations}

\begin{tikzpicture}
	[->,>=stealth',shorten >=2pt,node distance=3cm,
	thick,main node/.style={circle,draw,scale=0.25,transform canvas={scale=0.75},font=\sffamily\Small\bfseries},
	blacknode/.style={shape=circle, draw=black, line width=2},
	bluenode/.style={shape=circle, draw=blue, line width=2},
	greennode/.style={shape=circle, draw=green, line width=2},
	rednode/.style={shape=circle, draw=red, line width=2}
	]
	%-------Legend ------------------------------------------------------------	
	\matrix [draw,below left] at (current bounding box.south) {
		\node [state,label=right:State] {}; Description. \\
		\node [shapeSquare,label=right:Square-.] {}; \\
		\node [shapeEllipse,label=right:Ellipse-.] {}; \\
		\node [shapeTriangle,label=right:Triangle-.] {}; \\
		\node [shapeHexagon,label=right:Hexagon-.] {}; \\
	};
\end{tikzpicture}

\centering	
\begin{table}[H]\tiny
\caption{Gene/Protein/Enzymes Description and References}	
\begin{tabular}{r|p{3cm}|l}
\hline	
Gene/Protein & Description & Reference \\
\hline 
\hline 
\end{tabular}
\end{table}


\section{Background : Mathematical}

\subsection{Equation System A}

\begin{align*} 
\tiny
\frac{d^{\alpha_1}X_1(t)}{dt^{\alpha_1}} = a_{11} *X_1(t - \tau_1) + \\
a_{12} *\frac{X_1(t)^{\beta_1}}{(1-X_1(t - \tau_1))^{\beta_1}} - a_{15}X_1(t) + \\
\epsilon_1(t) \\
\frac{d^{\alpha_1}X_2(t)}{dt^{\alpha_1}} = a_{21} *X_2(t - \tau_2) + \\
a_{22} *\frac{X_2(t)^{\beta_1}}{(1-X_2(t - \tau_2))^{\beta_1}} - a_{25}X_2(t) + \\
\epsilon_2(t) \\ \\
\frac{d^{\alpha_1}X_3(t)}{dt^{\alpha_1}} = a_{31} *X_3(t - \tau_3) + \\
a_{32} *\frac{X_3(t)^{\beta_1}}{(1-X_3(t - \tau_3))^{\beta_1}} - a_{35}X_3(t) + \\
\epsilon_3(t) \\ \\
\frac{d^{\alpha_1}X_4(t)}{dt^{\alpha_1}} = a_{41} *X_4(t - \tau_4) + \\
a_{42} *\frac{X_4(t)^{\beta_1}}{(1-X_4(t - \tau_4))^{\beta_1}} - a_{45}X_4(t) + \\
\epsilon_4(t) \\ \\
\frac{d^{\alpha_1}X_5(t)}{dt^{\alpha_1}} = a_{51} *X_5(t - \tau_5) + \\
a_{52} *\frac{X_5(t)^{\beta_1}}{(1-X_5(t - \tau_5))^{\beta_1}} - a_{55}X_5(t) + \\
\epsilon_5(t) \\
\end{align*}

\subsubsection{Parameter Table}

\begin{table}[h]\footnotesize
	\caption{Parameter Description and Value}
	\begin{tabular}{rllp{2cm}l}
		\hline	
		Parameter & Value & Interval & Description & Reference \\
		\hline 
		a11 & 0 & [0,1] & Equation 1 & \cite{key1}
		a12 & 0 & [0,1] & Equation 1 & \cite{key1}
		a13 & 0 & [0,1] & Equation 1 & \cite{key1}
		a14 & 0 & [0,1] & Equation 1 & \cite{key1}
		a15 & 0 & [0,1] & Equation 1 & \cite{key1}
		\hline
		a21 & 0 & [0,1] & Equation 2 & \cite{key1}
		a22 & 0 & [0,1] & Equation 2 & \cite{key1}
		a23 & 0 & [0,1] & Equation 2 & \cite{key1}
		a24 & 0 & [0,1] & Equation 2 & \cite{key1}
		a25 & 0 & [0,1] & Equation 2 & \cite{key1}
		\hline
		a31 & 0 & [0,1] & Equation 3 & \cite{key1}
		a32 & 0 & [0,1] & Equation 3 & \cite{key1}
		a33 & 0 & [0,1] & Equation 3 & \cite{key1}
		a34 & 0 & [0,1] & Equation 3 & \cite{key1}
		a35 & 0 & [0,1] & Equation 3 & \cite{key1}
		\hline
		a41 & 0 & [0,1] & Equation 4 & \cite{key1}
		a42 & 0 & [0,1] & Equation 4 & \cite{key1}
		a43 & 0 & [0,1] & Equation 4 & \cite{key1}
		a44 & 0 & [0,1] & Equation 4 & \cite{key1}
		a45 & 0 & [0,1] & Equation 4 & \cite{key1}
		\hline
		a51 & 0 & [0,1] & Equation 5 & \cite{key1}
		a52 & 0 & [0,1] & Equation 5 & \cite{key1}
		a53 & 0 & [0,1] & Equation 5 & \cite{key1}
		a54 & 0 & [0,1] & Equation 5 & \cite{key1}
		a55 & 0 & [0,1] & Equation 5 & \cite{key1}
		\hline
		$\tau_1$ & 1 & [1,2] & Equation 1 & \cite{key1}
		$\tau_2$ & 1 & [1,2] & Equation 2 & \cite{key1}
		$\tau_3$ & 1 & [1,2] & Equation 3 & \cite{key1}
		$\tau_4$ & 1 & [1,2] & Equation 4 & \cite{key1}
		$\tau_5$ & 1 & [1,2] & Equation 5 & \cite{key1}
		\hline
		$\alpha_1$ & 1 & (0,2] & Equation 1 & \cite{key1}
		$\alpha_2$ & 1 & (0,2] & Equation 2 & \cite{key1}
		$\alpha_3$ & 1 & (0,2] & Equation 3 & \cite{key1}
		$\alpha_4$ & 1 & (0,2] & Equation 4 & \cite{key1}
		$\alpha_5$ & 1 & (0,2] & Equation 5 & \cite{key1}
		\hline
		$\beta_1$ & 1 & (0,2] & Equation 1 & \cite{key1}
		$\beta_2$ & 1 & (0,2] & Equation 2 & \cite{key1}
		$\beta_3$ & 1 & (0,2] & Equation 3 & \cite{key1}
		$\beta_4$ & 1 & (0,2] & Equation 4 & \cite{key1}
		$\beta_5$ & 1 & (0,2] & Equation 5 & \cite{key1}
	\end{tabular}	
\end{table}

\subsection{Equation System B}

\subsection{Protein Interaction Diagram and Annotations}

\begin{tikzpicture}
	[->,>=stealth',shorten >=2pt,node distance=3cm,
	thick,main node/.style={circle,draw,scale=0.25,transform canvas={scale=0.75},font=\sffamily\Small\bfseries},
	blacknode/.style={shape=circle, draw=black, line width=2},
	bluenode/.style={shape=circle, draw=blue, line width=2},
	greennode/.style={shape=circle, draw=green, line width=2},
	rednode/.style={shape=circle, draw=red, line width=2}
	]
	%-------Legend ------------------------------------------------------------	
	\matrix [draw,below left] at (current bounding box.south) {
		\node [state,label=right:State] {}; Description. \\
		\node [shapeSquare,label=right:Square-.] {}; \\
		\node [shapeEllipse,label=right:Ellipse-.] {}; \\
		\node [shapeTriangle,label=right:Triangle-.] {}; \\
		\node [shapeHexagon,label=right:Hexagon-.] {}; \\
	};
\end{tikzpicture}

\centering	
\begin{table}[H]\tiny
\caption{Gene/Protein/Enzymes Description and References}	
\begin{tabular}{r|p{3cm}|l}
\hline	
Gene/Protein & Description & Reference \\
\hline 
\hline 
\end{tabular}
\end{table}

\begin{align*} 
\tiny
\frac{d^{\alpha_1}X_1(t)}{dt^{\alpha_1}} = a_{11} *X_1(t - \tau_1) + \\
a_{12} *\frac{X_1(t)^{\beta_1}}{(1-X_1(t - \tau_1))^{\beta_1}} - a_{15}X_1(t) + \\
\epsilon_1(t) \\
\frac{d^{\alpha_1}X_2(t)}{dt^{\alpha_1}} = a_{21} *X_2(t - \tau_2) + \\
a_{22} *\frac{X_2(t)^{\beta_1}}{(1-X_2(t - \tau_2))^{\beta_1}} - a_{25}X_2(t) + \\
\epsilon_2(t) \\ \\
\frac{d^{\alpha_1}X_3(t)}{dt^{\alpha_1}} = a_{31} *X_3(t - \tau_3) + \\
a_{32} *\frac{X_3(t)^{\beta_1}}{(1-X_3(t - \tau_3))^{\beta_1}} - a_{35}X_3(t) + \\
\epsilon_3(t) \\ \\
\frac{d^{\alpha_1}X_4(t)}{dt^{\alpha_1}} = a_{41} *X_4(t - \tau_4) + \\
a_{42} *\frac{X_4(t)^{\beta_1}}{(1-X_4(t - \tau_4))^{\beta_1}} - a_{45}X_4(t) + \\
\epsilon_4(t) \\ \\
\frac{d^{\alpha_1}X_5(t)}{dt^{\alpha_1}} = a_{51} *X_5(t - \tau_5) + \\
a_{52} *\frac{X_5(t)^{\beta_1}}{(1-X_5(t - \tau_5))^{\beta_1}} - a_{55}X_5(t) + \\
\epsilon_5(t) \\
\end{align*}

\subsubsection{Parameter Table}

\begin{table}[h]\footnotesize
	\caption{Parameter Description and Value}
	\begin{tabular}{rllp{2cm}l}
		\hline	
		Parameter & Value & Interval & Description & Reference \\
		\hline 
		a11 & 0 & [0,1] & Equation 1 & \cite{key1}
		a12 & 0 & [0,1] & Equation 1 & \cite{key1}
		a13 & 0 & [0,1] & Equation 1 & \cite{key1}
		a14 & 0 & [0,1] & Equation 1 & \cite{key1}
		a15 & 0 & [0,1] & Equation 1 & \cite{key1}
		\hline
		a21 & 0 & [0,1] & Equation 2 & \cite{key1}
		a22 & 0 & [0,1] & Equation 2 & \cite{key1}
		a23 & 0 & [0,1] & Equation 2 & \cite{key1}
		a24 & 0 & [0,1] & Equation 2 & \cite{key1}
		a25 & 0 & [0,1] & Equation 2 & \cite{key1}
		\hline
		a31 & 0 & [0,1] & Equation 3 & \cite{key1}
		a32 & 0 & [0,1] & Equation 3 & \cite{key1}
		a33 & 0 & [0,1] & Equation 3 & \cite{key1}
		a34 & 0 & [0,1] & Equation 3 & \cite{key1}
		a35 & 0 & [0,1] & Equation 3 & \cite{key1}
		\hline
		a41 & 0 & [0,1] & Equation 4 & \cite{key1}
		a42 & 0 & [0,1] & Equation 4 & \cite{key1}
		a43 & 0 & [0,1] & Equation 4 & \cite{key1}
		a44 & 0 & [0,1] & Equation 4 & \cite{key1}
		a45 & 0 & [0,1] & Equation 4 & \cite{key1}
		\hline
		a51 & 0 & [0,1] & Equation 5 & \cite{key1}
		a52 & 0 & [0,1] & Equation 5 & \cite{key1}
		a53 & 0 & [0,1] & Equation 5 & \cite{key1}
		a54 & 0 & [0,1] & Equation 5 & \cite{key1}
		a55 & 0 & [0,1] & Equation 5 & \cite{key1}
		\hline
		$\tau_1$ & 1 & [1,2] & Equation 1 & \cite{key1}
		$\tau_2$ & 1 & [1,2] & Equation 2 & \cite{key1}
		$\tau_3$ & 1 & [1,2] & Equation 3 & \cite{key1}
		$\tau_4$ & 1 & [1,2] & Equation 4 & \cite{key1}
		$\tau_5$ & 1 & [1,2] & Equation 5 & \cite{key1}
		\hline
		$\alpha_1$ & 1 & (0,2] & Equation 1 & \cite{key1}
		$\alpha_2$ & 1 & (0,2] & Equation 2 & \cite{key1}
		$\alpha_3$ & 1 & (0,2] & Equation 3 & \cite{key1}
		$\alpha_4$ & 1 & (0,2] & Equation 4 & \cite{key1}
		$\alpha_5$ & 1 & (0,2] & Equation 5 & \cite{key1}
		\hline
		$\beta_1$ & 1 & (0,2] & Equation 1 & \cite{key1}
		$\beta_2$ & 1 & (0,2] & Equation 2 & \cite{key1}
		$\beta_3$ & 1 & (0,2] & Equation 3 & \cite{key1}
		$\beta_4$ & 1 & (0,2] & Equation 4 & \cite{key1}
		$\beta_5$ & 1 & (0,2] & Equation 5 & \cite{key1}
	\end{tabular}	
\end{table}




\section{Results}


\subsection{Tables}

\centering
\begin{table}[H]\footnotesize
	\caption{}
	\begin{tabular}{rp{1cm}p{2cm}p{3cm}p{1cm}}
		\hline
		ID & A & B & C & Reference \\
		\hline
		\hline
	\end{tabular}
\end{table}
\raggedright

\centering
\begin{table}[H]\footnotesize
	\caption{}
	\begin{tabular}{rp{1cm}p{2cm}p{3cm}p{1cm}}
		\hline
		ID & A & B & C & Reference \\
		\hline
		\hline
	\end{tabular}
\end{table}
\raggedright


\subsection{Figures}

\begin{figure}[H]
	\centering
	\begin{minipage}[b]{0.5\linewidth}
	%\includegraphics[scale=0.25]{Example_1_Figure_1.png}
	\end{minipage}\hfill
	\begin{minipage}[b]{0.5\linewidth}
	%\includegraphics[scale=0.25]{Example_1_Figure_2.png}
	\end{minipage}\hfill	
	\begin{minipage}[b]{0.5\linewidth}
	%\includegraphics[scale=0.25]{Example_1_Figure_3.png}
	\end{minipage}\hfill
	\begin{minipage}[b]{0.5\linewidth}
	%\includegraphics[scale=0.25]{Example_1_Figure_4.png}
	\end{minipage}\hfill
	\caption{1, 2, 3 and 4}
	\label{fig:Figure1}
\end{figure} 

\section{Conclusions}

\centering
\begin{table}[H]\footnotesize
	\caption{}
	\begin{tabular}{rp{1cm}p{2cm}p{3cm}p{1cm}}
		\hline
		ID & A & B & C & Reference \\
		\hline
		\hline
	\end{tabular}
\end{table}
\raggedright


\section{R Application Programming Interfaces (APIs)}




\bibliographystyle{plain}
\begin{thebibliography}{00}

\subsection{Main Classroom Article}

\bibitem[100]{key100}Cali, Yongli, Yun Kang, Weiming Wang (2017)
\newblock \textbf{A Stochastic SIRS epidemic model with nonlinear incidence rate}
\newblock Applied Mathematics and Computation 305: 221-240.

\bibitem[211]{key211}B.Fred,C.-C.Carlos
\newblock Mathematical Models in Population Biology and Epidemiology,
\newblock second ed., Springer,2012.

\bibitem[201]{key2001}Michael J. Grayling (2014). 
\newblock phaseR:Phase Plane Analysis of One and Two Dimensional Autonomous ODE Systems. 
\newblock R package version 1.3. https://CRAN.R-project.org/package=phaseR

\bibitem[202]{key2002} Wei and Viliam Simko (2016).
\newblock corrplot: Visualization of a Correlation Matrix. 
\newblock R package version 0.77. https://CRAN.R-project.org/package=corrplot

\bibitem[203]{key2003} Soetaert (2017). 
\newblock plot3D: Plotting Multi-Dimensional Data. 
\newblock R package version 1.1.1. https://CRAN.R-project.org/package=plot3D

\bibitem[204]{key2004} Ligges, U. and Mächler, M. (2003).
\newblock Scatterplot3d - an R Package for Visualizing Multivariate Data.
\newblock Journal of Statistical Software 8(11), 1-20.

\bibitem[205]{key2005}Daniel Adler, Duncan Murdoch and others (2017). 
\newblock rgl: 3D Visualization Using OpenGL. 
\newblock R package version 0.98.1. https://CRAN.R-project.org/package=rgl

\bibitem[206]{key2006} Almende B.V., Benoit Thieurmel and Titouan Robert (2017). visNetwork:
\newblock Network Visualization using 'vis.js' Library. R package version 2.0.1.
\newblock https://CRAN.R-project.org/package=visNetwork

\bibitem[207]{key2007} Nicholas Hamilton (2017). 
\newblock ggtern: An Extension to 'ggplot2', for the Creation of Ternary Diagrams. 
\newblock R package version 2.2.1. https://CRAN.R-project.org/package=ggtern

\subsection{Stability and Convergence Analysis}

\bibitem[300]{key300}Bahar, Arifah and Mao, Xuerong (2004) 
\newblock Stochastic delay Lotka-Volterra model. 
\newblock Journal of Mathematical Analysis and Applications, 292 (2). pp. 364-380. ISSN 0022-247X , http://dx.doi.org/10.1016
/j.jmaa.2003.12.004

\bibitem[301]{key301} X.Li,X.Mao
\newblock Population dynamical behavior of non-autonomous Lotka–Volterra competitive system with random perturbation, 
\newblock Discret. Contin. Dyn. Syst. 24 (2) (2009) 523–593 .

\bibitem[302]{key302} M. Liu, M.Fan, 
\newblock Permanence of stochastic Lotka–Volterra systems,
\newblock J. Nonlinear Sci. (2016), doi:10.1007/s00332-016-9337-2.

\bibitem[303]{key303} A. Lahrouz, L. Omari,
\newblock Extinction and stationary distribution of a stochastic SIRS epidemic model with non-linear incidence,
\newblock Stat. Probab. Lett. 83 (2013) 960–968.

\bibitem[304]{key304} Q.Yang, X. Mao,
\newblock Stochastic dynamics of SIRS epidemic models with random perturbation,
\newblock Math. Biosci. Eng. 1

\bibitem[305]{key305} C. Xu
\newblock Global threshold dynamics of a stochastic differential equation SIS model
\newblock J. Math. Anal. Appl. 447 (2017) 736–757.

\bibitem[306]{key306}Arifah, B. and X. Mao 
\newblock Stochastic Delay Lotka-Volterra Model 
\newblock Journal of Math Anal App. 292 (2) (2004) 364-380.

\bibitem[307]{key307} D.Jiang,N.Shi,X.Li,
\newblock Global stability and stochastic permanence of a non-autonomous logistic equation with random perturbation
\newblock J.Math. Anal. Appl. 340 (1) (2008) 588–597

\bibitem[308]{key308}L.R. Bellet,
\newblock Ergodic properties of Markov processes, in: Open Quantum Systems II,
\newblock Springer, Berlin Heidelberg,2006, pp.1–39.

\bibitem[309]{key309} J. Cushing and J. Hudson, 2012. 
\newblock Evolutionary dynamics and strong Allee effects,
\newblock Journal  of  Biological Dynamics, 6 (2), 941-958.

\bibitem[310]{key310}H.W.Hethcote,
\newblock The mathematics of infectious diseases,
\newblock SIAM Rev. 42(4)(2000) 599–653.


\bibitem[1000]{key1000}R Core Team (2015). 
\newblock R: A language and environment for statistical computing. R Foundation for Statistical Computing, Vienna, Austria.
\newblock URL https://www.R-project.org/.


\end{thebibliography}
\end{document}
