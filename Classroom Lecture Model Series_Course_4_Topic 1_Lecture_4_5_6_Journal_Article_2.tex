%-------------------------------------------------------------------------%
%--------------------Classroom Lecture Model Series-----------------------%
%-------------------------------------------------------------------------%


\begin{document}
\twocolumn
\scriptsize
\begin{frontmatter}
		\title{}
		\author{\corref{cor1}\fnref{fn1}}
		\cortext[cor1]{Corresponding author}
		\address{The Mathematical Learning Space}
		\ead{http://mathlearningspace.weebly.com}	
\end{frontmatter}	

Introduction:
\begin{enumerate}
\item Objective 1:
\item Objective 2:
\item Objective 3:
\end{enumerate}
Conclusion:

keywords: Music Sequences, Sequence Progressions, Motifs, Cluster Analysis, Contemporary Classical Music

\section{Introduction}

\subsection{Plan of the Article}

The following sequence of items provides the direction for the solution of the objectives in the article.

\begin{enumerate}
\item Music Theory Review
\item Review of Mathematical Methods
\item Presentation of Results with Tables and Figures
\item Topics for Discussion in the Classroom
\end{enumerate}

\section{Music Theory Review}

\subsection{Topic A: Music Motifs }

\begin{music}
\smallmusicsize
\nostartrule    
\def\Notes{\vnotes4\elemskip}
\def\NOtes{\vnotes5.5\elemskip}
\def\NOTes{\vnotes6.2\elemskip}
\setlength\parindent{0pt}
\staffbotmarg=3.5\Interligne
\stafftopmarg=3.5\Interligne
\instrumentnumber{1} 
\interinstrument=16\internote
\generalmeter{\meterfrac44}
\generalsignature{-1}
\scale{0.45}
\setsize 1{1.0}
%--------Introduction----------------%
\startextract
\endextract
%--------Theme A----------------%
\startextract
\endextract
%--------Theme B----------------%
\startextract
\endextract
%--------Conclusion----------------%
\startextract
\endextract

\end{music}

\centering	
\begin{table}[H]\tiny
	\caption{}	
	\begin{tabular}{r|p{4cm}|l}
		\hline	
		Topic & Description & Reference \\
		\hline 
		\hline 
	\end{tabular}
\end{table}

\section{Mathematical Background}

\subsection{Cluster Analysis}

\centering	
\begin{table}[H]\tiny
	\caption{}	
	\begin{tabular}{r|p{4cm}|l}
		\hline	
		Models & Description & Reference \\
		\hline 
		\hline 
	\end{tabular}
\end{table}

\section{Results}

\subsection{Tables}

\centering	
\begin{table}[H]\tiny
	\caption{}	
	\begin{tabular}{r|p{4cm}|l}
		\hline	
		Model & Description & Results \\
		\hline 
		\hline 
	\end{tabular}
\end{table}

\subsection{Figures}

\section{Topics in the Classroom}


\bibliographystyle{plain}
\begin{thebibliography}{00}
\bibitem[1]{key100}Sueur J., Aubin T., Simonis C. (2008). 
\newblock Seewave: a free modular tool for sound analysis and synthesis. 
\newblock Bioacoustics, 18: 213-226.

\bibitem[2]{key101}Uwe Ligges, Sebastian Krey, Olaf Mersmann, and Sarah Schnackenberg (2016). 
\newblock tuneR: Analysis of music. 
\newblock URL: http://r-forge.r-project.org/projects/tuner/.

\bibitem[3]{key102}A. Anikin (2017). 
\newblock soundgen: Parametric Voice Synthesis. 
\newblock R package version 1.1.1.

\bibitem[4]{key103}Pieretti N, Farina A, Morri FD (2011) 
\newblock A new methodology to infer the singing activity of an avian community: the Acoustic Complexity Index (ACI). 
\newblock Ecological Indicators, 11, 868-873.
\end{thebibliography}

\end{document}
