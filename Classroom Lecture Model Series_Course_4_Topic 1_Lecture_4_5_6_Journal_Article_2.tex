%-------------------------------------------------------------------------%
%--------------------Classroom Lecture Model Series-----------------------%
%-------------------------------------------------------------------------%


\begin{document}
\twocolumn
\scriptsize
\begin{frontmatter}
		\title{DRAFT Sample: Multivariate Distribution Analysis based on Moment Spectral Properties for a Contemporary Jazz Album}
		\author{\corref{cor1}\fnref{fn1}}
		\cortext[cor1]{Corresponding author}
		\address{The Mathematical Learning Space}
		\ead{http://mathlearningspace.weebly.com}	
\end{frontmatter}	

Introduction:
\begin{enumerate}
\item Objective 1:
\item Objective 2:
\item Objective 3:
\end{enumerate}
Conclusion:

keywords: Music Sequences, Sequence Progressions, Motifs, Cluster Analysis, Contemporary Classical Music

\section{Introduction}

\subsection{Plan of the Article}

The following sequence of items provides the direction for the solution of the objectives in the article.

\begin{enumerate}
\item Music Theory Review
\item Review of Mathematical Methods
\item Presentation of Results with Tables and Figures
\item Topics for Discussion in the Classroom
\end{enumerate}

\section{Music Theory Review}

\begin{table}[H]
	\caption{Composition Motifs in Album with Duration in Seconds and Acoustic Complexity}	
	\begin{tabular}{p{1cm}p{4cm}p{2cm}p{2cm}}
	\hline
	Track No & Name & Duration (seconds) & Acoustic Complexity\\
	\hline
	\hline 
	\end{tabular}
\end{table}

\subsection{Topic A: Music Motifs }

\begin{music}
\smallmusicsize
\nostartrule    
\def\Notes{\vnotes4\elemskip}
\def\NOtes{\vnotes5.5\elemskip}
\def\NOTes{\vnotes6.2\elemskip}
\setlength\parindent{0pt}
\staffbotmarg=3.5\Interligne
\stafftopmarg=3.5\Interligne
\instrumentnumber{1} 
\interinstrument=16\internote
\generalmeter{\meterfrac44}
\generalsignature{-1}
\scale{0.45}
\setsize 1{1.0}
%--------Introduction----------------%
\startextract
\endextract
%--------Theme A----------------%
\startextract
\endextract
%--------Theme B----------------%
\startextract
\endextract
%--------Conclusion----------------%
\startextract
\endextract

\end{music}

\centering	
\begin{table}[H]\tiny
	\caption{}	
	\begin{tabular}{r|p{4cm}|l}
		\hline	
		Topic & Description & Reference \\
		\hline 
		\hline 
	\end{tabular}
\end{table}

\section{Mathematical Background}

\subsection{Motif Design}

\subsubsection{Music Motif Descriptors}

\subsection{Hidden Markov Models and Markov Chains}

\subsubsection{Transition and Adjacency Matrices}

\subsection{SemiMarkov Models}

\subsection{Cluster Analysis}

\centering	
\begin{table}[H]\tiny
	\caption{}	
	\begin{tabular}{r|p{4cm}|l}
		\hline	
		Models & Description & Reference \\
		\hline 
		\hline 
	\end{tabular}
\end{table}

\section{Results}

\subsection{Data}

Table 1 provides the themes and instrument collection for each of the compositions.
	
\begin{table}[H]
\caption{Composition Collection}	
\begin{tabular}{p{1cm}p{4cm}p{2cm}p{1cm}p{1cm}}
\hline
ID & Name & Theme Pattern & Instruments & \\
\hline 
1 & Composition I &  &  & \\
2 & Composition II &  &  & \\
3 & Composition III &  & \\
4 & Composition IV & & \\
5 & Composition V & & & \\
\hline 
6 & Composition VI &  &  & \\
7 & Composition VII &  &  & \\
8 & Composition VIII &  & \\
9 & Composition IX & & \\
10 & Composition X & & & \\
\end{tabular}
\end{table}

\subsection{Tables}

\centering	
\begin{table}[H]\tiny
	\caption{}	
	\begin{tabular}{r|p{4cm}|l}
		\hline	
		Model & Description & Results \\
		\hline 
		\hline 
	\end{tabular}
\end{table}

\subsection{Figures}

\subsection{Group A}

\begin{figure}[H]
	\centering
	\begin{minipage}[b]{0.5\linewidth}
	%\includegraphics[scale=0.25]{Example_1_Figure_1.png}
	\end{minipage}\hfill
	\begin{minipage}[b]{0.5\linewidth}
	%\includegraphics[scale=0.25]{Example_1_Figure_2.png}
	\end{minipage}\hfill	
	\begin{minipage}[b]{0.5\linewidth}
	%\includegraphics[scale=0.25]{Example_1_Figure_3.png}
	\end{minipage}\hfill
	\begin{minipage}[b]{0.5\linewidth}
	%\includegraphics[scale=0.25]{Example_1_Figure_4.png}
	\end{minipage}\hfill
	\caption{1, 2, 3 and 4}
	\label{fig:Figure1}
\end{figure} 

\subsection{Group B}

\begin{figure}[H]
	\centering
	\begin{minipage}[b]{0.3\linewidth}
		%\includegraphics[scale=0.15]{Example_2_Figure_1.png}
	\end{minipage}\hfill
	\begin{minipage}[b]{0.3\linewidth}
		%\includegraphics[scale=0.15]{Example_2_Figure_2.png}
	\end{minipage}\hfill	
	\begin{minipage}[b]{0.3\linewidth}
		%\includegraphics[scale=0.15]{Example_2_Figure_3.png}
	\end{minipage}\hfill
	\begin{minipage}[b]{0.3\linewidth}
		%\includegraphics[scale=0.15]{Example_2_Figure_4.png}
	\end{minipage}\hfill
	\begin{minipage}[b]{0.3\linewidth}
		%\includegraphics[scale=0.15]{Example_2_Figure_5.png}
	\end{minipage}\hfill	
	\begin{minipage}[b]{0.3\linewidth}
		%\includegraphics[scale=0.15]{Example_2_Figure_6.png}
	\end{minipage}\hfill
	\begin{minipage}[b]{0.3\linewidth}
		%\includegraphics[scale=0.15]{Example_2_Figure_7.png}
	\end{minipage}\hfill
	\begin{minipage}[b]{0.3\linewidth}
		%\includegraphics[scale=0.15]{Example_2_Figure_8.png}
	\end{minipage}\hfill	
	\begin{minipage}[b]{0.3\linewidth}
		%\includegraphics[scale=0.15]{Example_2_Figure_9.png}
	\end{minipage}\hfill
	\caption{(a)}
	\label{fig:Figure1}
\end{figure} 


\section{Topics in the Classroom}

\begin{enumerate}
\end{enumerate}

\bibliographystyle{plain}
\begin{thebibliography}{00}

\bibitem[1]{key1}Creative Sequences Techniques for Music Production
\bibitem[2]{key2}A MIDI Sequencer that widens access to the compositional possibilities of novel tunings
\bibitem[3]{key3}Mondrian Music Description Language and Sequencer
\bibitem[4]{key4}Subvision schemes and multiresolution modelling for automated music synthesis and analysis
\bibitem[5]{key5}A Differential Equation Based Approach to Sound Synthesis and Sequencing
\bibitem[6]{key6}Sound Synthesis and Sampling
\bibitem[7]{key7}Algorithmic Clustering of Music Based on String Compression
\bibitem[8]{key8}A Survey of Computer Systems for Expressive Music Performance
\bibitem[9]{key9}Procedural Sequencing: A Form of Proocedural Music Creation
\bibitem[10]{key10}SentiMozart: Music Generation based on Emotions
\bibitem[11]{key11}DeepJ: Style-Specific Music Generation
\bibitem[12]{key12}An End to End Model for Automatic Music Generation: Combining Deep Raw and Symbolic Audio Networks
\bibitem[13]{key13}The Challenge of Realistic Music Generation:Modelling raw audio at scale
\bibitem[14]{key14}Sound Synthesis Based on Ordinary Differential Equations
\bibitem[15]{key15}The Euclidean Algorithm Generates Traditional Musical Rhythms
\bibitem[16]{key16}Apply Learning Algorithms to Music Generation
\bibitem[17]{key17}On the Evaluation of Generative Models in Music
\bibitem[18]{key18}Project Milestone: Generating music with Machine Learning
\bibitem[19]{key19}Xiaolce Band: A Melody and Arrangement Generation Framework for Pop Music
	
\bibitem[100]{key100}Sueur J., Aubin T., Simonis C. (2008). 
\newblock Seewave: a free modular tool for sound analysis and synthesis. 
\newblock Bioacoustics, 18: 213-226.
\bibitem[101]{key101}Uwe Ligges, Sebastian Krey, Olaf Mersmann, and Sarah Schnackenberg (2016). 
\newblock tuneR: Analysis of music. 
\newblock URL: http://r-forge.r-project.org/projects/tuner/.
\bibitem[102]{key102}A. Anikin (2017). 
\newblock soundgen: Parametric Voice Synthesis. 
\newblock R package version 1.1.1.
\bibitem[103]{key103}Pieretti N, Farina A, Morri FD (2011) 
\newblock A new methodology to infer the singing activity of an avian community: the Acoustic Complexity Index (ACI). 
\newblock Ecological Indicators, 11, 868-873.
\end{thebibliography}

\end{document}
