%-------------------------------------------------------------------------%
%--------------------Classroom Lecture Model Series-----------------------%
%-------------------------------------------------------------------------%

\begin{document}
\twocolumn
\scriptsize
\begin{frontmatter}
		\title{}
		\author{\corref{cor1}\fnref{fn1}}
		\cortext[cor1]{Corresponding author}
		\address{The Mathematical Learning Space}
		\ead{http://mathlearningspace.weebly.com}	
\end{frontmatter}	

Introduction:
\begin{enumerate}
\item Objective 1:
\item Objective 2:
\item Objective 3:
\end{enumerate}
Conclusion:

Keywords: Caputo Derivative, Fractional Delayed Differential Operators, p21, DNA Repair
Vocabulary Words:

\section{Introduction}

\subsection{Review Topic A: DNA Repair}


\subsection{Plan of the Article}


\section{Biology Background}

\begin{figure}[H]
\begin{tikzpicture}
	[->,>=stealth',shorten >=1pt,node distance=0.5cm,scale=0.15,
	thick,main node/.style={circle,draw,scale=0.25,transform canvas={scale=0.25},font=\sffamily\Small\bfseries},
	blacknode/.style={shape=circle, draw=black, line width=2},
	bluenode/.style={shape=circle, draw=blue, line width=2},
	greennode/.style={shape=circle, draw=green, line width=2},
	rednode/.style={shape=circle, draw=red, line width=2}
	]
};
\end{tikzpicture}
\end{figure}

\section{Mathematical Background}


\section{Algorithms}

\begin{algorithm}[H]
\footnotesize
\begin{algorithmic}[1]

\end{algorithmic}
\caption{Computation of the Caputo fractional derivative}
	\label{Algorithm_1}
\end{algorithm}

\section{Conclusions}


\section{R Application Programming Interfaces (APIs)}




\bibliographystyle{plain}
\begin{thebibliography}{00}


\bibitem[107]{key107} Ruben A. Cerutti
\newblock The k-Fractional Logistic Equation with k-Caputo Derivative
\newblock Pure Mathematical Sciences, Vol.  4, 2015, no.  1, 9 - 15


\bibitem[1000]{key1000}R Core Team (2015). 
\newblock R: A language and environment for statistical computing. R Foundation for Statistical Computing, Vienna, Austria.
\newblock URL https://www.R-project.org/.

\end{thebibliography}
\end{document}
