%-------------------------------------------------------------------------%
%--------------------Classroom Lecture Model Series-----------------------%
%-------------------------------------------------------------------------%

\begin{document}
\twocolumn
\scriptsize
\begin{frontmatter}
		\title{}
		\author{\corref{cor1}\fnref{fn1}}
		\cortext[cor1]{Corresponding author}
		\address{The Mathematical Learning Space}
		\ead{http://mathlearningspace.weebly.com}	
\end{frontmatter}	

Introduction:
\begin{enumerate}
\item Objective 1:
\item Objective 2:
\item Objective 3:
\end{enumerate}
Conclusion:

Keywords: Caputo Derivative, Fractional Delayed Differential Operators, p21, DNA Repair
Vocabulary Words:

\section{Introduction}

\subsection{Review Topic A: DNA Repair}


\begin{table}[H]
\caption{DNA Repair Pathways KEGG ids and Biological Function}
\tiny
\begin{tabular}{rp{3.5cm}p{1cm}}
\hline
Pathway ID  & Biological Function & Description  \\
\hline
\hline
\end{tabular}
\end{table}


\subsection{Plan of the Article}


\section{Biology Background}

\begin{figure}[H]
\begin{tikzpicture}
	[->,>=stealth',shorten >=1pt,node distance=0.5cm,scale=0.15,
	thick,main node/.style={circle,draw,scale=0.25,transform canvas={scale=0.25},font=\sffamily\Small\bfseries},
	blacknode/.style={shape=circle, draw=black, line width=2},
	bluenode/.style={shape=circle, draw=blue, line width=2},
	greennode/.style={shape=circle, draw=green, line width=2},
	rednode/.style={shape=circle, draw=red, line width=2}
	]
};
\end{tikzpicture}
\end{figure}

\section{Mathematical Background}


\section{Algorithms}

\begin{algorithm}[H]
\footnotesize
\begin{algorithmic}[1]

\end{algorithmic}
\caption{Computation of the Caputo fractional derivative}
	\label{Algorithm_1}
\end{algorithm}

\section{Results}

\subsection{Changepoint Analysis, Synthesis of Dynamics of Aggregate Molecular Behavior}

\begin{table}[H]\tiny
  \caption{Change Points and Test Statistic ($\Delta$=First and second moments, TS=test statistic}
\begin{tabular}{rlp{0.7cm}lllll}
\hline	 
Experiment &Method & Type & TS & Penalty & Segment & Max Cpts & $\tau_{Location}$ \\
\hline
\hline
\end{tabular}
\end{table}

\begin{figure}[H]
\centering
%\includegraphics[scale=0.45]{Figure3a.png}
\caption{ (a) Experimnent 1 and (b) Experiment 2 and (c) Experiment 3 and (d) Experiment 4 }
\label{fig:Figure1}
\end{figure}

\subsection{Tables}


\subsection{Figures}

\begin{figure}[H]
	\centering
	\begin{minipage}[b]{0.5\linewidth}
	%\includegraphics[scale=0.25]{Example_1_Figure_1.png}
	\end{minipage}\hfill
	\begin{minipage}[b]{0.5\linewidth}
	%\includegraphics[scale=0.25]{Example_1_Figure_2.png}
	\end{minipage}\hfill	
	\begin{minipage}[b]{0.5\linewidth}
	%\includegraphics[scale=0.25]{Example_1_Figure_3.png}
	\end{minipage}\hfill
	\begin{minipage}[b]{0.5\linewidth}
	%\includegraphics[scale=0.25]{Example_1_Figure_4.png}
	\end{minipage}\hfill
	\caption{1, 2, 3 and 4}
	\label{fig:Figure1}
\end{figure} 


\section{Conclusions}


\section{R Application Programming Interfaces (APIs)}




\bibliographystyle{plain}
\begin{thebibliography}{00}


\bibitem[107]{key107} Ruben A. Cerutti
\newblock The k-Fractional Logistic Equation with k-Caputo Derivative
\newblock Pure Mathematical Sciences, Vol.  4, 2015, no.  1, 9 - 15.

\bibitem[310]{key3010} Angelo Canty and Brian Ripley (2017). 
\newblock boot: Bootstrap R (S-Plus) Functions. 
\newblock R package version 1.3-20.

\bibitem[311]{key3011}Yves Tillé and Alina Matei (2016). 
\newblock sampling: Survey Sampling. 
\newblock R package version 2.8. https://CRAN.R-project.org/package=sampling

\bibitem[312]{key3012} Davison, A. C. and Hinkley, D. V. (1997) 
\newblock Bootstrap Methods and Their Applications. 
\newblock Cambridge University Press, Cambridge. ISBN 0-521-57391-2.

\bibitem[313]{key3013}Hans Werner Borchers (2017). 
\newblock pracma: Practical Numerical Math Functions. 
\newblock R package version 2.0.7. https://CRAN.R-project.org/package=pracma

\bibitem[314]{key3014}Marie Laure Delignette-Muller, Christophe Dutang (2015). 
\newblock fitdistrplus:An R Package for Fitting Distributions. 
\newblock Journal of Statistical Software, 64(4), 1-34. URL http://www.jstatsoft.org/v64/i04/.

\bibitem[315]{key3015} Scott Chasalow (2012). 
\newblock combinat: combinatorics utilities. 
\newblock R package version 0.0-8. https://CRAN.R-project.org/package=combinat

\bibitem[316]{key3016} C. Dutang, V. Goulet and M. Pigeon (2008). 
\newblock actuar: An R Package for Actuarial Science. 
\newblock Journal of Statistical Software, vol. 25, no. 7, 1-37. URL http://www.jstatsoft.org/v25/i07

\bibitem[317]{key3017} Douglas Bates and Martin Maechler (2017). 
\newblock Matrix: Sparse and Dense Matrix Classes and Methods. 
\newblock R package version 1.2-11.https://CRAN.R-project.org/package=Matrix

\bibitem[318]{key3018} Killick R, Haynes K and Eckley IA (2016).
\newblock changepoint: An R package for changepoint analysis. R package version
\newblock 2.2.2, <URL: https://CRAN.R-project.org/package=changepoint>.

\bibitem[1000]{key1000}R Core Team (2015). 
\newblock R: A language and environment for statistical computing. R Foundation for Statistical Computing, Vienna, Austria.
\newblock URL https://www.R-project.org/.

\end{thebibliography}
\end{document}
