%-------------------------------------------------------------------------%
%--------------------Classroom Lecture Model Series-----------------------%
%-------------------------------------------------------------------------%

\begin{document}
\twocolumn
\scriptsize
\begin{frontmatter}
		\title{}
		\author{\corref{cor1}\fnref{fn1}}
		\cortext[cor1]{Corresponding author}
		\address{The Mathematical Learning Space}
		\ead{http://mathlearningspace.weebly.com}	
\end{frontmatter}	

Introduction:
\begin{enumerate}
\item Objective 1:
\item Objective 2:
\item Objective 3:
\end{enumerate}
Conclusion:

Keywords: Non-Linear Time Series Models, Markov Models
Vocabulary Words:

\section{Introduction}


\subsection{Plan of the Article}


\section{Mathematical Review Topic: Nonlinear Time Series Models}


\section{Results}

\subsection{Tables}

\subsection{Figures}

\begin{figure}[H]
	\centering
	\begin{minipage}[b]{0.5\linewidth}
	%\includegraphics[scale=0.25]{Example_1_Figure_1.png}
	\end{minipage}\hfill
	\begin{minipage}[b]{0.5\linewidth}
	%\includegraphics[scale=0.25]{Example_1_Figure_2.png}
	\end{minipage}\hfill	
	\begin{minipage}[b]{0.5\linewidth}
	%\includegraphics[scale=0.25]{Example_1_Figure_3.png}
	\end{minipage}\hfill
	\begin{minipage}[b]{0.5\linewidth}
	%\includegraphics[scale=0.25]{Example_1_Figure_4.png}
	\end{minipage}\hfill
	\caption{1, 2, 3 and 4}
	\label{fig:Figure1}
\end{figure} 


\section{Conclusions}


\section{R Application Programming Interfaces (APIs)}

\bibliographystyle{plain}
\begin{thebibliography}{00}

\bibitem[100]{key11}. H. F. Tam, C. Chang, and Y. S. Hung, (2012)
\newblock Application of Granger causality to gene regulatory network discovery  
\newblock Proc. IEEE 6th International Conference on Systems Biology, Xi'an, China, pp. 232239.

\bibitem[101]{key12} C. Granger, (1969)
\newblock Investigating causal relations by econometric models and cross-spectral methods
\newblock Econometrica, vol. 37, no. 3, pp. 424-438.

\bibitem[102]{key13}M. Ding, Y. Chen, and S. L. Bressler, (2006)
\newblock "Granger causality: basic theory and application to neuroscience 
\newblock in Handbook of Time Series Analysis, S. Schelter, M. Winterhalder, and J. Timmer, Eds. Wienheim: Wiley, pp. 438-460.

\bibitem[103]{key14} Hafner, C. M. and Herwartz, H. (2009) 
\newblock Testing for linear vector autoregressive dynamics under multivariate generalized autoregressive heteroskedasticity 
\newblock Statistica Neerlandica, 63: 294-323

\bibitem[104]{key15}Hamilton, J. (1994), 
\newblock Time Series Analysis 
\newblock Princeton University Press, Princeton.

\bibitem[105]{key16}Lütkepohl, H. (2006), 
\newblock New Introduction to Multiple Time Series Analysis
\newblock Springer, New York.

\bibitem[201]{key201} Petitjean F, Ketterlin A and Gancarski P (2011). 
\newblock  A global averaging method for dynamic time warping, with applications to clustering.
\newblock  Pattern Recognition, 44(3), pp. 678 - 693. ISSN 0031-3203,

\bibitem[202]{key202} Nuno Castro, Paulo J. Azevedo (2010)
\newblock  Multiresolution Motif Discovery in Time Series. 
\newblock  SDM 665-676 
 
\bibitem[203]{key203} Chiu, B., Keogh, E., and Lonardi, S. (2003). 
\newblock  Probabilistic discovery of time series motifs. 
\newblock  In Proceedings of the Ninth ACM SIGKDD international Conference on Knowledge Discovery and Data Mining (Washington, D.C., August 24 - 27, 2003). KDD '03. ACM Press, New York, NY, 493-498.

\bibitem[204]{key204} J. Lin, E. Keogh, P. Patel, and S. Lonardi, (2002).
\newblock  Finding Motifs in Time Series, the 2nd Workshop on Temporal Data Mining, 
\newblock  The 8th ACM Int'l Conference on Knowledge Discovery and Data Mining. Edmonton, Alberta, Canada, pp. 53-68. 

\bibitem[205]{key205} H. Tang and S.S. Liao. (2008)
\newblock  Discovering original motifs with different lengths from time series. 
\newblock  Knowledge.-Based System. 21, 7, 666-671. 

\bibitem[501]{key501}S.A. Pirogov, A.N. Rybko, A.S. Kalinina, M.S. Gelfand
\newblock Recombination processes and non-linear Markov chains
\newblock http://arxiv.org/abs/1312.7653v1

\bibitem[502]{key502}Lecture 1 - Markov Processes
\newblock Structure  and  Approximation  for  Markov  Chains in  (Bio)chemical  Kinetics
\newblock Arizona State University

\bibitem[503]{key503}Balzter, H. (2000): 
\newblock Markov chain models for vegetation dynamics. 
\newblock Ecological Modelling 126, 139-154.

\bibitem[504]{key504}Ivan B. Djordjevic
\newblock Markov Chain-Like Quantum Biological Modeling of Mutations, Aging, and Evolution 
\newblock Life 2015, 5, 1518-1538; doi:10.3390/life5031518 

\bibitem[1000]{key1000}R Core Team (2015). 
\newblock R: A language and environment for statistical computing. R Foundation for Statistical Computing, Vienna, Austria.
\newblock URL https://www.R-project.org/.
\end{thebibliography}
\end{document}
