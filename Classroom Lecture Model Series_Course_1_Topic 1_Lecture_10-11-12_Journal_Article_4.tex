%-------------------------------------------------------------------------%
%--------------------Classroom Lecture Model Series-----------------------%
%-------------------------------------------------------------------------%

%-----------------Work in Progress for the Classroom----------------------%

\begin{document}
\twocolumn
\scriptsize
\begin{frontmatter}
		\title{DRAFT Sample: A Mathematical Model of Molecular Complexity and Epigenetic Modifications with Mucin Regulation in GastroIntestinal Cancers}
		\author{\corref{cor1}\fnref{fn1}}
		\cortext[cor1]{Corresponding author}
		\address{The Mathematical Learning Space}
		\ead{http://mathlearningspace.weebly.com}	
\end{frontmatter}	

Introduction:
\begin{enumerate}
\item Objective 1:
\item Objective 2:
\item Objective 3:
\end{enumerate}
Conclusion:

Keywords: Non-Linear Time Series Models, Markov Models
Vocabulary Words:

\section{Introduction}


\begin{figure}[H]
\begin{minipage}[b]{0.3\linewidth}
\includegraphics[scale=0.40]{GastroIntestinal_Cancer.png} 
\end{minipage}\hfill
\caption{(a) GastroIntestinal Cancer}
\label{fig:Figure1}
\end{figure} 


\begin{table}[H]\centering
	\begin{tabular}{p{1cm}p{4cm}p{3cm}}
		Article ID & Summary & Comments\\
		\hline
		\hline
	\end{tabular}
\end{table}


\subsection{Plan of the Article}

\begin{enumerate}
\end{enumerate}


\section{Protein Interaction Diagram and Annotations}

\begin{tikzpicture}
	[->,>=stealth',shorten >=2pt,node distance=3cm,
	thick,main node/.style={circle,draw,scale=0.25,transform canvas={scale=0.75},font=\sffamily\Small\bfseries},
	blacknode/.style={shape=circle, draw=black, line width=2},
	bluenode/.style={shape=circle, draw=blue, line width=2},
	greennode/.style={shape=circle, draw=green, line width=2},
	rednode/.style={shape=circle, draw=red, line width=2}
	]
	%-------Legend ------------------------------------------------------------	
	\matrix [draw,below left] at (current bounding box.south) {
		\node [state,label=right:State] {}; Description. \\
		\node [shapeSquare,label=right:Square-.] {}; \\
		\node [shapeEllipse,label=right:Ellipse-.] {}; \\
		\node [shapeTriangle,label=right:Triangle-.] {}; \\
		\node [shapeHexagon,label=right:Hexagon-.] {}; \\
	};
\end{tikzpicture}

\centering	
\begin{table}[H]\tiny
\caption{Gene/Protein/Enzymes Description and References}	
\begin{tabular}{r|p{3cm}|l}
\hline	
Gene/Protein & Description & Reference \\
\hline 
\hline 
\end{tabular}
\end{table}

\section{Mathematical Review Topic: Nonlinear Time Series Models}

\centering
\begin{table}[H]\footnotesize
	\caption{}
	\begin{tabular}{rp{1cm}p{2cm}p{3cm}p{1cm}}
		\hline
		ID & A & B & C & Reference \\
		\hline
		\hline
	\end{tabular}
\end{table}
\raggedright

\subsection{Model A:}

\begin{equation}
X_1= a11*X_1(t- \tau_1) + a12*\epsilon_1(t)
\end{equation}

\begin{equation}
X_2= a21*X_2(t- \tau_2) + a22*\epsilon_2(t)
\end{equation}

\begin{equation}
X_3= a31*X_3(t- \tau_3) + a32*\epsilon_3(t)
\end{equation}

\begin{equation}
X_4= a41*X_4(t- \tau_4) + a42*\epsilon_4(t)
\end{equation}

\begin{equation}
X_5= a51*X_5(t- \tau_5) + a52*\epsilon_5(t)
\end{equation}

\subsubsection{Parameter Table}

\begin{table}[h]\footnotesize
	\caption{Parameter Description and Value}
	\begin{tabular}{rllp{2cm}l}
		\hline	
		Parameter & Value & Interval & Description & Reference \\
		\hline 
		a11 & 0 & [0,1] & Equation 1 & \cite{key1}
		a12 & 0 & [0,1] & Equation 1 & \cite{key1}
		a13 & 0 & [0,1] & Equation 1 & \cite{key1}
		a14 & 0 & [0,1] & Equation 1 & \cite{key1}
		a15 & 0 & [0,1] & Equation 1 & \cite{key1}
		\hline
		a21 & 0 & [0,1] & Equation 2 & \cite{key1}
		a22 & 0 & [0,1] & Equation 2 & \cite{key1}
		a23 & 0 & [0,1] & Equation 2 & \cite{key1}
		a24 & 0 & [0,1] & Equation 2 & \cite{key1}
		a25 & 0 & [0,1] & Equation 2 & \cite{key1}
		\hline
		a31 & 0 & [0,1] & Equation 3 & \cite{key1}
		a32 & 0 & [0,1] & Equation 3 & \cite{key1}
		a33 & 0 & [0,1] & Equation 3 & \cite{key1}
		a34 & 0 & [0,1] & Equation 3 & \cite{key1}
		a35 & 0 & [0,1] & Equation 3 & \cite{key1}
		\hline
		a41 & 0 & [0,1] & Equation 4 & \cite{key1}
		a42 & 0 & [0,1] & Equation 4 & \cite{key1}
		a43 & 0 & [0,1] & Equation 4 & \cite{key1}
		a44 & 0 & [0,1] & Equation 4 & \cite{key1}
		a45 & 0 & [0,1] & Equation 4 & \cite{key1}
		\hline
		a51 & 0 & [0,1] & Equation 5 & \cite{key1}
		a52 & 0 & [0,1] & Equation 5 & \cite{key1}
		a53 & 0 & [0,1] & Equation 5 & \cite{key1}
		a54 & 0 & [0,1] & Equation 5 & \cite{key1}
		a55 & 0 & [0,1] & Equation 5 & \cite{key1}
		\hline
		$\tau_1$ & 1 & [1,2] & Equation 1 & \cite{key1}
		$\tau_2$ & 1 & [1,2] & Equation 2 & \cite{key1}
		$\tau_3$ & 1 & [1,2] & Equation 3 & \cite{key1}
		$\tau_4$ & 1 & [1,2] & Equation 4 & \cite{key1}
		$\tau_5$ & 1 & [1,2] & Equation 5 & \cite{key1}
		\hline
		$\alpha_1$ & 1 & (0,2] & Equation 1 & \cite{key1}
		$\alpha_2$ & 1 & (0,2] & Equation 2 & \cite{key1}
		$\alpha_3$ & 1 & (0,2] & Equation 3 & \cite{key1}
		$\alpha_4$ & 1 & (0,2] & Equation 4 & \cite{key1}
		$\alpha_5$ & 1 & (0,2] & Equation 5 & \cite{key1}
		\hline
		$\beta_1$ & 1 & (0,2] & Equation 1 & \cite{key1}
		$\beta_2$ & 1 & (0,2] & Equation 2 & \cite{key1}
		$\beta_3$ & 1 & (0,2] & Equation 3 & \cite{key1}
		$\beta_4$ & 1 & (0,2] & Equation 4 & \cite{key1}
		$\beta_5$ & 1 & (0,2] & Equation 5 & \cite{key1}
	\end{tabular}	
\end{table}

\subsection{Model B:}

\begin{equation}
X_1= a11*X_1(t- \tau_1) + a12*\epsilon_1(t)
\end{equation}

\begin{equation}
X_2= a21*X_2(t- \tau_2) + a22*\epsilon_2(t)
\end{equation}

\begin{equation}
X_3= a31*X_3(t- \tau_3) + a32*\epsilon_3(t)
\end{equation}

\begin{equation}
X_4= a41*X_4(t- \tau_4) + a42*\epsilon_4(t)
\end{equation}

\begin{equation}
X_5= a51*X_5(t- \tau_5) + a52*\epsilon_5(t)
\end{equation}


\subsubsection{Parameter Table}

\begin{table}[h]\footnotesize
	\caption{Parameter Description and Value}
	\begin{tabular}{rllp{2cm}l}
		\hline	
		Parameter & Value & Interval & Description & Reference \\
		\hline 
		a11 & 0 & [0,1] & Equation 1 & \cite{key1}
		a12 & 0 & [0,1] & Equation 1 & \cite{key1}
		a13 & 0 & [0,1] & Equation 1 & \cite{key1}
		a14 & 0 & [0,1] & Equation 1 & \cite{key1}
		a15 & 0 & [0,1] & Equation 1 & \cite{key1}
		\hline
		a21 & 0 & [0,1] & Equation 2 & \cite{key1}
		a22 & 0 & [0,1] & Equation 2 & \cite{key1}
		a23 & 0 & [0,1] & Equation 2 & \cite{key1}
		a24 & 0 & [0,1] & Equation 2 & \cite{key1}
		a25 & 0 & [0,1] & Equation 2 & \cite{key1}
		\hline
		a31 & 0 & [0,1] & Equation 3 & \cite{key1}
		a32 & 0 & [0,1] & Equation 3 & \cite{key1}
		a33 & 0 & [0,1] & Equation 3 & \cite{key1}
		a34 & 0 & [0,1] & Equation 3 & \cite{key1}
		a35 & 0 & [0,1] & Equation 3 & \cite{key1}
		\hline
		a41 & 0 & [0,1] & Equation 4 & \cite{key1}
		a42 & 0 & [0,1] & Equation 4 & \cite{key1}
		a43 & 0 & [0,1] & Equation 4 & \cite{key1}
		a44 & 0 & [0,1] & Equation 4 & \cite{key1}
		a45 & 0 & [0,1] & Equation 4 & \cite{key1}
		\hline
		a51 & 0 & [0,1] & Equation 5 & \cite{key1}
		a52 & 0 & [0,1] & Equation 5 & \cite{key1}
		a53 & 0 & [0,1] & Equation 5 & \cite{key1}
		a54 & 0 & [0,1] & Equation 5 & \cite{key1}
		a55 & 0 & [0,1] & Equation 5 & \cite{key1}
		\hline
		$\tau_1$ & 1 & [1,2] & Equation 1 & \cite{key1}
		$\tau_2$ & 1 & [1,2] & Equation 2 & \cite{key1}
		$\tau_3$ & 1 & [1,2] & Equation 3 & \cite{key1}
		$\tau_4$ & 1 & [1,2] & Equation 4 & \cite{key1}
		$\tau_5$ & 1 & [1,2] & Equation 5 & \cite{key1}
		\hline
		$\alpha_1$ & 1 & (0,2] & Equation 1 & \cite{key1}
		$\alpha_2$ & 1 & (0,2] & Equation 2 & \cite{key1}
		$\alpha_3$ & 1 & (0,2] & Equation 3 & \cite{key1}
		$\alpha_4$ & 1 & (0,2] & Equation 4 & \cite{key1}
		$\alpha_5$ & 1 & (0,2] & Equation 5 & \cite{key1}
		\hline
		$\beta_1$ & 1 & (0,2] & Equation 1 & \cite{key1}
		$\beta_2$ & 1 & (0,2] & Equation 2 & \cite{key1}
		$\beta_3$ & 1 & (0,2] & Equation 3 & \cite{key1}
		$\beta_4$ & 1 & (0,2] & Equation 4 & \cite{key1}
		$\beta_5$ & 1 & (0,2] & Equation 5 & \cite{key1}
	\end{tabular}	
\end{table}



\subsection{Model C:}

\begin{equation}
X_1= a11*X_1(t- \tau_1) + a12*\epsilon_1(t)
\end{equation}

\begin{equation}
X_2= a21*X_2(t- \tau_2) + a22*\epsilon_2(t)
\end{equation}

\begin{equation}
X_3= a31*X_3(t- \tau_3) + a32*\epsilon_3(t)
\end{equation}

\begin{equation}
X_4= a41*X_4(t- \tau_4) + a42*\epsilon_4(t)
\end{equation}

\begin{equation}
X_5= a51*X_5(t- \tau_5) + a52*\epsilon_5(t)
\end{equation}


\subsubsection{Parameter Table}

\begin{table}[h]\footnotesize
	\caption{Parameter Description and Value}
	\begin{tabular}{rllp{2cm}l}
		\hline	
		Parameter & Value & Interval & Description & Reference \\
		\hline 
		a11 & 0 & [0,1] & Equation 1 & \cite{key1}
		a12 & 0 & [0,1] & Equation 1 & \cite{key1}
		a13 & 0 & [0,1] & Equation 1 & \cite{key1}
		a14 & 0 & [0,1] & Equation 1 & \cite{key1}
		a15 & 0 & [0,1] & Equation 1 & \cite{key1}
		\hline
		a21 & 0 & [0,1] & Equation 2 & \cite{key1}
		a22 & 0 & [0,1] & Equation 2 & \cite{key1}
		a23 & 0 & [0,1] & Equation 2 & \cite{key1}
		a24 & 0 & [0,1] & Equation 2 & \cite{key1}
		a25 & 0 & [0,1] & Equation 2 & \cite{key1}
		\hline
		a31 & 0 & [0,1] & Equation 3 & \cite{key1}
		a32 & 0 & [0,1] & Equation 3 & \cite{key1}
		a33 & 0 & [0,1] & Equation 3 & \cite{key1}
		a34 & 0 & [0,1] & Equation 3 & \cite{key1}
		a35 & 0 & [0,1] & Equation 3 & \cite{key1}
		\hline
		a41 & 0 & [0,1] & Equation 4 & \cite{key1}
		a42 & 0 & [0,1] & Equation 4 & \cite{key1}
		a43 & 0 & [0,1] & Equation 4 & \cite{key1}
		a44 & 0 & [0,1] & Equation 4 & \cite{key1}
		a45 & 0 & [0,1] & Equation 4 & \cite{key1}
		\hline
		a51 & 0 & [0,1] & Equation 5 & \cite{key1}
		a52 & 0 & [0,1] & Equation 5 & \cite{key1}
		a53 & 0 & [0,1] & Equation 5 & \cite{key1}
		a54 & 0 & [0,1] & Equation 5 & \cite{key1}
		a55 & 0 & [0,1] & Equation 5 & \cite{key1}
		\hline
		$\tau_1$ & 1 & [1,2] & Equation 1 & \cite{key1}
		$\tau_2$ & 1 & [1,2] & Equation 2 & \cite{key1}
		$\tau_3$ & 1 & [1,2] & Equation 3 & \cite{key1}
		$\tau_4$ & 1 & [1,2] & Equation 4 & \cite{key1}
		$\tau_5$ & 1 & [1,2] & Equation 5 & \cite{key1}
		\hline
		$\alpha_1$ & 1 & (0,2] & Equation 1 & \cite{key1}
		$\alpha_2$ & 1 & (0,2] & Equation 2 & \cite{key1}
		$\alpha_3$ & 1 & (0,2] & Equation 3 & \cite{key1}
		$\alpha_4$ & 1 & (0,2] & Equation 4 & \cite{key1}
		$\alpha_5$ & 1 & (0,2] & Equation 5 & \cite{key1}
		\hline
		$\beta_1$ & 1 & (0,2] & Equation 1 & \cite{key1}
		$\beta_2$ & 1 & (0,2] & Equation 2 & \cite{key1}
		$\beta_3$ & 1 & (0,2] & Equation 3 & \cite{key1}
		$\beta_4$ & 1 & (0,2] & Equation 4 & \cite{key1}
		$\beta_5$ & 1 & (0,2] & Equation 5 & \cite{key1}
	\end{tabular}	
\end{table}

\subsection{Estimation}

\subsection{Statistical Features}

\begin{table}[H]\centering
	\begin{tabular}{p{1cm}p{4cm}p{3cm}}
		Feature ID & Statistical Test & Comments\\
		\hline
		\hline
	\end{tabular}
\end{table}

\subsection{Prediction}

\centering
\begin{table}[H]\footnotesize
	\caption{}
	\begin{tabular}{rp{1cm}p{2cm}p{3cm}p{1cm}}
		\hline
		ID & A & B & C & Reference \\
		\hline
		\hline
	\end{tabular}
\end{table}
\raggedright

\section{Results}

\begin{enumerate}
\end{enumerate}

\subsection{Tables}

\centering
\begin{table}[H]\footnotesize
	\caption{}
	\begin{tabular}{rp{1cm}p{2cm}p{3cm}p{1cm}}
		\hline
		ID & A & B & C & Reference \\
		\hline
		\hline
	\end{tabular}
\end{table}
\raggedright

\centering
\begin{table}[H]\footnotesize
	\caption{}
	\begin{tabular}{rp{1cm}p{2cm}p{3cm}p{1cm}}
		\hline
		ID & A & B & C & Reference \\
		\hline
		\hline
	\end{tabular}
\end{table}
\raggedright

\centering
\begin{table}[H]\footnotesize
	\caption{}
	\begin{tabular}{rp{1cm}p{2cm}p{3cm}p{1cm}}
		\hline
		ID & A & B & C & Reference \\
		\hline
		\hline
	\end{tabular}
\end{table}
\raggedright

\subsection{Figures}

\subsubsection{ES:A}

\begin{figure}[H]
	\centering
	\begin{minipage}[b]{0.5\linewidth}
	%\includegraphics[scale=0.25]{Example_1_Figure_1.png}
	\end{minipage}\hfill
	\begin{minipage}[b]{0.5\linewidth}
	%\includegraphics[scale=0.25]{Example_1_Figure_2.png}
	\end{minipage}\hfill	
	\begin{minipage}[b]{0.5\linewidth}
	%\includegraphics[scale=0.25]{Example_1_Figure_3.png}
	\end{minipage}\hfill
	\begin{minipage}[b]{0.5\linewidth}
	%\includegraphics[scale=0.25]{Example_1_Figure_4.png}
	\end{minipage}\hfill
	\caption{1, 2, 3 and 4}
	\label{fig:Figure1}
\end{figure} 

\subsubsection{ES:B}

\begin{figure}[H]
	\centering
	\begin{minipage}[b]{0.5\linewidth}
	%\includegraphics[scale=0.25]{Example_2_Figure_1.png}
	\end{minipage}\hfill
	\begin{minipage}[b]{0.5\linewidth}
	%\includegraphics[scale=0.25]{Example_2_Figure_2.png}
	\end{minipage}\hfill	
	\begin{minipage}[b]{0.5\linewidth}
	%\includegraphics[scale=0.25]{Example_2_Figure_3.png}
	\end{minipage}\hfill
	\begin{minipage}[b]{0.5\linewidth}
	%\includegraphics[scale=0.25]{Example_2_Figure_4.png}
	\end{minipage}\hfill
	\caption{1, 2, 3 and 4}
	\label{fig:Figure1}
\end{figure} 

\subsubsection{ES:C}

\begin{figure}[H]
	\centering
	\begin{minipage}[b]{0.5\linewidth}
	%\includegraphics[scale=0.25]{Example_3_Figure_1.png}
	\end{minipage}\hfill
	\begin{minipage}[b]{0.5\linewidth}
	%\includegraphics[scale=0.25]{Example_3_Figure_2.png}
	\end{minipage}\hfill	
	\begin{minipage}[b]{0.5\linewidth}
	%\includegraphics[scale=0.25]{Example_3_Figure_3.png}
	\end{minipage}\hfill
	\begin{minipage}[b]{0.5\linewidth}
	%\includegraphics[scale=0.25]{Example_3_Figure_4.png}
	\end{minipage}\hfill
	\caption{1, 2, 3 and 4}
	\label{fig:Figure1}
\end{figure} 


\section{Conclusions and Topics in the Classroom}

\begin{table}[H]\centering
	\begin{tabular}{p{1cm}p{4cm}p{3cm}}
		Article ID & Summary & Comments\\
		\hline
		\hline
	\end{tabular}
\end{table}


\section{R Application Programming Interfaces (APIs)}

\bibliographystyle{plain}
\begin{thebibliography}{00}

\bibitem[100]{key11}. H. F. Tam, C. Chang, and Y. S. Hung, (2012)
\newblock Application of Granger causality to gene regulatory network discovery  
\newblock Proc. IEEE 6th International Conference on Systems Biology, Xi'an, China, pp. 232239.

\bibitem[101]{key12} C. Granger, (1969)
\newblock Investigating causal relations by econometric models and cross-spectral methods
\newblock Econometrica, vol. 37, no. 3, pp. 424-438.

\bibitem[102]{key13}M. Ding, Y. Chen, and S. L. Bressler, (2006)
\newblock "Granger causality: basic theory and application to neuroscience 
\newblock in Handbook of Time Series Analysis, S. Schelter, M. Winterhalder, and J. Timmer, Eds. Wienheim: Wiley, pp. 438-460.

\bibitem[103]{key14} Hafner, C. M. and Herwartz, H. (2009) 
\newblock Testing for linear vector autoregressive dynamics under multivariate generalized autoregressive heteroskedasticity 
\newblock Statistica Neerlandica, 63: 294-323

\bibitem[104]{key15}Hamilton, J. (1994), 
\newblock Time Series Analysis 
\newblock Princeton University Press, Princeton.

\bibitem[105]{key16}Lütkepohl, H. (2006), 
\newblock New Introduction to Multiple Time Series Analysis
\newblock Springer, New York.

\bibitem[201]{key201} Petitjean F, Ketterlin A and Gancarski P (2011). 
\newblock  A global averaging method for dynamic time warping, with applications to clustering.
\newblock  Pattern Recognition, 44(3), pp. 678 - 693. ISSN 0031-3203,

\bibitem[202]{key202} Nuno Castro, Paulo J. Azevedo (2010)
\newblock  Multiresolution Motif Discovery in Time Series. 
\newblock  SDM 665-676 
 
\bibitem[203]{key203} Chiu, B., Keogh, E., and Lonardi, S. (2003). 
\newblock Probabilistic discovery of time series motifs. 
\newblock In Proceedings of the Ninth ACM SIGKDD international Conference on Knowledge Discovery and Data Mining (Washington, D.C., August 24 - 27, 2003). KDD '03. ACM Press, New York, NY, 493-498.

\bibitem[204]{key204} J. Lin, E. Keogh, P. Patel, and S. Lonardi, (2002).
\newblock Finding Motifs in Time Series, the 2nd Workshop on Temporal Data Mining, 
\newblock The 8th ACM Int'l Conference on Knowledge Discovery and Data Mining. Edmonton, Alberta, Canada, pp. 53-68. 

\bibitem[205]{key205} H. Tang and S.S. Liao. (2008)
\newblock Discovering original motifs with different lengths from time series. 
\newblock Knowledge.-Based System. 21, 7, 666-671. 

\bibitem[206]{key206} Alexis Sarda-Espinosa (2016). 
\newblock dtwclust: Time Series Clustering Along with Optimizations for the Dynamic Time Warping Distance. 
R package version 2.1.2. http://CRAN.R-project.org/package=dtwclust.

\bibitem[207]{key207}  Giorgino T (2009). 
\newblock Computing and Visualizing Dynamic Time Warping Alignments in R: The dtw Package.
\newblock Journal of Statistical Software,*31*(7), pp. 1-24. <URL: http://www.jstatsoft.org/v31/i07/>.

\bibitem[208]{key208} Cheng Fan (2015). 
\newblock TSMining: Mining Univariate and Multivariate Motifs in Time-Series Data. R package version 1.0.
\newblock http://CRAN.R-project.org/package=TSMining

\subsection{Markov Models}

\bibitem[250]{key250} Speculative Markov Blanket Discovery for Optimal Feature Selection
\bibitem[251]{key251} Using Markov Blankets for Causal Structure Learning
\bibitem[252]{key252} Bayesian Markov Blanket Estimation
\bibitem[253]{key253} Biological self-organisation and Markov blankets 
\bibitem[254]{key254} Bayesian Markov Blanket Estimation
\bibitem[255]{key255} Overview of Bayesian Networks With Examples in R
\bibitem[256]{key256} Machine Learning at CMU: Bayesian  Networks
\bibitem[257]{key257} Signal Processing Based on Hidden Markov Models for Extracting Small Channel Currents
\bibitem[258]{key258} A Markov Classification Model for Metabolic Pathways
\bibitem[259]{key259} Markov-Chain Modeling for Multicast Signaling Delay Analysis
\bibitem[260]{key260} An Introduction to Hidden Markov Models
\bibitem[261]{key261} A robust hidden semi-Markov model with application to aCGH data processing 

\bibitem[501]{key501}S.A. Pirogov, A.N. Rybko, A.S. Kalinina, M.S. Gelfand
\newblock Recombination processes and non-linear Markov chains
\newblock http://arxiv.org/abs/1312.7653v1

\bibitem[502]{key502}Lecture 1 - Markov Processes
\newblock Structure  and  Approximation  for  Markov  Chains in  (Bio)chemical  Kinetics
\newblock Arizona State University

\bibitem[503]{key503}Balzter, H. (2000): 
\newblock Markov chain models for vegetation dynamics. 
\newblock Ecological Modelling 126, 139-154.

\bibitem[504]{key504}Ivan B. Djordjevic
\newblock Markov Chain-Like Quantum Biological Modeling of Mutations, Aging, and Evolution 
\newblock Life 2015, 5, 1518-1538; doi:10.3390/life5031518 

\bibitem[1000]{key1000}R Core Team (2015). 
\newblock R: A language and environment for statistical computing. R Foundation for Statistical Computing, Vienna, Austria.
\newblock URL https://www.R-project.org/.
\end{thebibliography}
\end{document}
