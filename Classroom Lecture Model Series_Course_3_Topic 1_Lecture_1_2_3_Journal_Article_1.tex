%-------------------------------------------------------------------------%
%--------------------Classroom Lecture Model Series-----------------------%
%-------------------------------------------------------------------------%

\begin{document}
\twocolumn
\scriptsize
\begin{frontmatter}
		\title{}
		\author{\corref{cor1}\fnref{fn1}}
		\cortext[cor1]{Corresponding author}
		\address{The Mathematical Learning Space}
		\ead{http://mathlearningspace.weebly.com}	
\end{frontmatter}	

Introduction:
\begin{enumerate}
\item Objective 1:
\item Objective 2:
\item Objective 3:
\end{enumerate}
Conclusion:

Keywords: Ribosome, Endoplasmic Reticulum, Smooth ER Rough ER, Tridiagonal Systems
Vocabulary Words:

\section{Introduction}

\subsection{Plan of the Article}


\section{Topic Review}

\subsection{Topic A: Ribosome}

\subsection{Topic B: Endoplasmic Reticulum}

\subsection{Topic C: Nonlinear Tridiagonal mRNA Translation System}

\subsection{Diagram}

\section{Mathematical Model}

\section{Methods}

\section{Algorithms}

\begin{algorithm}[H]
	\begin{algorithmic}[1]
	\end{algorithmic}
\caption{Ribosome Tridiagonal Systems}
\label{RibosomeTRIDS_1}
\end{algorithm}
\begin{itemize}
	\item Observation 1: \\
	\item Observation 2:  \\
	\item Observation 3:  \\
\end{itemize}	

\section{Results}

\subsection{Tables}

\subsection{Figures}

\subsection{Classroom Discussion}

\section{Bibliography}

\bibliographystyle{plain}
\begin{thebibliography}{00}

\bibitem[1]{key1}Wikipedia contributors. 
\newblock Ribosome.
\newblock Wikipedia, The Free Encyclopedia.

\bibitem[601]{key601}McNew JA, Sogaard M, Lampen NM, Machida S, Ye RR, Lacomis L, Tempst P, Rothman JE, Sollner TH
\newblock Ykt6p, a prenylated SNARE essential for endoplasmic reticulum-Golgi transport.
\newblock J Biol Chem 272:17776-83 (1997) DOI:10.1074/jbc.272.28.17776

\bibitem[1000]{key1000}R Core Team (2015). 
\newblock R: A language and environment for statistical computing. R Foundation for Statistical Computing, Vienna, Austria.
\newblock URL https://www.R-project.org/.


\end{thebibliography}

\end{document}
