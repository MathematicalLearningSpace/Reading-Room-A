%-------------------------------------------------------------------------%
%--------------------Classroom Lecture Model Series-----------------------%
%-------------------------------------------------------------------------%


\begin{document}
\twocolumn
\scriptsize
\begin{frontmatter}
		\title{Music Chabot Interaction For Single and Multi-Minute Solo and Trio Classical and Jazz Music Album Designs for the Piano, Marimba and Guitar}
		\author{\corref{cor1}\fnref{fn1}}
		\cortext[cor1]{Corresponding author}
		\address{The Mathematical Learning Space}
		\ead{http://mathlearningspace.weebly.com}	
\end{frontmatter}	

Introduction:
\begin{enumerate}
\item Objective 1:
\item Objective 2:
\item Objective 3:
\end{enumerate}
Conclusion:

keywords: Composition Forms, Classical Music, Romantic Music, Jazz Music, Music Chabot, Album


\section{Introduction}

\subsection{Plan of the Article}

\begin{enumerate}
\item A Brief Review of Music Theory}
\item A Review of Different Compositional forms for Duets in several music genres: such as Contemporary Classical, Romantic and Jazz
\item A Mathematical Review of the Graph Theory and Linear Algebra for Probability Models of Sequence Relationships Over Time
\item A Presentation of the Results with both tabular information and rectangular plots
\item Additional Variations in Designs dicussed in the classroom
\end{enumerate}

\section{Music Theory Review}

\subsection{Topic A: Compositional Forms}

\centering	
\begin{table}[H]\tiny
	\caption{}	
	\begin{tabular}{r|p{4cm}|l}
		\hline	
		Topic & Description & Reference \\
		\hline 
		\hline 
	\end{tabular}
\end{table}

\subsubsection{Visualization}
\begin{figure}[H]
	\centering
	\begin{minipage}[b]{0.5\linewidth}
	%\includegraphics[scale=0.25]{Example_1_Figure_1.png}
	\end{minipage}\hfill
	\begin{minipage}[b]{0.5\linewidth}
	%\includegraphics[scale=0.25]{Example_1_Figure_2.png}
	\end{minipage}\hfill	
	\begin{minipage}[b]{0.5\linewidth}
	%\includegraphics[scale=0.25]{Example_1_Figure_3.png}
	\end{minipage}\hfill
	\begin{minipage}[b]{0.5\linewidth}
	%\includegraphics[scale=0.25]{Example_1_Figure_4.png}
	\end{minipage}\hfill
	\caption{1, 2, 3 and 4}
	\label{fig:Figure1}
\end{figure} 

\subsection{Topic B: Classical Music Examples}

\centering	
\begin{table}[H]\tiny
	\caption{}	
	\begin{tabular}{r|p{4cm}|l}
		\hline	
		Topic & Description & Reference \\
		\hline 
		\hline 
	\end{tabular}
\end{table}

\subsection{Tables}

\begin{table}[H]
	\caption{Composition Motifs in Album with Duration in Seconds and Acoustic Complexity}	
	\begin{tabular}{p{1cm}p{4cm}p{2cm}p{2cm}}
	\hline
	Track No & Name & Duration (seconds) & Acoustic Complexity\\
	\hline
	\hline 
	\end{tabular}
\end{table}



\subsubsection{Visualization}
\begin{figure}[H]
	\centering
	\begin{minipage}[b]{0.5\linewidth}
	%\includegraphics[scale=0.25]{Example_2_Figure_1.png}
	\end{minipage}\hfill
	\begin{minipage}[b]{0.5\linewidth}
	%\includegraphics[scale=0.25]{Example_2_Figure_2.png}
	\end{minipage}\hfill	
	\begin{minipage}[b]{0.5\linewidth}
	%\includegraphics[scale=0.25]{Example_2_Figure_3.png}
	\end{minipage}\hfill
	\begin{minipage}[b]{0.5\linewidth}
	%\includegraphics[scale=0.25]{Example_2_Figure_4.png}
	\end{minipage}\hfill
	\caption{1, 2, 3 and 4}
	\label{fig:Figure1}
\end{figure} 



\subsection{Topic C: Romantic Music Examples}

\centering	
\begin{table}[H]\tiny
	\caption{}	
	\begin{tabular}{r|p{4cm}|l}
		\hline	
		Topic & Description & Reference \\
		\hline 
		\hline 
	\end{tabular}
\end{table}

\subsubsection{Visualization}

\begin{figure}[H]
	\centering
	\begin{minipage}[b]{0.5\linewidth}
	%\includegraphics[scale=0.25]{Example_3_Figure_1.png}
	\end{minipage}\hfill
	\begin{minipage}[b]{0.5\linewidth}
	%\includegraphics[scale=0.25]{Example_3_Figure_2.png}
	\end{minipage}\hfill	
	\begin{minipage}[b]{0.5\linewidth}
	%\includegraphics[scale=0.25]{Example_3_Figure_3.png}
	\end{minipage}\hfill
	\begin{minipage}[b]{0.5\linewidth}
	%\includegraphics[scale=0.25]{Example_3_Figure_4.png}
	\end{minipage}\hfill
	\caption{1, 2, 3 and 4}
	\label{fig:Figure1}
\end{figure} 

\subsection{Topic D: Jazz Music}

\centering	
\begin{table}[H]\tiny
	\caption{}	
	\begin{tabular}{r|p{4cm}|l}
		\hline	
		Topic & Description & Reference \\
		\hline 
		\hline 
	\end{tabular}
\end{table}

\subsubsection{Visualization}

\begin{figure}[H]
	\centering
	\begin{minipage}[b]{0.5\linewidth}
	%\includegraphics[scale=0.25]{Example_4_Figure_1.png}
	\end{minipage}\hfill
	\begin{minipage}[b]{0.5\linewidth}
	%\includegraphics[scale=0.25]{Example_4_Figure_2.png}
	\end{minipage}\hfill	
	\begin{minipage}[b]{0.5\linewidth}
	%\includegraphics[scale=0.25]{Example_4_Figure_3.png}
	\end{minipage}\hfill
	\begin{minipage}[b]{0.5\linewidth}
	%\includegraphics[scale=0.25]{Example_4_Figure_4.png}
	\end{minipage}\hfill
	\caption{1, 2, 3 and 4}
	\label{fig:Figure1}
\end{figure} 

\section{Mathmatical Background}

\subsection{Graph Theory: Music Composition Models}

\begin{figure}[H]
	\centering
	\begin{minipage}[b]{0.5\linewidth}
	%\includegraphics[scale=0.25]{Example_5_Figure_1.png}
	\end{minipage}\hfill
	\begin{minipage}[b]{0.5\linewidth}
	%\includegraphics[scale=0.25]{Example_5_Figure_2.png}
	\end{minipage}\hfill	
	\begin{minipage}[b]{0.5\linewidth}
	%\includegraphics[scale=0.25]{Example_5_Figure_3.png}
	\end{minipage}\hfill
	\begin{minipage}[b]{0.5\linewidth}
	%\includegraphics[scale=0.25]{Example_5_Figure_4.png}
	\end{minipage}\hfill
	\caption{1, 2, 3 and 4}
	\label{fig:Figure1}
\end{figure} 

\subsection{Graph Theory: Toplogical Metrics}

\centering	
\begin{table}[H]\tiny
	\caption{}	
	\begin{tabular}{r|p{4cm}|l}
		\hline	
		Model & Description & Results \\
		\hline 
		\hline 
	\end{tabular}
\end{table}


\section{Results}

\subsection{Data}

Table 1 provides the themes and instrument collection for each of the compositions.
	
\begin{table}[H]
\caption{Composition Collection}	
\begin{tabular}{p{1cm}p{4cm}p{2cm}p{1cm}p{1cm}}
\hline
ID & Name & Theme Pattern & Instruments & \\
\hline 
1 & Composition I &  &  & \\
2 & Composition II &  &  & \\
3 & Composition III &  & \\
4 & Composition IV & & \\
5 & Composition V & & & \\
\hline 
6 & Composition VI &  &  & \\
7 & Composition VII &  &  & \\
8 & Composition VIII &  & \\
9 & Composition IX & & \\
10 & Composition X & & & \\
\end{tabular}
\end{table}




\subsection{Tables}

\centering	
\begin{table}[H]\tiny
	\caption{}	
	\begin{tabular}{r|p{4cm}|l}
		\hline	
		Model & Description & Results \\
		\hline 
		\hline 
	\end{tabular}
\end{table}


\subsection{Figures}

\begin{figure}[H]
	\centering
	\begin{minipage}[b]{0.5\linewidth}
	%\includegraphics[scale=0.25]{Example_6_Figure_1.png}
	\end{minipage}\hfill
	\begin{minipage}[b]{0.5\linewidth}
	%\includegraphics[scale=0.25]{Example_6_Figure_2.png}
	\end{minipage}\hfill	
	\begin{minipage}[b]{0.5\linewidth}
	%\includegraphics[scale=0.25]{Example_6_Figure_3.png}
	\end{minipage}\hfill
	\begin{minipage}[b]{0.5\linewidth}
	%\includegraphics[scale=0.25]{Example_6_Figure_4.png}
	\end{minipage}\hfill
	\caption{1, 2, 3 and 4}
	\label{fig:Figure1}
\end{figure} 


\section{Additional Topics in Classroom for Discussion}

\begin{enumerate}
\end{enumerate}


\bibliographystyle{plain}
\begin{thebibliography}{00}

\bibitem[1]{key100}Sueur J., Aubin T., Simonis C. (2008). 
\newblock Seewave: a free modular tool for sound analysis and synthesis. 
\newblock Bioacoustics, 18: 213-226.

\bibitem[2]{key101}Uwe Ligges, Sebastian Krey, Olaf Mersmann, and Sarah Schnackenberg (2016). 
\newblock tuneR: Analysis of music. 
\newblock URL: http://r-forge.r-project.org/projects/tuner/.

\bibitem[3]{key102}A. Anikin (2017). 
\newblock soundgen: Parametric Voice Synthesis. 
\newblock R package version 1.1.1.

\end{thebibliography}

\end{document}
