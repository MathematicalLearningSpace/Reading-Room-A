%-------------------------------------------------------------------------%
%--------------------Classroom Lecture Model Series-----------------------%
%-------------------------------------------------------------------------%

%----------------------Work in Progress-----------------------------------%

\begin{document}
\addtobeamertemplate{block end}{}{\vspace*{4ex}} 
\addtobeamertemplate{block alerted end}{}{\vspace*{4ex}} 
\setlength{\belowcaptionskip}{2ex}
\setlength\belowdisplayshortskip{2ex}
\begin{frame}[t]
\begin{columns}[t]
\begin{column}{\onecolwid} % The first column
%----------------------------------------------------------------------------------------
%	Objectives
%----------------------------------------------------------------------------------------
\begin{alertblock}{Abstract and Project Objectives}
Introduction:
\begin{itemize}
\item \textit{Objective 1}: 
\item \textit{Objective 2}: 
\item \textit{Objective 3}: 
\end{itemize}
Conclusion:
\end{alertblock}
%----------------------------------------------------------------------------------------
%	Keywords
%----------------------------------------------------------------------------------------
\begin{block}{Keywords}
\begin{itemize}
\item Music Sequences
\item Classification of Sequences
\item Chord Classification
\end{itemize}	
\end{block}
%----------------------------------------------------------------------------------------
%	Introduction
%----------------------------------------------------------------------------------------
\begin{alertblock}{Introduction}
\end{alertblock}
%----------------------------------------------------------------------------------------
%	Music
%----------------------------------------------------------------------------------------
\begin{alertblock}{Review: Music}

\begin{table}[H]
	\centering
	\begin{tabular}{r|p{12cm}|l}
	\hline
	Article ID  & Summary & Comments \\
	\hline
	\hline
	\end{tabular}
	\caption{Summary of Literature Review}
\end{table}		

\end{alertblock}

\end{column}

\begin{column}{\onecolwid} % The second column
%----------------------------------------------------------------------------------------
%	Definitions
%----------------------------------------------------------------------------------------
\begin{alertblock}{Definitions}
\end{alertblock}
%----------------------------------------------------------------------------------------
%	Propositions
%----------------------------------------------------------------------------------------
\begin{alertblock}{Propositions}
\end{alertblock}
%----------------------------------------------------------------------------------------
%	Lemmas
%----------------------------------------------------------------------------------------
\begin{alertblock}{Lemmas}
\end{alertblock}
%----------------------------------------------------------------------------------------
%	Theorems
%----------------------------------------------------------------------------------------
\begin{alertblock}{Theorems}
\end{alertblock}
%----------------------------------------------------------------------------------------
%	Mathematics
%----------------------------------------------------------------------------------------
\begin{alertblock}{Review: Mathematics}

\begin{table}[H]
	\centering
	\begin{tabular}{r|p{12cm}|l}
	\hline
	Article ID  & Summary & Comments \\
	\hline
	\hline
	\end{tabular}
	\caption{Summary of Literature Review}
\end{table}		

\end{alertblock}
\end{column}
\begin{column}{\onecolwid} % The Third column
%----------------------------------------------------------------------------------------
%	Equation System
%----------------------------------------------------------------------------------------
\begin{alertblock}{Equation System}

\begin{equation}
X=
\begin{bmatrix}
\end{bmatrix}
\end{equation}  

\begin{equation}
Y=
\begin{bmatrix}
\end{bmatrix}
\end{equation}  

\begin{equation}
W=
\begin{bmatrix}
\end{bmatrix}
\end{equation}  

\end{alertblock}
%----------------------------------------------------------------------------------------
%	Parameter Table
%----------------------------------------------------------------------------------------
\begin{alertblock}{Parameter Table}

\vspace{4pt}
\centering
\begin{table}[h]\footnotesize
	\caption{Parameter Description and Value}
	\begin{tabular}{rllp{2cm}l}
		\hline	
		Parameter & Value & Interval & Description & Reference \\
		\hline 
	\end{tabular}	
\end{table}

\end{alertblock}
%----------------------------------------------------------------------------------------
%	Algorithms
%----------------------------------------------------------------------------------------
\begin{alertblock}{Review: Algorithms}
\end{alertblock}
\end{column}
\begin{column}{\onecolwid} % The Fourth column
%----------------------------------------------------------------------------------------
%	Results
%----------------------------------------------------------------------------------------
\begin{alertblock}{Results}
\end{alertblock}
%----------------------------------------------------------------------------------------
%	Data
%----------------------------------------------------------------------------------------
\begin{alertblock}{Data}
Table 1 provides the themes and instrument collection for each of the compositions.

\begin{table}[H]
\caption{Composition Collection}	
\begin{tabular}{p{1cm}p{4cm}p{2cm}p{1cm}p{1cm}}
\hline
ID & Name & Theme Pattern & Instruments & \\
\hline 
1 & Composition I &  &  & \\
2 & Composition II &  &  & \\
3 & Composition III &  & \\
4 & Composition IV & & \\
5 & Composition V & & & \\
\hline 
6 & Composition VI &  &  & \\
7 & Composition VII &  &  & \\
8 & Composition VIII &  & \\
9 & Composition IX & & \\
10 & Composition X & & & \\
\end{tabular}
\end{table}
\end{alertblock}
%----------------------------------------------------------------------------------------
%	Tables
%----------------------------------------------------------------------------------------
\begin{alertblock}{Tables}
\end{alertblock}
%----------------------------------------------------------------------------------------
%	Figures
%----------------------------------------------------------------------------------------
\begin{alertblock}{Figures}
\end{alertblock}
%----------------------------------------------------------------------------------------
%	Classroom Discussion Topics
%----------------------------------------------------------------------------------------
\begin{alertblock}{Classroom Discussion Topics}
\end{alertblock}
%----------------------------------------------------------------------------------------
%	Additional Resources
%----------------------------------------------------------------------------------------
\begin{alertblock}{Additional Resources}
\begin{enumerate}
\end{enumerate}
\end{alertblock}
\bibliographystyle{plain}
\begin{thebibliography}{00}
\footnotesize		 
\bibitem[1]{key1}\href{}Creative Sequences Techniques for Music Production
\bibitem[2]{key2}\href{}A MIDI Sequencer that widens access to the compositional possibilties of novel tunings
\bibitem[3]{key3}\href{}Mondrian Music Description Language and Sequencer
\bibitem[4]{key4}\href{}Subvision schemes and multiresolution modelling for automated music synthesis and analysis
\bibitem[5]{key5}\href{}A Differential Equation Based Approach to Sound Synthesis and Sequencing
\bibitem[6]{key6}\href{}Sound Synthesis and Sampling
\bibitem[7]{key7}\href{}Algorithmic Clustering of Music Based on String Compression
\bibitem[8]{key8}\href{}A Survey of Computer Systems for Expressive Music Performance
\bibitem[9]{key9}\href{}Procedural Sequencing: A Form of Prpocedural Music Creation
\bibitem[10]{key10}\href{}SentiMozart: Music Generation based on Emotions
\bibitem[11]{key11}\href{}DeepJ: Style-Specific Music Generation
\bibitem[12]{key12}\href{}An End to End Model for Automatic Music Generation: Combining Deep Raw and Symbolic Audio Networks
\bibitem[13]{key13}\href{}The Challenge of Realistic Music Generation:Modelling raw audio at scale
\bibitem[14]{key14}\href{}Sound Synthesis Based on Ordinary Differential Equations
\bibitem[15]{key15}\href{}The Euclidean Algorithm Generates Traditional Musical Rhythms
\bibitem[16]{key16}\href{}Apply Learning Algorithms to Music Generation
\bibitem[17]{key17}\href{}On the Evaluation of Generative Models in Music
\bibitem[18]{key18}\href{}Project Milestone: Generating music with Machine Learning
\bibitem[19]{key19}\href{}Xiaolce Band: A Melody and Arrangement Generation Framework for Pop Music
\bibitem[1000]{key1000}R Core Team (2015). 
\newblock R: A language and environment for statistical computing. R Foundation for Statistical Computing, Vienna, Austria.
\newblock URL https://www.R-project.org/.		
\end{thebibliography}
\end{column}
\end{columns}
\end{frame}
\end{document}
