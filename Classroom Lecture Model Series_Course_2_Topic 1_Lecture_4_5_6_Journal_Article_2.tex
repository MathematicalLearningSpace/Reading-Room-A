%-------------------------------------------------------------------------%
%--------------------Classroom Lecture Model Series-----------------------%
%-------------------------------------------------------------------------%
\begin{document}
\twocolumn
\scriptsize
\begin{frontmatter}
		\title{}
		\author{\corref{cor1}\fnref{fn1}}
		\cortext[cor1]{Corresponding author}
		\address{The Mathematical Learning Space}
		\ead{http://mathlearningspace.weebly.com}	
\end{frontmatter}	

Introduction:
\begin{enumerate}
\item Objective 1:
\item Objective 2:
\item Objective 3:
\end{enumerate}
Conclusion:

Keywords: Signal Transduction, Gene Mutations, Cancer
Vocabulary Words:

\section{Introduction}

\begin{table}[H]\centering
	\begin{tabular}{p{1cm}p{4cm}p{3cm}}
		Article ID & Summary & Comments\\
		\hline
		\hline
	\end{tabular}
\end{table}

\subsection{Biology Review Topic: Signal Transduction Networks}

\begin{table}[H]\centering
	\begin{tabular}{p{1cm}p{4cm}p{3cm}}
		Article ID & Summary & Comments\\
		\hline
		\hline
	\end{tabular}
\end{table}

\subsection{Biology Review Topic: Cancer Mutations}

\begin{table}[H]\centering
	\begin{tabular}{p{1cm}p{4cm}p{3cm}}
		Article ID & Summary & Comments\\
		\hline
		\hline
	\end{tabular}
\end{table}


\begin{table}[H]
\tiny
\caption{Gene Mutations (GM) for Cancer Type (CT) and Signal Transduction Pathway (STP)}
\begin{tabular}{rp{2cm}p{4cm}p{0.5cm}}
\hline 
ID & CT or STP & GM & Comment \\ 
\hline 	
\hline
\end{tabular}
\end{table}

\subsection{Plan of the Article}

\begin{enumerate}
\end{enumerate}


\section{Mathematical Background}


\subsection{QSAR Properties}

\subsubsection{Molecular Descriptors}

\begin{table}[H]\centering
	\begin{tabular}{p{1cm}p{2cm}p{4cm}p{3cm}}
		ID & Descriptor & Summary & Comments\\
		\hline
		\hline
	\end{tabular}
\end{table}


\begin{figure}[H]
%--------------------------------------------------------------------------------------------
%     Motif A
%--------------------------------------------------------------------------------------------
\begin{tikzpicture}[scale=0.5]
\begin{axis}[width=0.5\textwidth, xlabel={$T-hours$},ylabel={$Y$}]
\addplot[domain=0:48,color=red] {10};
\addplot[domain=0:48,color=blue]{10+cos(2^(4)*x)};
\node[above,red] at (80,0.1) {\footnotesize (A) Y=10};
\node[above,blue] at(150,180) {\footnotesize (B) Y=$10+sin(2^{4}x)$};
\end{axis}
\end{tikzpicture}
%--------------------------------------------------------------------------------------------
%     Motif B
%--------------------------------------------------------------------------------------------
\begin{tikzpicture}[scale=0.5]
\begin{axis}[width=0.5\textwidth, xlabel={$T-hours$},ylabel={$Y$}]
\addplot[domain=0:48,color=red] {5};
\addplot[domain=0:48,color=blue]{5+sin(2^(6)*x)};
\node[above,red] at (80,0.1) {\footnotesize Y=5};
\node[above,blue] at(150,180) {\footnotesize (B) Y=$5+sin(2^{6}x)$};
\end{axis}
\end{tikzpicture}
\caption{}
\end{figure}

\section{Results}


\centering	
\begin{table}[H]\tiny
 \caption{Gene/Protein/Enzymes Description and References}	
\begin{tabular}{r|p{4cm}|llll}
\hline	
Gene/Protein & Description &  Feature 1 & Feature 2 & Feature 3 & Reference \\
\hline 
\hline 
\end{tabular}
\end{table}
\vspace{4pt}

\section{Conclusions}


\section{Topics for the Classroom}

\begin{enumerate}
\end{enumerate}

\section{R Application Programming Interfaces (APIs)}




\bibliographystyle{plain}
\begin{thebibliography}{00}

\bibitem[4]{key4} D.A.J. van Zwieten1, J.E. Rooda1,D.Armbruster and J.D. Nagy
\newblock Simulating feedback and reversibility in substrate-enzyme reactions
\newblock Eur. Phys. J. B 84, 673–684 (2011) DOI: 10.1140/epjb/e2011-10911-x

\bibitem[5]{key5} Pilwon Kima and Chang Hyeong Leeb
\newblock A probability generating function method for stochastic reaction networks
\newblock THE JOURNAL OF CHEMICAL PHYSICS 136, 234108 (2012)

\bibitem[400]{key400} Kanehisa, Furumichi, M., Tanabe, M., Sato, Y., and Morishima, K.; 
\newblock KEGG: new perspectives on genomes, pathways, diseases and drugs. 
\newblock Nucleic Acids Res. 45, D353-D361 (2017).

\bibitem[401]{key401} Kanehisa, M., Sato, Y., Kawashima, M., Furumichi, M., and Tanabe, M.; 
\newblock KEGG as a reference resource for gene and protein annotation. 
\newblock Nucleic Acids Res. 44, D457-D462 (2016).

\bibitem[402]{key402} Kanehisa, M. and Goto, S.; 
\newblock KEGG: Kyoto Encyclopedia of Genes and Genomes. 
\newblock Nucleic Acids Res. 28, 27-30 (2000). 

\bibitem[403]{key403} Rouillard AD, Gundersen GW, Fernandez NF, Wang Z, Monteiro CD, McDermott MG, Ma'ayan A. 
\newblock The harmonizome: a collection of processed datasets gathered to serve and mine knowledge about genes and proteins. 
\newblock Database (Oxford). 2016 Jul 3;2016. pii: baw100.

\bibitem[1000]{key1000}R Core Team (2015). 
\newblock R: A language and environment for statistical computing. R Foundation for Statistical Computing, Vienna, Austria.
\newblock URL https://www.R-project.org/.


\end{thebibliography}
\end{document}
