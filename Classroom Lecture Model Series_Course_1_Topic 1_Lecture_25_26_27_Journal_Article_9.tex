%-------------------------------------------------------------------------%
%--------------------Classroom Lecture Model Series-----------------------%
%-------------------------------------------------------------------------%

\begin{document}
\twocolumn
\scriptsize
\begin{frontmatter}
		\title{DRAFT Sample: A Delayed Differential Equation Model of Phytochemicals in GastroIntestinal Cancers}
		\author{\corref{cor1}\fnref{fn1}}
		\cortext[cor1]{Corresponding author}
		\address{The Mathematical Learning Space}
		\ead{http://mathlearningspace.weebly.com}	
\end{frontmatter}	

Introduction:
\begin{enumerate}
\item Objective 1:
\item Objective 2:
\item Objective 3:
\end{enumerate}
Conclusion:

Keywords: Differential Equations, Stability, Phase Space
Vocabulary Words:

\section{Introduction}

\begin{table}[H]\centering
	\begin{tabular}{p{1cm}p{4cm}p{3cm}}
		Article ID & Summary & Comments\\
		\hline
		\hline
	\end{tabular}
\end{table}

\subsection{Plan of the Article}

\begin{enumerate}
\end{enumerate}

\section{Mathematical Background}


\subsection{Toplogical Dynamics}

\begin{table}[H]\centering
	\begin{tabular}{p{1cm}p{4cm}p{3cm}}
		Article ID & Summary & Comments\\
		\hline
		\hline
	\end{tabular}
\end{table}

\subsection{Phase Space Analysis}

\subsection{Stability}
\centering
\begin{table}[H]\footnotesize
	\caption{}
	\begin{tabular}{rp{1cm}p{2cm}p{3cm}p{1cm}}
		\hline
		ID & A & B & C & Reference \\
		\hline
		\hline
	\end{tabular}
\end{table}
\raggedright


\section{Results}


\subsection{Tables}

\centering
\begin{table}[H]\footnotesize
	\caption{}
	\begin{tabular}{rp{1cm}p{2cm}p{3cm}p{1cm}}
		\hline
		ID & A & B & C & Reference \\
		\hline
		\hline
	\end{tabular}
\end{table}
\raggedright

\centering
\begin{table}[H]\footnotesize
	\caption{}
	\begin{tabular}{rp{1cm}p{2cm}p{3cm}p{1cm}}
		\hline
		ID & A & B & C & Reference \\
		\hline
		\hline
	\end{tabular}
\end{table}
\raggedright

\centering
\begin{table}[H]\footnotesize
	\caption{}
	\begin{tabular}{rp{1cm}p{2cm}p{3cm}p{1cm}}
		\hline
		ID & A & B & C & Reference \\
		\hline
		\hline
	\end{tabular}
\end{table}
\raggedright


\subsection{Figures}

\begin{figure}[H]
	\centering
	\begin{minipage}[b]{0.5\linewidth}
	%\includegraphics[scale=0.25]{Example_1_Figure_1.png}
	\end{minipage}\hfill
	\begin{minipage}[b]{0.5\linewidth}
	%\includegraphics[scale=0.25]{Example_1_Figure_2.png}
	\end{minipage}\hfill	
	\begin{minipage}[b]{0.5\linewidth}
	%\includegraphics[scale=0.25]{Example_1_Figure_3.png}
	\end{minipage}\hfill
	\begin{minipage}[b]{0.5\linewidth}
	%\includegraphics[scale=0.25]{Example_1_Figure_4.png}
	\end{minipage}\hfill
	\caption{1, 2, 3 and 4}
	\label{fig:Figure1}
\end{figure} 



\section{Conclusions and Discussion Topics}

\begin{table}[H]\centering
	\begin{tabular}{p{1cm}p{4cm}p{3cm}}
		Topic ID & Summary & Comments\\
		\hline
		\hline
	\end{tabular}
\end{table}


\section{R Application Programming Interfaces (APIs)}




\bibliographystyle{plain}
\begin{thebibliography}{00}

\subsection{Main Classroom Article}

\bibitem[100]{key100}Cali, Yongli, Yun Kang, Weiming Wang (2017)
\newblock \textbf{A Stochastic SIRS epidemic model with nonlinear incidence rate}
\newblock Applied Mathematics and Computation 305: 221-240.


\subsection{Stability and Convergence Analysis}

\bibitem[200]{key200}Bahar, Arifah and Mao, Xuerong (2004) 
\newblock Stochastic delay Lotka-Volterra model. 
\newblock Journal of Mathematical Analysis and Applications, 292 (2). pp. 364-380. ISSN 0022-247X , http://dx.doi.org/10.1016
/j.jmaa.2003.12.004

\bibitem[201]{key201} X.Li,X.Mao
\newblock Population dynamical behavior of non-autonomous Lotka–Volterra competitive system with random perturbation, 
\newblock Discret. Contin. Dyn. Syst. 24 (2) (2009) 523–593 .

\bibitem[202]{key202} M. Liu, M.Fan, 
\newblock Permanence of stochastic Lotka–Volterra systems,
\newblock J. Nonlinear Sci. (2016), doi:10.1007/s00332-016-9337-2.

\bibitem[203]{key203} A. Lahrouz, L. Omari,
\newblock Extinction and stationary distribution of a stochastic SIRS epidemic model with non-linear incidence,
\newblock Stat. Probab. Lett. 83 (2013) 960–968.

\bibitem[204]{key204} Q.Yang, X. Mao,
\newblock Stochastic dynamics of SIRS epidemic models with random perturbation,
\newblock Math. Biosci. Eng. 1

\bibitem[205]{key205} C. Xu
\newblock Global threshold dynamics of a stochastic differential equation SIS model
\newblock J. Math. Anal. Appl. 447 (2017) 736–757.

\bibitem[206]{key206}Arifah, B. and X. Mao 
\newblock Stochastic Delay Lotka-Volterra Model 
\newblock Journal of Math Anal App. 292 (2) (2004) 364-380.

\bibitem[207]{key207} D.Jiang,N.Shi,X.Li,
\newblock Global stability and stochastic permanence of a non-autonomous logistic equation with random perturbation
\newblock J.Math. Anal. Appl. 340 (1) (2008) 588–597

\bibitem[208]{key208}L.R. Bellet,
\newblock Ergodic properties of Markov processes, in: Open Quantum Systems II,
\newblock Springer, Berlin Heidelberg,2006, pp.1–39.

\bibitem[209]{key209} J. Cushing and J. Hudson, 2012. 
\newblock Evolutionary dynamics and strong Allee effects,
\newblock Journal  of  Biological Dynamics, 6 (2), 941-958.

\bibitem[210]{key210}H.W.Hethcote,
\newblock The mathematics of infectious diseases,
\newblock SIAM Rev. 42(4)(2000) 599–653.

\bibitem[211]{key211}B.Fred,C.-C.Carlos
\newblock Mathematical Models in Population Biology and Epidemiology,
\newblock second ed., Springer,2012.


\bibitem[1000]{key1000}R Core Team (2015). 
\newblock R: A language and environment for statistical computing. R Foundation for Statistical Computing, Vienna, Austria.
\newblock URL https://www.R-project.org/.

\end{thebibliography}
\end{document}
