%-------------------------------------------------------------------------%
%--------------------Classroom Lecture Model Series-----------------------%
%-------------------------------------------------------------------------%

%-----Work in Progress for the Classroom----------------------------------%

\begin{document}
\twocolumn
\scriptsize
\begin{frontmatter}
		\title{}
		\author{\corref{cor1}\fnref{fn1}}
		\cortext[cor1]{Corresponding author}
		\address{The Mathematical Learning Space}
		\ead{http://mathlearningspace.weebly.com}	
\end{frontmatter}	

Introduction:
\begin{enumerate}
\item Objective 1:
\item Objective 2:
\item Objective 3:
\end{enumerate}
Conclusion:

keywords: Compositon Theory, Music Theory

\section{Introduction}


\subsection{Plan of the Article}

\begin{enumerate}
\item Review of Music Theory
\item Review of Music Theory Topics
\item Design of Three Different Albums
\item Analysis of the Design of Each of the Albums with Track Recommendations
\item Additional Topics for Discussion
\end{enumerate}

\section{Music Theory Review}

\subsection{Topic A:}

\centering	
\begin{table}[H]\tiny
	\caption{}	
	\begin{tabular}{r|p{4cm}|l}
		\hline	
		Topic & Description & Reference \\
		\hline 
		\hline 
	\end{tabular}
\end{table}

\begin{table}[H]
	\caption{Composition Motifs in Album with Duration in Seconds and Acoustic Complexity}	
	\begin{tabular}{p{1cm}p{4cm}p{2cm}p{2cm}}
	\hline
	Track No & Name & Duration (seconds) & Acoustic Complexity\\
	\hline
	\hline 
	\end{tabular}
\end{table}

\section{Mathmatical Background}

\section{Item-based Collaborative Filtering Recommender Model}

\centering	
\begin{table}[H]\tiny
	\caption{}	
	\begin{tabular}{r|p{4cm}|l}
		\hline	
		Model & Description & Results \\
		\hline 
		\hline 
	\end{tabular}
\end{table}

\section{Algorithms}

\subsection{Algorithm 1}

\begin{algorithm}[H]
\begin{algorithmic}[1]

\end{algorithmic}
	\caption{Recommender Algorithm I}
	\label{Recommender_1}
\end{algorithm}

\subsection{Algorithm 2}

\begin{algorithm}[H]
\begin{algorithmic}[1]

\end{algorithmic}
\caption{Recommender Algorithm II}
\label{Recommender_1}
\end{algorithm}

\subsection{Evaluation Top-N recommendations}

\centering	
\begin{table}[H]\tiny
	\caption{}	
	\begin{tabular}{r|p{4cm}|l}
		\hline	
		Model & Description & Metric \\
		\hline 
		\hline 
	\end{tabular}
\end{table}

\subsection{Composition Form Model A}

\begin{table}[H]
\caption{Composition Collection}	
\begin{tabular}{p{1cm}p{4cm}p{2cm}p{1cm}p{1cm}}
\hline
ID & Name & Theme Pattern & Instruments & \\
\hline 
A1 & Composition I &  &  & \\
A2 & Composition II &  &  & \\
A3 & Composition III &  & \\
A4 & Composition IV & & \\
A5 & Composition V & & & \\
\hline 
A6 & Composition VI &  &  & \\
A7 & Composition VII &  &  & \\
A8 & Composition VIII &  & \\
A9 & Composition IX & & \\
A10 & Composition X & & & \\
\end{tabular}
\end{table}

\subsection{Composition Form Model B}

\begin{table}[H]
\caption{Composition Collection}	
\begin{tabular}{p{1cm}p{4cm}p{2cm}p{1cm}p{1cm}}
\hline
ID & Name & Theme Pattern & Instruments & \\
\hline 
B1 & Composition I &  &  & \\
B2 & Composition II &  &  & \\
B3 & Composition III &  & \\
B4 & Composition IV & & \\
B5 & Composition V & & & \\
\hline 
B6 & Composition VI &  &  & \\
B7 & Composition VII &  &  & \\
B8 & Composition VIII &  & \\
B9 & Composition IX & & \\
B10 & Composition X & & & \\
\end{tabular}
\end{table}

\subsection{Composition Form Model C}


\begin{table}[H]
\caption{Composition Collection}	
\begin{tabular}{p{1cm}p{4cm}p{2cm}p{1cm}p{1cm}}
\hline
ID & Name & Theme Pattern & Instruments & \\
\hline 
C1 & Composition I &  &  & \\
C2 & Composition II &  &  & \\
C3 & Composition III &  & \\
C4 & Composition IV & & \\
C5 & Composition V & & & \\
\hline 
C6 & Composition VI &  &  & \\
C7 & Composition VII &  &  & \\
C8 & Composition VIII &  & \\
C9 & Composition IX & & \\
C10 & Composition X & & & \\
\end{tabular}
\end{table}


\section{Results}

\subsection{Tables}

\centering	
\begin{table}[H]\tiny
	\caption{}	
	\begin{tabular}{r|p{4cm}|l}
		\hline	
		Model & Description & Results \\
		\hline 
		\hline 
	\end{tabular}
\end{table}

\subsection{Figures}

\begin{figure}[H]
	\centering
	\begin{minipage}[b]{0.5\linewidth}
	%\includegraphics[scale=0.25]{Example_1_Figure_1.png}
	\end{minipage}\hfill
	\begin{minipage}[b]{0.5\linewidth}
	%\includegraphics[scale=0.25]{Example_1_Figure_2.png}
	\end{minipage}\hfill	
	\begin{minipage}[b]{0.5\linewidth}
	%\includegraphics[scale=0.25]{Example_1_Figure_3.png}
	\end{minipage}\hfill
	\begin{minipage}[b]{0.5\linewidth}
	%\includegraphics[scale=0.25]{Example_1_Figure_4.png}
	\end{minipage}\hfill
	\caption{1, 2, 3 and 4}
	\label{fig:Figure1}
\end{figure} 


\bibliographystyle{plain}
\begin{thebibliography}{00}

\subsection{Recommender Systems}

\bibitem[300]{key300}Desrosiers C, Karypis G (2011). 
\newblock A Comprehensive Survey of Neighborhood-based Recommendation Methods.” In F Ricci, L Rokach, B Shapira, PB Kantor (eds.), 
\newblock Recommender Systems Handbook, chapter 4, pp. 107–144. Springer US, Boston, MA. ISBN 978-0-38785819-7

\bibitem[301]{key301}Deshpande M, Karypis G (2004). 
\newblock Item-based top-N recommendation algorithms.
\newblock ACM Transations on Information Systems, 22(1), 143–177. ISSN 1046-8188.

\bibitem[302]{key302}Koren Y, Bell R, Volinsky C (2009). 
\newblock “Matrix Factorization Techniques for Recommender Systems.” 
\newblock Computer, 42, 30–37. doi:http://doi.ieeecomputersociety.org/10.1109/ MC.2009.263.


\end{thebibliography}

\end{document}
