%-------------------------------------------------------------------------%
%--------------------Classroom Lecture Model Series-----------------------%
%-------------------------------------------------------------------------%

%-----Work in Progress for the Classroom----------------------------------%

\begin{document}
\twocolumn
\scriptsize
\begin{frontmatter}
		\title{}
		\author{\corref{cor1}\fnref{fn1}}
		\cortext[cor1]{Corresponding author}
		\address{The Mathematical Learning Space}
		\ead{http://mathlearningspace.weebly.com}	
\end{frontmatter}	

Introduction:
\begin{enumerate}
\item Objective 1:
\item Objective 2:
\item Objective 3:
\end{enumerate}
Conclusion:

keywords: Compositon Theory, Music Theory

\section{Introduction}


\subsection{Plan of the Article}

\begin{enumerate}
\item Review of Music Theory
\item Review of Music Theory Topics
\item Design of Several 100 Measure Compositional Forms
\item Analysis of the Design of Each of the Forms
\end{enumerate}

\section{Music Theory Review}

\subsection{Topic A:}

\centering	
\begin{table}[H]\tiny
	\caption{}	
	\begin{tabular}{r|p{4cm}|l}
		\hline	
		Topic & Description & Reference \\
		\hline 
		\hline 
	\end{tabular}
\end{table}

\section{Mathmatical Background}

\section{Item-based Collaborative Filtering Recommender Model}

\begin{algorithm}[H]
\begin{algorithmic}[1]

\end{algorithmic}
	\caption{Recommender Algorithm I}
	\label{Recommender_1}
\end{algorithm}

\begin{algorithm}[H]
\begin{algorithmic}[1]

\end{algorithmic}
\caption{Recommender Algorithm II}
\label{Recommender_1}
\end{algorithm}

\subsection{ Evaluation Top-N recommendations}


\subsection{Composition Form Model A}

\subsection{Composition Form Model B}

\subsection{Composition Form Model C}

\section{Results}

\subsection{Tables}

\centering	
\begin{table}[H]\tiny
	\caption{}	
	\begin{tabular}{r|p{4cm}|l}
		\hline	
		Model & Description & Results \\
		\hline 
		\hline 
	\end{tabular}
\end{table}

\subsection{Figures}

\begin{figure}[H]
	\centering
	\begin{minipage}[b]{0.5\linewidth}
	%\includegraphics[scale=0.25]{Example_1_Figure_1.png}
	\end{minipage}\hfill
	\begin{minipage}[b]{0.5\linewidth}
	%\includegraphics[scale=0.25]{Example_1_Figure_2.png}
	\end{minipage}\hfill	
	\begin{minipage}[b]{0.5\linewidth}
	%\includegraphics[scale=0.25]{Example_1_Figure_3.png}
	\end{minipage}\hfill
	\begin{minipage}[b]{0.5\linewidth}
	%\includegraphics[scale=0.25]{Example_1_Figure_4.png}
	\end{minipage}\hfill
	\caption{1, 2, 3 and 4}
	\label{fig:Figure1}
\end{figure} 


\bibliographystyle{plain}
\begin{thebibliography}{00}

\subsection{Recommender Systems}

\bibitem[300]{key300}Desrosiers C, Karypis G (2011). 
\newblock A Comprehensive Survey of Neighborhood-based Recommendation Methods.” In F Ricci, L Rokach, B Shapira, PB Kantor (eds.), 
\newblock Recommender Systems Handbook, chapter 4, pp. 107–144. Springer US, Boston, MA. ISBN 978-0-38785819-7

\bibitem[301]{key301}Deshpande M, Karypis G (2004). 
\newblock Item-based top-N recommendation algorithms.
\newblock ACM Transations on Information Systems, 22(1), 143–177. ISSN 1046-8188.

\bibitem[302]{key302}Koren Y, Bell R, Volinsky C (2009). 
\newblock “Matrix Factorization Techniques for Recommender Systems.” 
\newblock Computer, 42, 30–37. doi:http://doi.ieeecomputersociety.org/10.1109/ MC.2009.263.


\end{thebibliography}

\end{document}
