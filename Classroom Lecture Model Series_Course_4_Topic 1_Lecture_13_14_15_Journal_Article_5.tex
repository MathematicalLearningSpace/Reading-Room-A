%-------------------------------------------------------------------------%
%--------------------Classroom Lecture Model Series-----------------------%
%-------------------------------------------------------------------------%


\begin{document}
\twocolumn
\scriptsize
\begin{frontmatter}
		\title{}
		\author{\corref{cor1}\fnref{fn1}}
		\cortext[cor1]{Corresponding author}
		\address{The Mathematical Learning Space}
		\ead{http://mathlearningspace.weebly.com}	
\end{frontmatter}	

Introduction:
\begin{enumerate}
\item Objective 1:
\item Objective 2:
\item Objective 3:
\end{enumerate}
Conclusion:

keywords: Harmonic Analysis, Classical Compositions, Duets and Quartets, Piano, Guitar

\section{Introduction}

\subsection{Plan of the Article}

\begin{enumerate}
\item Review of Music Vocabulary for Harmonics
\item Review of Literature Searches on Harmonic Analysis from the Perspetive of Musical Genre
\item Review of Mathematical methods of harmonic analysis
\end{enumerate}

\section{Music Theory Review}

\subsection{Topic A: Harmonic Analysis}

\centering	
\begin{table}[H]\tiny
	\caption{}	
	\begin{tabular}{r|p{4cm}|l}
		\hline	
		Topic & Description & Reference \\
		\hline 
		\hline 
	\end{tabular}
\end{table}

\subsection{Topic B: Music Instrumentation}

\centering	
\begin{table}[H]\tiny
	\caption{}	
	\begin{tabular}{r|p{4cm}|l}
		\hline	
		Topic & Description & Reference \\
		\hline 
		\hline 
	\end{tabular}
\end{table}

\section{Mathmatical Background}

\subsection{Topic A: Harmonic Analysis}

\subsection{Topic B: Parametric Models}

\section{Results}

\subsection{Data}

Table 1 provides the themes and instrument collection for each of the compositions.
	
\begin{table}[H]
\caption{Composition Collection}	
\begin{tabular}{p{1cm}p{4cm}p{2cm}p{1cm}p{1cm}}
\hline
ID & Name & Theme Pattern & Instruments & \\
\hline 
1 & Composition I &  &  & \\
2 & Composition II &  &  & \\
3 & Composition III &  & \\
4 & Composition IV & & \\
5 & Composition V & & & \\
\hline 
6 & Composition VI &  &  & \\
7 & Composition VII &  &  & \\
8 & Composition VIII &  & \\
9 & Composition IX & & \\
10 & Composition X & & & \\
\end{tabular}
\end{table}


\subsection{Tables}

\centering	
\begin{table}[H]\tiny
	\caption{}	
	\begin{tabular}{r|p{4cm}|l}
		\hline	
		Model & Description & Results \\
		\hline 
		\hline 
	\end{tabular}
\end{table}

\subsection{Figures]

\begin{figure}[H]
	\centering
	\begin{minipage}[b]{0.5\linewidth}
	%\includegraphics[scale=0.25]{Example_1_Figure_1.png}
	\end{minipage}\hfill
	\begin{minipage}[b]{0.5\linewidth}
	%\includegraphics[scale=0.25]{Example_1_Figure_2.png}
	\end{minipage}\hfill	
	\begin{minipage}[b]{0.5\linewidth}
	%\includegraphics[scale=0.25]{Example_1_Figure_3.png}
	\end{minipage}\hfill
	\begin{minipage}[b]{0.5\linewidth}
	%\includegraphics[scale=0.25]{Example_1_Figure_4.png}
	\end{minipage}\hfill
	\caption{1, 2, 3 and 4}
	\label{fig:Figure1}
\end{figure} 



\bibliographystyle{plain}

\begin{thebibliography}{00}
\bibitem[1]{key100}Sueur J., Aubin T., Simonis C. (2008). 
\newblock Seewave: a free modular tool for sound analysis and synthesis. 
\newblock Bioacoustics, 18: 213-226.

\bibitem[2]{key101}Uwe Ligges, Sebastian Krey, Olaf Mersmann, and Sarah Schnackenberg (2016). 
\newblock tuneR: Analysis of music. 
\newblock URL: http://r-forge.r-project.org/projects/tuner/.

\bibitem[3]{key102}A. Anikin (2017). 
\newblock soundgen: Parametric Voice Synthesis. 
\newblock R package version 1.1.1.

\bibitem[4]{key103}Pieretti N, Farina A, Morri FD (2011) 
\newblock A new methodology to infer the singing activity of an avian community: the Acoustic Complexity Index (ACI). 
\newblock Ecological Indicators, 11, 868-873.
\end{thebibliography}

\end{document}
