%-------------------------------------------------------------------------%
%--------------------Classroom Lecture Model Series-----------------------%
%-------------------------------------------------------------------------%

%--------------------Work In Progress for the Classroom-------------------%

\begin{document}
\twocolumn
\scriptsize
\begin{frontmatter}
		\title{DRAFT Sample: Delayed Differential Equations for Intestinal Metaplasia of Gene Expression Modulation in Gastric Cancer}
		\author{\corref{cor1}\fnref{fn1}}
		\cortext[cor1]{Corresponding author}
		\address{The Mathematical Learning Space}
		\ead{http://mathlearningspace.weebly.com}	
\end{frontmatter}	

Introduction:
\begin{enumerate}
\item Objective 1:
\item Objective 2:
\item Objective 3:
\end{enumerate}
Conclusion:

Keywords: Fractional Calculus, Stochastic Calculus, Fractional Differential Equations, Stochastic Diffferential Equations, Mittag-Leffler functions, Laplace Transforms

\section{Introduction}

\begin{enumerate}
	\item A \cite{key1}
\end{enumerate}

\subsection{Review Notes}

\begin{figure}[H]
\begin{minipage}[b]{0.3\linewidth}
\includegraphics[scale=0.40]{Gastric_Cancer.png} 
\end{minipage}\hfill
\caption{(a) Gastric Cancer}
\label{fig:Figure1}
\end{figure} 

\begin{table}[H]\centering
	\begin{tabular}{p{1cm}p{4cm}p{3cm}}
		Article ID & Summary & Comments\\
		\hline
		\hline
	\end{tabular}
\end{table}

\centering	
\begin{table}[H]\tiny
	\caption{Gene/Protein/Enzymes Description and References}	
	\begin{tabular}{r|p{4cm}|l}
		\hline	
		Gene/Protein & Description & Reference \\
		\hline 
		\hline 
	\end{tabular}
\end{table}

\subsection{Plan of the Article}

\begin{enumerate}
\item List Fractional Derivatives
\item List Drift and Diffusion Equations
\item List Algorithms
\item Moment Analysis
\item Equilibrium and Stability Analysis
\item Model Comparision
\item Additional Topics for Discussion
\end{enumerate}

\section{Protein Interaction Diagram and Annotations}

\begin{tikzpicture}
	[->,>=stealth',shorten >=2pt,node distance=3cm,
	thick,main node/.style={circle,draw,scale=0.25,transform canvas={scale=0.75},font=\sffamily\Small\bfseries},
	blacknode/.style={shape=circle, draw=black, line width=2},
	bluenode/.style={shape=circle, draw=blue, line width=2},
	greennode/.style={shape=circle, draw=green, line width=2},
	rednode/.style={shape=circle, draw=red, line width=2}
	]
	%-------Legend ------------------------------------------------------------	
	\matrix [draw,below left] at (current bounding box.south) {
		\node [state,label=right:State] {}; Description. \\
		\node [shapeSquare,label=right:Square-.] {}; \\
		\node [shapeEllipse,label=right:Ellipse-.] {}; \\
		\node [shapeTriangle,label=right:Triangle-.] {}; \\
		\node [shapeHexagon,label=right:Hexagon-.] {}; \\
	};
\end{tikzpicture}


\section{Mathematical Methods}

\begin{table}[H]\centering
	\begin{tabular}{p{1cm}p{4cm}p{3cm}}
		Article ID & Summary & Comments\\
		\hline
		\hline
	\end{tabular}
\end{table}

\subsection{Definitions}

\centering
\begin{table}[H]\footnotesize
	\caption{}
	\begin{tabular}{rp{1cm}p{2cm}p{3cm}p{1cm}}
		\hline
		ID & A & B & C & Reference \\
		\hline
		\hline
	\end{tabular}
\end{table}
\raggedright

\subsection{Theorems}

\subsection{Exponential and Geometric Family Distributions}

\begin{table}[H]\centering
	\begin{tabular}{p{1cm}p{4cm}p{3cm}}
		Equation ID & Description & Comments\\
		\hline
		\hline
	\end{tabular}
\end{table}

\begin{center}
	\begin{tikzpicture}
	[->,>=stealth',shorten >=2pt,node distance=3cm,
	thick,main node/.style={circle,draw,scale=0.25,transform canvas={scale=0.75},font=\sffamily\Small\bfseries},
	blacknode/.style={shape=circle, draw=black, line width=2},
	bluenode/.style={shape=circle, draw=blue, line width=2},
	greennode/.style={shape=circle, draw=green, line width=2},
	rednode/.style={shape=circle, draw=red, line width=2}
	]
%-------Legend ------------------------------------------------------------	
	\matrix [draw,below left] at (current bounding box.south) {
		\node [state,label=right:State] {}; A. \\
		\node [shapeSquare,label=right:Square- B.] {}; \\
		\node [shapeEllipse,label=right:Ellipse-C.] {}; \\
		\node [shapeTriangle,label=right:Triangle-D.] {}; \\
		\node [shapeHexagon,label=right:Hexagon-E.] {}; \\
	};
	\end{tikzpicture}
\end{center}

\subsection{Fractional Derivative Operators}

\begin{equation}
D^{\alpha} y = \frac{1}{\Gamma (\alpha - n)} \int_{0}^{t} \frac{y^n * x}{(t - x)^{\alpha+1-n}} dx
\end{equation}

where $\alpha \in (n,n+1) $.  

Here the ratio of Gamma functions and the ratio of polynomials

\begin{equation}	
D^{\alpha} y x^{u} = \frac{\Gamma(u +1)}{\Gamma(u + 1 - \alpha)} x^{u-\alpha}
\end{equation}

\begin{equation}
\begin{cases}
I_{a}^{\alpha} f(t) = \int_{a}^{t} \frac{(t-\tau)^{a-1}}{\Gamma(\alpha)} f(\tau) d \tau \\
D_{a}^{\alpha} f(t) = \frac{d}{dt} \int_{a}^{t} \frac{(t-\tau)^{a-1}}{\Gamma(1-\alpha)} f(\tau) d \tau \\
\end{cases}
\end{equation}

\begin{equation}
cD^{\alpha} f(t) = \frac{1}{\Gamma(n - \mu)} \int_{0}^{t} \frac{f^{(n)} (\epsilon)}{(t-\epsilon(^{\mu - n + 1})} d\epsilon 
\end{equation}

\begin{equation}
J = \frac{1}{2} || E_{j} ||_{2}^{2} = \frac{1}{2} \sum_{k=1}^{N} (e_j (k))^2 
\end{equation}

$E_j =(e_{j}(1),e_{j}(2),...,e_{j}(N))^{T}$ and

\begin{equation}
e_{j}(k) = f(x_{k},y_{j}(x_{k})) - D_{0}^{\alpha} y_{j}(x_{k})
\end{equation}

\begin{table}[H]\centering
	\begin{tabular}{p{1cm}p{4cm}p{3cm}}
		Operator ID & Description & Comments\\
		\hline
		\hline
	\end{tabular}
\end{table}


\subsection{Equation Systems}

\begin{align*} 
\tiny
\frac{d^{\alpha_1}X_1(t)}{dt^{\alpha_1}} = a_{11} *X_1(t - \tau_1) + \\
a_{12} *\frac{X_1(t)^{\beta_1}}{(1-X_1(t - \tau_1))^{\beta_1}} - a_{15}X_1(t) + \\
\epsilon_1(t) \\
\frac{d^{\alpha_1}X_2(t)}{dt^{\alpha_1}} = a_{21} *X_2(t - \tau_2) + \\
a_{22} *\frac{X_2(t)^{\beta_1}}{(1-X_2(t - \tau_2))^{\beta_1}} - a_{25}X_2(t) + \\
\epsilon_2(t) \\ \\
\frac{d^{\alpha_1}X_3(t)}{dt^{\alpha_1}} = a_{31} *X_3(t - \tau_3) + \\
a_{32} *\frac{X_3(t)^{\beta_1}}{(1-X_3(t - \tau_3))^{\beta_1}} - a_{35}X_3(t) + \\
\epsilon_3(t) \\ \\
\frac{d^{\alpha_1}X_4(t)}{dt^{\alpha_1}} = a_{41} *X_4(t - \tau_4) + \\
a_{42} *\frac{X_4(t)^{\beta_1}}{(1-X_4(t - \tau_4))^{\beta_1}} - a_{45}X_4(t) + \\
\epsilon_4(t) \\ \\
\frac{d^{\alpha_1}X_5(t)}{dt^{\alpha_1}} = a_{51} *X_5(t - \tau_5) + \\
a_{52} *\frac{X_5(t)^{\beta_1}}{(1-X_5(t - \tau_5))^{\beta_1}} - a_{55}X_5(t) + \\
\epsilon_5(t) \\
\end{align*}

\subsubsection{Parameter Table}

\begin{table}[h]\footnotesize
	\caption{Parameter Description and Value}
	\begin{tabular}{rllp{2cm}l}
		\hline	
		Parameter & Value & Interval & Description & Reference \\
		\hline 
		a11 & 0 & [0,1] & Equation 1 & \cite{key1}
		a12 & 0 & [0,1] & Equation 1 & \cite{key1}
		a13 & 0 & [0,1] & Equation 1 & \cite{key1}
		a14 & 0 & [0,1] & Equation 1 & \cite{key1}
		a15 & 0 & [0,1] & Equation 1 & \cite{key1}
		\hline
		a21 & 0 & [0,1] & Equation 2 & \cite{key1}
		a22 & 0 & [0,1] & Equation 2 & \cite{key1}
		a23 & 0 & [0,1] & Equation 2 & \cite{key1}
		a24 & 0 & [0,1] & Equation 2 & \cite{key1}
		a25 & 0 & [0,1] & Equation 2 & \cite{key1}
		\hline
		a31 & 0 & [0,1] & Equation 3 & \cite{key1}
		a32 & 0 & [0,1] & Equation 3 & \cite{key1}
		a33 & 0 & [0,1] & Equation 3 & \cite{key1}
		a34 & 0 & [0,1] & Equation 3 & \cite{key1}
		a35 & 0 & [0,1] & Equation 3 & \cite{key1}
		\hline
		a41 & 0 & [0,1] & Equation 4 & \cite{key1}
		a42 & 0 & [0,1] & Equation 4 & \cite{key1}
		a43 & 0 & [0,1] & Equation 4 & \cite{key1}
		a44 & 0 & [0,1] & Equation 4 & \cite{key1}
		a45 & 0 & [0,1] & Equation 4 & \cite{key1}
		\hline
		a51 & 0 & [0,1] & Equation 5 & \cite{key1}
		a52 & 0 & [0,1] & Equation 5 & \cite{key1}
		a53 & 0 & [0,1] & Equation 5 & \cite{key1}
		a54 & 0 & [0,1] & Equation 5 & \cite{key1}
		a55 & 0 & [0,1] & Equation 5 & \cite{key1}
		\hline
		$\tau_1$ & 1 & [1,2] & Equation 1 & \cite{key1}
		$\tau_2$ & 1 & [1,2] & Equation 2 & \cite{key1}
		$\tau_3$ & 1 & [1,2] & Equation 3 & \cite{key1}
		$\tau_4$ & 1 & [1,2] & Equation 4 & \cite{key1}
		$\tau_5$ & 1 & [1,2] & Equation 5 & \cite{key1}
		\hline
		$\alpha_1$ & 1 & (0,2] & Equation 1 & \cite{key1}
		$\alpha_2$ & 1 & (0,2] & Equation 2 & \cite{key1}
		$\alpha_3$ & 1 & (0,2] & Equation 3 & \cite{key1}
		$\alpha_4$ & 1 & (0,2] & Equation 4 & \cite{key1}
		$\alpha_5$ & 1 & (0,2] & Equation 5 & \cite{key1}
		\hline
		$\beta_1$ & 1 & (0,2] & Equation 1 & \cite{key1}
		$\beta_2$ & 1 & (0,2] & Equation 2 & \cite{key1}
		$\beta_3$ & 1 & (0,2] & Equation 3 & \cite{key1}
		$\beta_4$ & 1 & (0,2] & Equation 4 & \cite{key1}
		$\beta_5$ & 1 & (0,2] & Equation 5 & \cite{key1}
	\end{tabular}	
\end{table}


\begin{table}[H]\centering
	\begin{tabular}{p{1cm}p{4cm}p{3cm}}
		Article ID & Summary & Comments\\
		\hline
		\hline
	\end{tabular}
\end{table}

\subsubsection{Drift}

\centering
\begin{table}[H]\footnotesize
	\caption{}
	\begin{tabular}{rp{1cm}p{2cm}p{3cm}p{1cm}}
		\hline
		ID & A & B & C & Reference \\
		\hline
		\hline
	\end{tabular}
\end{table}
\raggedright

\subsection{Diffusion}

\centering
\begin{table}[H]\footnotesize
	\caption{}
	\begin{tabular}{rp{1cm}p{2cm}p{3cm}p{1cm}}
		\hline
		ID & A & B & C & Reference \\
		\hline
		\hline
	\end{tabular}
\end{table}
\raggedright

\subsection{Stochastic Error}


\section{Algorithms}
%----------------------------------------------------------------------------
%						Algorithm 1
%----------------------------------------------------------------------------		
\begin{algorithm}[H]
\begin{algorithmic}[1]
\end{algorithmic}
\caption{Fractional Differential Equation Computation}
	\label{Algorithm_1}
\end{algorithm}

%----------------------------------------------------------------------------
%						Algorithm 2
%----------------------------------------------------------------------------		
\begin{algorithm}[H]
\begin{algorithmic}[1]
\end{algorithmic}
\caption{Stochastic Differential Equation Computation}
	\label{Algorithm_2}
\end{algorithm}

\section{Results}

\subsection{Tables}

\centering
\begin{table}[H]\footnotesize
	\caption{}
	\begin{tabular}{rp{1cm}p{2cm}p{3cm}p{1cm}}
		\hline
		ID & A & B & C & Reference \\
		\hline
		\hline
	\end{tabular}
\end{table}
\raggedright

\centering
\begin{table}[H]\footnotesize
	\caption{}
	\begin{tabular}{rp{1cm}p{2cm}p{3cm}p{1cm}}
		\hline
		ID & A & B & C & Reference \\
		\hline
		\hline
	\end{tabular}
\end{table}
\raggedright

\centering
\begin{table}[H]\footnotesize
	\caption{}
	\begin{tabular}{rp{1cm}p{2cm}p{3cm}p{1cm}}
		\hline
		ID & A & B & C & Reference \\
		\hline
		\hline
	\end{tabular}
\end{table}
\raggedright

\subsection{Figures}

\subsection{Moment Analysis}
\subsection{Equilibrium and Stability Analysis}
\subsection{Model Comparision}

\section{Additional Topics for the Classroom}

\begin{enumerate}
\item 
\end{enumerate}


\bibliographystyle{plain}
\begin{thebibliography}{00}

\bibitem[100]{key100} Mistry, L. A. Khan, and D.L. Suthar (2016)
\newblock Review on Fractional Differential Equations and their Applications
\newblock Proceedings of International Conference on Emerging Technologies in Engineering, Biomedical, Management and Science 

\bibitem[101]{key101} Odibat Z.M. and S. Momani
\newblock An Algorithm for the Numerical Solution of Differential Equations of Fractional Order
\newblock Journal Applied Mathematical and Informatics 26:(2008) 1:2pp.15-27

\bibitem[102]{key102} Roberto Garrappa
\newblock Numerical Solution of Fractional Differential Equations: A Survey and a Software Tutorial
\newblock Mathematics 2018,6, 16; doi:10.3390/math6020016 www.mdpi.com/journal/mathematics

\bibitem[103]{key103} Nan Xia, Tashpolat Tiyip, Ardak Kelimu, et al., 
\newblock Influence of Fractional Differential on Correlation Coefficient between EC1:5 and Reflectance Spectra of Saline Soil,
\newblock Journal of Spectroscopy, vol. 2017, Article ID 1236329, 11 pages, 2017. https://doi.org/10.1155/2017/1236329.

\bibitem[104]{key104}K. Diethelm, N.J. Ford, A.D. Freed, 
\newblock A predictor–corrector approach for the numerical solution of fractional differential equations, 
\newblock Nonlinear Dyn. 29 (2002) 3–22.
 
\bibitem[105]{key105} K. Diethelm, 
\newblock Detailed error analysis for a fractional Adams method
\newblock Numerical Algorithms, in press

\bibitem[106]{key106} Hilfer,  R.
\newblock Threefold  Introduccion  to  fractional  derivatives.  
\newblock Anomalous transport:  Foundation and applications. (2008)

\bibitem[107]{key107} Ruben A. Cerutti
\newblock The k-Fractional Logistic Equation with k-Caputo Derivative
\newblock Pure Mathematical Sciences, Vol.  4, 2015, no.  1, 9 - 15

\bibitem[201]{key201}B. Ghayebi and S. M. Hosseini
\newblock A Simplified Milstein Scheme for SPDEs with Multiplicative Noise
\newblock Abstract and Applied Analysis Vol. 2014, Article ID 140849

\bibitem[301]{key301}Yu Wang and Tianzeng Li
\newblock Stability Analysis of Fractional-Order Nonlinear Systems with Delay
\newblock Mathematical Problems in Engineering Vol. 2014, Article ID 301235


\subsection{Differential Equations}

\bibitem[1]{key1}A New Approach and Solution Technique to Solve Time Fractional Nonlinear Reaction-Diffusion Equations
\bibitem[1]{key1}Stability Analysis of Fractional-Order Nonlinear Systems with Delay
\bibitem[1]{key1}Application of the Multistep Generalized Differential Transform Method to Solve a Time-Fractional Enzyme Kinetics
\bibitem[1]{key1}Wavelet Methods for Solving Fractional Order Differential Equations
\bibitem[1]{key1}Numerical Methods for Pricing American Options with Time-Fractional PDE Models
\bibitem[1]{key1}Application of Multistep Generalized Differential Transform Method for the Solutions of the Fractional-Order Chua System
\bibitem[1]{key1}Numerical Solution of Some Types of Fractional Optimal Control Problems
\bibitem[1]{key1}An Efficient Series Solution for Fractional Differential Equations
\bibitem[1]{key1}Approximate Analytical Solution for Nonlinear System of Fractional Differential Equations by BPs Operational Matrices
\bibitem[1]{key1}Numerical Solution for Complex Systems of Fractional Order
\bibitem[1]{key1}Stability Analysis of Fractional-Order Nonlinear Systems with Delay
\bibitem[1]{key1}Numerical Study for Time Delay Multistrain Tuberculosis Model of Fractional Order
\bibitem[1]{key1}A Numerical Method for Solving Fractional Differential Equations by Using Neural Network
\bibitem[1]{key1}Numerical Studies for Fractional-Order Logistic Differential Equation with Two Different Delays
\bibitem[1]{key1}Numerical Modeling of Fractional-Order Biological Systems
\bibitem[1]{key1}Numerical Solution of Some Types of Fractional Optimal Control Problems
\bibitem[1]{key1}A Numerical Method for Delayed Fractional-Order Differential Equations
\bibitem[1]{key1} New Insights into the Fractional Order Diffusion Equation Using Entropy and Kurtosis
\bibitem[1]{key1} DELAY DIFFERENTIAL EQUATIONS IN SINGLE SPECIES DYNAMICS
\bibitem[1]{key1} An Improved Artificial Bee Colony Algorithm Based on Elite Strategy and Dimension Learning
\bibitem[1]{key1}Operators of Fractional Calculus and Their Applications
\bibitem[1]{key1} Modelling Physiological and Pharmacological Control on Cell Proliferation to Optimise Cancer Treatments
\bibitem[1]{key1}Press, W. H., S. A. Teukolsky, W. T Vetterling, and B. P. Flannery (2007). 
\newblock Numerical Recipes: The Art of Numerical Computing. 
\newblock Third Edition, Cambridge University Press, New York.

\bibitem[100]{key100} Mistry, L. A. Khan, and D.L. Suthar (2016)
\newblock Review on Fractional Differential Equations and their Applications
\newblock Proceedings of International Conference on Emerging Technologies in Engineering, Biomedical, Management and Science 

\bibitem[101]{key101} Odibat Z.M. and S. Momani
\newblock An Algorithm for the Numerical Solution of Differential Equations of Fractional Order
\newblock Journal Applied Mathematical and Informatics 26:(2008) 1:2pp.15-27

\bibitem[102]{key102} Roberto Garrappa
\newblock Numerical Solution of Fractional Differential Equations: A Survey and a Software Tutorial
\newblock Mathematics 2018,6, 16; doi:10.3390/math6020016 www.mdpi.com/journal/mathematics

\bibitem[103]{key103} Nan Xia, Tashpolat Tiyip, Ardak Kelimu, et al., 
\newblock Influence of Fractional Differential on Correlation Coefficient between EC1:5 and Reflectance Spectra of Saline Soil,
\newblock Journal of Spectroscopy, vol. 2017, Article ID 1236329, 11 pages, 2017. https://doi.org/10.1155/2017/1236329.

\bibitem[104]{key104}K. Diethelm, N.J. Ford, A.D. Freed, 
\newblock A predictor–corrector approach for the numerical solution of fractional differential equations, 
\newblock Nonlinear Dyn. 29 (2002) 3–22.
 
\bibitem[105]{key105} K. Diethelm, 
\newblock Detailed error analysis for a fractional Adams method
\newblock Numerical Algorithms, in press

\bibitem[106]{key106} Hilfer,  R.
\newblock Threefold  Introduccion  to  fractional  derivatives.  
\newblock Anomalous transport:  Foundation and applications. (2008)

\bibitem[107]{key107} Ruben A. Cerutti
\newblock The k-Fractional Logistic Equation with k-Caputo Derivative
\newblock Pure Mathematical Sciences, Vol.  4, 2015, no.  1, 9 - 15

\bibitem[201]{key201}B. Ghayebi and S. M. Hosseini
\newblock A Simplified Milstein Scheme for SPDEs with Multiplicative Noise
\newblock Abstract and Applied Analysis Vol. 2014, Article ID 140849

\bibitem[301]{key301}Yu Wang and Tianzeng Li
\newblock Stability Analysis of Fractional-Order Nonlinear Systems with Delay
\newblock Mathematical Problems in Engineering Vol. 2014, Article ID 301235

\bibitem[1000]{key1000}R Core Team (2015). 
\newblock R: A language and environment for statistical computing. R Foundation for Statistical Computing, Vienna, Austria.
\newblock URL https://www.R-project.org/.

\end{thebibliography}

\end{document}
