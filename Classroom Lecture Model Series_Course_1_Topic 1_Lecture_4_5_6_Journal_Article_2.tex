%-------------------------------------------------------------------------%
%--------------------Classroom Lecture Model Series-----------------------%
%-------------------------------------------------------------------------%

%--------------------Work In Progress for the Classroom-------------------%

\begin{document}
\twocolumn
\scriptsize
\begin{frontmatter}
		\title{}
		\author{\corref{cor1}\fnref{fn1}}
		\cortext[cor1]{Corresponding author}
		\address{The Mathematical Learning Space}
		\ead{http://mathlearningspace.weebly.com}	
\end{frontmatter}	

Introduction:
\begin{enumerate}
\item Objective 1:
\item Objective 2:
\item Objective 3:
\end{enumerate}
Conclusion:

Keywords: Fractional Calculus, Stochastic Calculus, Fractional Differential Equations, Stochastic Diffferential Equations, Mittag-Leffler functions, Laplace Transforms

\section{Introduction}

\begin{enumerate}
	\item A \cite{key1}
\end{enumerate}

\subsection{Review Notes}

\begin{table}[H]\centering
	\begin{tabular}{p{1cm}p{4cm}p{3cm}}
		Article ID & Summary & Comments\\
		\hline
		\hline
	\end{tabular}
\end{table}

\centering	
\begin{table}[H]\tiny
	\caption{Gene/Protein/Enzymes Description and References}	
	\begin{tabular}{r|p{4cm}|l}
		\hline	
		Gene/Protein & Description & Reference \\
		\hline 
		\hline 
	\end{tabular}
\end{table}

\subsection{Plan of the Article}

\begin{enumerate}
\item List Fractional Derivatives
\item List Drift and Diffusion Equations
\item List Algorithms
\item Moment Analysis
\item Equilibrium and Stability Analysis
\item Model Comparision
\item Additional Topics for Discussion
\end{enumerate}

\section{Mathematical Methods}

\begin{table}[H]\centering
	\begin{tabular}{p{1cm}p{4cm}p{3cm}}
		Article ID & Summary & Comments\\
		\hline
		\hline
	\end{tabular}
\end{table}

\subsection{Definitions}

\centering
\begin{table}[H]\footnotesize
	\caption{}
	\begin{tabular}{rp{1cm}p{2cm}p{3cm}p{1cm}}
		\hline
		ID & A & B & C & Reference \\
		\hline
		\hline
	\end{tabular}
\end{table}
\raggedright

\subsection{Theorems}

\subsection{Exponential and Geometric Family Distributions}

\begin{table}[H]\centering
	\begin{tabular}{p{1cm}p{4cm}p{3cm}}
		Equation ID & Description & Comments\\
		\hline
		\hline
	\end{tabular}
\end{table}

\begin{center}
	\begin{tikzpicture}
	[->,>=stealth',shorten >=2pt,node distance=3cm,
	thick,main node/.style={circle,draw,scale=0.25,transform canvas={scale=0.75},font=\sffamily\Small\bfseries},
	blacknode/.style={shape=circle, draw=black, line width=2},
	bluenode/.style={shape=circle, draw=blue, line width=2},
	greennode/.style={shape=circle, draw=green, line width=2},
	rednode/.style={shape=circle, draw=red, line width=2}
	]
%-------Legend ------------------------------------------------------------	
	\matrix [draw,below left] at (current bounding box.south) {
		\node [state,label=right:State] {}; A. \\
		\node [shapeSquare,label=right:Square- B.] {}; \\
		\node [shapeEllipse,label=right:Ellipse-C.] {}; \\
		\node [shapeTriangle,label=right:Triangle-D.] {}; \\
		\node [shapeHexagon,label=right:Hexagon-E.] {}; \\
	};
	\end{tikzpicture}
\end{center}

\subsection{Fractional Derivative Operators}

\begin{table}[H]\centering
	\begin{tabular}{p{1cm}p{4cm}p{3cm}}
		Operator ID & Description & Comments\\
		\hline
		\hline
	\end{tabular}
\end{table}


\subsection{Equation Systems}

\begin{table}[H]\centering
	\begin{tabular}{p{1cm}p{4cm}p{3cm}}
		Article ID & Summary & Comments\\
		\hline
		\hline
	\end{tabular}
\end{table}

\subsubsection{Drift}

\centering
\begin{table}[H]\footnotesize
	\caption{}
	\begin{tabular}{rp{1cm}p{2cm}p{3cm}p{1cm}}
		\hline
		ID & A & B & C & Reference \\
		\hline
		\hline
	\end{tabular}
\end{table}
\raggedright

\subsection{Diffusion}

\centering
\begin{table}[H]\footnotesize
	\caption{}
	\begin{tabular}{rp{1cm}p{2cm}p{3cm}p{1cm}}
		\hline
		ID & A & B & C & Reference \\
		\hline
		\hline
	\end{tabular}
\end{table}
\raggedright

\subsection{Stochastic Error}


\section{Algorithms}
%----------------------------------------------------------------------------
%						Algorithm 1
%----------------------------------------------------------------------------		
\begin{algorithm}[H]
\begin{algorithmic}[1]
\end{algorithmic}
\caption{Fractional Differential Equation Computation}
	\label{Algorithm_1}
\end{algorithm}

%----------------------------------------------------------------------------
%						Algorithm 2
%----------------------------------------------------------------------------		
\begin{algorithm}[H]
\begin{algorithmic}[1]
\end{algorithmic}
\caption{Stochastic Differential Equation Computation}
	\label{Algorithm_2}
\end{algorithm}

\section{Results}

\subsection{Tables}

\centering
\begin{table}[H]\footnotesize
	\caption{}
	\begin{tabular}{rp{1cm}p{2cm}p{3cm}p{1cm}}
		\hline
		ID & A & B & C & Reference \\
		\hline
		\hline
	\end{tabular}
\end{table}
\raggedright

\centering
\begin{table}[H]\footnotesize
	\caption{}
	\begin{tabular}{rp{1cm}p{2cm}p{3cm}p{1cm}}
		\hline
		ID & A & B & C & Reference \\
		\hline
		\hline
	\end{tabular}
\end{table}
\raggedright

\centering
\begin{table}[H]\footnotesize
	\caption{}
	\begin{tabular}{rp{1cm}p{2cm}p{3cm}p{1cm}}
		\hline
		ID & A & B & C & Reference \\
		\hline
		\hline
	\end{tabular}
\end{table}
\raggedright

\subsection{Figures}

\subsection{Moment Analysis}
\subsection{Equilibrium and Stability Analysis}
\subsection{Model Comparision}

\section{Additional Topics for the Classroom}

\begin{enumerate}
\item 
\end{enumerate}


\bibliographystyle{plain}
\begin{thebibliography}{00}

\bibitem[100]{key100} Mistry, L. A. Khan, and D.L. Suthar (2016)
\newblock Review on Fractional Differential Equations and their Applications
\newblock Proceedings of International Conference on Emerging Technologies in Engineering, Biomedical, Management and Science 

\bibitem[101]{key101} Odibat Z.M. and S. Momani
\newblock An Algorithm for the Numerical Solution of Differential Equations of Fractional Order
\newblock Journal Applied Mathematical and Informatics 26:(2008) 1:2pp.15-27

\bibitem[102]{key102} Roberto Garrappa
\newblock Numerical Solution of Fractional Differential Equations: A Survey and a Software Tutorial
\newblock Mathematics 2018,6, 16; doi:10.3390/math6020016 www.mdpi.com/journal/mathematics

\bibitem[103]{key103} Nan Xia, Tashpolat Tiyip, Ardak Kelimu, et al., 
\newblock Influence of Fractional Differential on Correlation Coefficient between EC1:5 and Reflectance Spectra of Saline Soil,
\newblock Journal of Spectroscopy, vol. 2017, Article ID 1236329, 11 pages, 2017. https://doi.org/10.1155/2017/1236329.

\bibitem[104]{key104}K. Diethelm, N.J. Ford, A.D. Freed, 
\newblock A predictor–corrector approach for the numerical solution of fractional differential equations, 
\newblock Nonlinear Dyn. 29 (2002) 3–22.
 
\bibitem[105]{key105} K. Diethelm, 
\newblock Detailed error analysis for a fractional Adams method
\newblock Numerical Algorithms, in press

\bibitem[106]{key106} Hilfer,  R.
\newblock Threefold  Introduccion  to  fractional  derivatives.  
\newblock Anomalous transport:  Foundation and applications. (2008)

\bibitem[107]{key107} Ruben A. Cerutti
\newblock The k-Fractional Logistic Equation with k-Caputo Derivative
\newblock Pure Mathematical Sciences, Vol.  4, 2015, no.  1, 9 - 15

\bibitem[201]{key201}B. Ghayebi and S. M. Hosseini
\newblock A Simplified Milstein Scheme for SPDEs with Multiplicative Noise
\newblock Abstract and Applied Analysis Vol. 2014, Article ID 140849

\bibitem[301]{key301}Yu Wang and Tianzeng Li
\newblock Stability Analysis of Fractional-Order Nonlinear Systems with Delay
\newblock Mathematical Problems in Engineering Vol. 2014, Article ID 301235

\bibitem[1000]{key1000}R Core Team (2015). 
\newblock R: A language and environment for statistical computing. R Foundation for Statistical Computing, Vienna, Austria.
\newblock URL https://www.R-project.org/.

\end{thebibliography}

\end{document}
