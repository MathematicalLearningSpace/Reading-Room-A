\documentclass[preprint, 8pt]{elsarticle}
\usepackage{xcolor}
\usepackage{chemfig}
\usepackage{tikz}
\usepackage{graphicx}
\usepackage{amsmath, amssymb}
\setlength{\parindent}{0pt}
\usepackage{pgfplots}
\pgfplotsset{compat=1.3,}
\pgfplotscreateplotcyclelist{line styles}{ 
	black,solid\\
	blue,dashed\\
	red,dotted\\
	orange,dashdotted\\
}
\newcommand*\GnuplotDefs{
	set samples 50;
	cdfn(x,mu,sd) = 0.5 * ( 1 + erf( (x-mu)/sd/sqrt(2)) );
	pdfn(x,mu,sd) = 1/(sd*sqrt(2*pi)) * exp( -(x-mu)^2 / (2*sd^2) );
	tpdfn(x,mu,sd,a,b) = pdfn(x,mu,sd) / ( cdfn(b,mu,sd) - cdfn(a,mu,sd) );
}
\usepackage[a4paper,bindingoffset=0.2in,
            left=0.5in,right=0.5in,top=1in,bottom=1in,
            footskip=.25in]{geometry}
%\usepackage{geometry}
\usepackage{mathtools}
\usepackage{tkz-berge}
 \usetikzlibrary{calc} 
\usetikzlibrary{automata}
\usetikzlibrary{arrows}
\usetikzlibrary{positioning,shapes,shadows,arrows}
\usetikzlibrary{shapes.geometric}
\usetikzlibrary{calendar,shadings}
\renewcommand*{\familydefault}{\sfdefault}
\colorlet{winter}{blue}
\colorlet{spring}{green!60!black}
\colorlet{summer}{orange}
\colorlet{fall}{red}
\newcount\mycount
\newcommand\shapeLarge{50mm}
\newcommand\shapeMedium{25mm}
\newcommand\shapeSmall{5mm}
\newcommand*{\xMin}{0}
\newcommand*{\xMax}{6}
\newcommand*{\yMin}{0}
\newcommand*{\yMax}{6}
\newcommand*{\zMax}{6}
\newcommand*{\zMin}{0}
\definecolor{colorwaveA}{RGB}{98,145,224}
\definecolor{colorwaveB}{RGB}{250,250,50}
\definecolor{colorwaveC}{RGB}{25,125,25}
\definecolor{colorwaveD}{RGB}{100,100,100}
\definecolor{colorwaveE}{RGB}{80,100,1}
\definecolor{colorwaveF}{RGB}{60,1,1}
\definecolor{colorwaveG}{RGB}{25,1,100}
\definecolor{colorwaveH}{RGB}{1,90,1}
\definecolor{colorwaveI}{RGB}{1,100,1}
\definecolor{colorwaveJ}{RGB}{1,1,1}
\tikzset{
	shapeTriangle/.style={draw,shape=regular polygon,fill=colorwaveA,circular drop shadow,regular polygon sides=3,minimum size=\shapeSmall,inner sep=0pt,outer sep=0pt},
	shapeTriangle3/.style={shapeTriangle,fill=colorwaveD,circular drop shadow,shape border rotate=45},
	shapeTriangle4/.style={shapeTriangle,fill=colorwaveA,circular drop shadow,shape border rotate=90},
	shapeTriangle5/.style={shapeTriangle,fill=colorwaveB,shape border rotate=135},
	shapeTriangle6/.style={shapeTriangle,fill=colorwaveC,shape border rotate=180},
	shapeTriangle7/.style={shapeTriangle,fill=colorwaveE,shape border rotate=225},
	shapeTriangle8/.style={shapeTriangle,fill=colorwaveF,shape border rotate=270},
	shapeTriangle9/.style={shapeTriangle,fill=colorwaveG,shape border rotate=315},
}
\tikzset{shapeUgaritic/.style={draw,shape=regular polygon,fill=colorwaveD,circular drop shadow,regular polygon sides=3,minimum size=\shapeSmall,inner sep=0pt,outer sep=0pt},
}

\tikzset{
	shapeSquare/.style={draw,shape=regular polygon,fill=colorwaveC,circular drop shadow,regular polygon sides=4,minimum size=\shapeSmall,inner sep=0pt,outer sep=0pt},
	shapeSquare2/.style={shapeSquare,shape border rotate=45},
}

\tikzset{
	shapeHexagon/.style={draw,shape=regular polygon,fill=colorwaveA,circular drop shadow,regular polygon sides=6,minimum size=\shapeSmall,inner sep=0pt,outer sep=0pt},
	shapeHexagon2/.style={shapeHexagon,shape border rotate=90},
}

\tikzset{
	shapeOctagon/.style={draw,shape=regular polygon,fill=colorwaveB,circular drop shadow,regular polygon sides=8,minimum size=\shapeSmall,inner sep=0pt,outer sep=0pt},
	shapeOctagon2/.style={shapeHexagon,shape border rotate=45},
}
\tikzset{
	shapeEllipse/.style={draw,shape=ellipse,minimum size=\shapeSmall,inner sep=0pt,outer sep=0pt},
	shapeEllipse2/.style={shapeEllipse,shape border rotate=90},
}

\tikzset{
	closedFigure/.style={draw=\draw[->,rounded corners=0.2cm,shorten >=2pt]
		(1.5,0.5)-- ++(0,-1)-- ++(1,0)-- ++(0,2)-- ++(-1,0)-- ++(0,2)-- ++(1,0)--
		++(0,1)-- ++(-1,0)-- ++(0,-1)-- ++(-2,0)-- ++(0,3)-- ++(2,0)-- ++(0,-1)--
		++(1,0)-- ++(0,1)-- ++(1,0)-- ++(0,-1)-- ++(1,0)-- ++(0,-3)-- ++(-2,0)--
		++(1,0)-- ++(0,-3)-- ++(1,0)-- ++(0,-1)-- ++(-6,0)-- ++(0,3)-- ++(2,0)--
		++(0,-1)-- ++(1,0)}
}
\tikzstyle{start}=[circle, draw=none,,minimum size=\shapeMedium, fill=blue, circular drop shadow,text centered, anchor=north, text=white]
\tikzstyle{finish}=[circle, draw=none,,minimum size=\shapeMedium, fill=blue,circular drop shadow,text centered, anchor=north, text=white]
\tikzstyle{finish}=[rectangle, draw=none, ,minimum size=\shapeMedium,fill=blue,circular drop shadow,text centered, anchor=north, text=white]
\usepackage[noadjust]{cite}
\usepackage{algpseudocode}
\usepackage{listings}
\usepackage{algorithm}
\usepackage{color}
\usepackage{parskip}
\usepackage{amsfonts}
\usepackage{amsthm}
\usepackage{tikz}
\usepackage{tkz-berge}
\usepackage{caption}
\usepackage{hyperref}
\usepackage{amsrefs}
\usepackage{mathtools, amssymb}
\usepackage{graphicx}
\usepackage{subcaption}
\usepackage{tabularx,ragged2e}
\usepackage[framemethod=tikz]{mdframed}
\newcommand{\N}{\mathbb N}
\newcommand{\Q}{\mathbb Q}
\theoremstyle{definition}
\newtheorem{definition}{Definition}[section] 
\newtheorem{theorem}{Theorem}[section]
\newtheorem{example}{Example}[section]
\renewcommand{\qedsymbol}{$\blacksquare$}
\newtheorem{corollary}{Corollary}[theorem]
\newtheorem{lemma}[theorem]{Lemma}
\renewcommand{\rmdefault}{ptm} 
\graphicspath{{Figures/}}
\twocolumn

\begin{document}
	
\section{Introduction}

\section{Material and Methods}

\section{Networks}

A sample of one arrangment of four separate networks for the dynamical description of (A) DNA damage repair (B) Cell Cycle Arrest  (C) Oncogene dynamics with PI3K /PTEN/Akt signaling pathway, and (D) Apoptosome Formation for Apoptosis. Figure 1 shows the relationships. 

\begin{figure}[H]
\begin{tikzpicture}[auto, thick, node distance=1.25cm, >=triangle 45,scale=.15]
\tiny
\draw
	%--------------------------------------------------Drawing of Network A---------------------------------------------%
	node at (0,0)[right=-3mm]{\small Start}
	node [input, name=input1] {} 
	node [block, right of=input1] (UV) {UV}
	node [block, below of=UV] (ATR) {ATR}
        node [block, right of=UV] (DNA) {DNA Damage}
	node[block, left of=ATR](RPA) {RPA}
        node [block, below of=DNA] (ATRp) {ATRp};
  	
	\draw[->] (input1) --(UV);   
	\draw[->] (UV) --(DNA); 
	\draw[->] (ATR) --(ATRp); 
	\draw[->,dashed] (RPA) --(ATR); 
	\draw[->,dashed] (DNA)--(12,-8);
	%\draw [color=gray,thick](0,0) rectangle (1,1);
	\node at (1,3) [above=1mm, right=0mm] {\textsc{Network A}};	
	%------------------------------------------------Drawing Network B------------------------------------------------%
	\draw
	node at (30,1) [block] (p21) {p21}
    	node [block, right of=p21] (DDb2) {DDb2}
    	node [block, below of=DDb2] (Rbp) {Rbp}
    	node [block, below of=Rbp] (Cyce) {Cyce}
	node [block, right of=Rbp] (RB) {RB}
	node [block, below of=RB] (EF21) {EF21}
	node at (30,-5) [block](CC) {Cell Cycle}
	;
	\draw[->] (p21) --(CC);  
	 \draw[->] (RB) --(EF21);
	 \draw[->] (RB) --(Rbp);
	 \draw[->] (Rbp) --(RB);
	 \draw[->] (Rbp) --(EF21);
	 \draw[->] (DDb2) --(p21);
	 \draw[->] (EF21) --(Cyce);
	 \draw[->] (p21) --(Cyce);
	%\draw [color=gray,thick](5.5,-3) rectangle (12,2);
	\node at (35,4) [below=5mm, right=0mm] {\textsc{Network B}};
	%----------------------------------------------- Draw Network C-------------------------------------------------%
	\draw
	node at  (0.5,-20) [sum, name=p53] {P53}
	node [sum, right of=p53] (mdm2n) {Mdm2n}
	node [block, right of=mdm2n] (mdm2cp) {Mdm2cp}
	node [block, right of=mdm2cp] (aktp) {Aktp}
	node [block, below of=p53] (p53p) {p53p}
    	node [block, right of=p53p] (pten) {Pten}
	node [block, below of=mdm2cp] (mdm2c) {Mdm2c}
	node [block, right of=mdm2c] (akt) {Akt}
	node [block, below of=pten] (pip2) {PIP2}
	node [block, below of=mdm2c] (pip3) {PIP3}
	;
	\draw[->] (p53) --(p53p);  
	\draw[->,dashed] (p53p) --(p53); 
	\draw[->] (ATRp) --(mdm2n);  
	\draw[->,dashed] (p21) --(p53);  
	 \draw[->] (pip2) --(pip3);
 	\draw[->] (pip3) --(pip2);
 	\draw[->] (akt) --(aktp);
	 \draw[->] (aktp) --(akt);
	\draw[->] (p53p) --(pten);  
	\draw[->] (mdm2cp) --(mdm2c);  
	\draw[->] (mdm2c) --(mdm2cp);
	\draw[->] (pip3) --(akt);
	\draw[->,dashed] (p53p) --(mdm2c);
	\draw[->] (mdm2n) --(p53);
	%\draw [color=gray,thick](-0.5,-9) rectangle (7,-4);
	\node at (-0.5,-40) [below=5mm, right=0mm] {\textsc{Network C}};
	%----------------------------------------------Draw Network D-----------------------------------------------%
	\draw
	node  at (30, -20) [block] (bax) {BAX}
	node [block,below of=bax ](cytoc) {CytoC}
	node [block, right of=bax] (apops) {Apops}
	node [block, below of=apops] (casp9) {CASP9}
	node [block, right of=apops] (procasp3) {Procasp3}
	node [block, below of=cytoc] (apaf1) {Apaf-1}
	node [block, below  of=casp9] (procasp9) {Procasp9}
	node [block, below of=procasp3] (casp3) {casp3}
	node [block, below of=apaf1] (APTX) {APTX}
	node [block, below of=casp3] (PARP1) {PARP1}
	node [block, below of=PARP1] (PARP3) {PARP3}
	node[sum, below of=PARP3] (Apoptosis){Apoptosis}
	;
	\draw[->,dashed] (p53) to [controls=+(30:12) and +(30:12)] (bax);
	\draw[->,dashed] (bax) --(cytoc);
  	\draw[->,dashed] (cytoc) --(apops);
	\draw[->,dashed] (apaf1) --(apops);
	\draw[->,dashed] (apops) --(casp9);
	\draw[->,dashed] (casp9) --(procasp9);
	\draw[->,dashed] (procasp9) --(casp9);
	\draw[->,dashed] (casp3) --(procasp3);
	\draw[->,dashed] (procasp3) --(casp3);
	\draw[->,dashed] (casp3) --(casp9);
	\draw[->] (PARP1) --(APTX); 
	\draw[->] (PARP1) --(PARP3); 
	\draw[->] (casp3) --(PARP1); 
	\draw[->] (bax) --(casp3); 
	\draw[->] (PARP3) --(Apoptosis);
	%\draw [color=gray,thick](17.5,-9) rectangle (12.5,-4);
	\node at (35,-60) [below=5mm, right=0mm] {\textsc{Network D}};
\end{tikzpicture}
\end{figure}

\section{Definitions}

\section{Theorems}

\section{Algorithms}

\section{Equation Systems}

\section{Parameter Matrix}


\section{Results}


\section{Discussion Topics for the Classroom}



\section{Acknowledgements}


\section{Conflict of Interest}

\section{References}

\bibliographystyle{plain}
\begin{thebibliography}{00}
\footnotesize

\subsection{R References}

\bibitem[1000]{key1000}R Core Team (2015). 
\newblock R: A language and environment for statistical computing. R Foundation for Statistical Computing, Vienna, Austria.
\newblock URL https://www.R-project.org/.

\end{thebibliography}

\end{document}
