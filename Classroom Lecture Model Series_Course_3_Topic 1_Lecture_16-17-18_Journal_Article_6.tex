%-------------------------------------------------------------------------%
%--------------------Classroom Lecture Model Series-----------------------%
%-------------------------------------------------------------------------%

%--------------------Work in Progress for the Classroom-------------------%

\begin{document}
\twocolumn
\scriptsize
\begin{frontmatter}
		\title{}
		\author{\corref{cor1}\fnref{fn1}}
		\cortext[cor1]{Corresponding author}
		\address{The Mathematical Learning Space}
		\ead{http://mathlearningspace.weebly.com}	
\end{frontmatter}	

Introduction:
\begin{enumerate}
\item Objective 1:
\item Objective 2:
\item Objective 3:
\end{enumerate}
Conclusion:

Keywords: Ribosome, bioengineering, molecular dynamics, force fields, normal mode analysis, modeling, simulation 
Vocabulary Words:

\section{Introduction}

\subsection{Review Topic: Bioengineering}

\subsection{Plan of the Article}

\section{Biology: Ribosome}
\vspace{6pt}

\subsection{Ribosome Review 40S and 60s Components}

\centering	
\begin{table}[H]\tiny
	\caption{}	
	\begin{tabular}{rp{1cm}|p{4cm}|l}
		\hline	
		PubmedID & Topic & Description & Citation \\
		\hline 
		\hline 
	\end{tabular}
\end{table}

\subsection{ER Sheet with a Ribosome Matrix - helicoidal Ramp Models}

\centering	
\begin{table}[H]\tiny
	\caption{}	
	\begin{tabular}{rp{1cm}|p{4cm}|l}
		\hline	
		PubmedID & Topic & Description & Citation \\
		\hline 
		\hline 
	\end{tabular}
\end{table}

\subsection{Examples of Ribosome Component Models}

\begin{figure}[H]
	\centering
\begin{minipage}[b]{0.3\linewidth}
%\includegraphics[scale=0.3]{Top.png}
	\caption{}
	\label{fig:FigureA}
\end{minipage}\hfill
\begin{minipage}[b]{0.3\linewidth}
%\includegraphics[scale=0.3]{Middle.png}
	\caption{}
	\label{fig:FigureB}
\end{minipage}\hfill
\begin{minipage}[b]{0.3\linewidth}
%\includegraphics[scale=0.3]{Bottom.png}
	\caption{}
	\label{fig:FigureC}
\end{minipage}\hfill
	\caption{}
	\label{fig:Figure_1}
\end{figure}


\section{Review of Mathematical Topics A}


\section{Ribosome Design}

\begin{table}[H]
\centering
\begin{tabular}{rll}
Equation & Molecular Species & Reaction Description\\ 
\hline
\hline
\end{tabular}
\caption{Molecular Species Description}
\end{table}


\subsection{Multidimensional Structure}

\subsection{Normal Mode Analysis}

\subsection{Functional Form Specification of Force Fields (FF)}

\begin{table}[H]
\caption{Table 1}
\tiny
\begin{tabular}{p{1cm}p{5cm}p{2cm}}
\hline
FF Category  & Specification & Comment  \\
\hline
\hline
\end{tabular}
\end{table}

\section{Ribosome Model}


\section{Results}

\subsection{Riboosome Simulation}


\subsection{Tables}

\subsection{Figures}

\begin{figure}[H]
	\centering
	\begin{minipage}[b]{0.5\linewidth}
	%\includegraphics[scale=0.25]{Example_1_Figure_1.png}
	\end{minipage}\hfill
	\begin{minipage}[b]{0.5\linewidth}
	%\includegraphics[scale=0.25]{Example_1_Figure_2.png}
	\end{minipage}\hfill	
	\begin{minipage}[b]{0.5\linewidth}
	%\includegraphics[scale=0.25]{Example_1_Figure_3.png}
	\end{minipage}\hfill
	\begin{minipage}[b]{0.5\linewidth}
	%\includegraphics[scale=0.25]{Example_1_Figure_4.png}
	\end{minipage}\hfill
	\caption{1, 2, 3 and 4}
	\label{fig:Figure1}
\end{figure} 


\section{Conclusions}


\section{R Application Programming Interfaces (APIs)}




\bibliographystyle{plain}
\begin{thebibliography}{00}

\bibitem[1]{key1} Bajzer Z, Vuk-Pavlovic S (2005)  
\newblock Modeling positive regulatory feedbacks in cell cell interactions. 
\newblock Biosystems 80: 1–10. 

\bibitem[100]{key100}Wikipedia contributors. 
\newblock Normal mode.
\newblock Wikipedia, The Free Encyclopedia.

\bibitem[101]{key101}Wikipedia contributors. 
\newblock Force field (chemistry).
\newblock Wikipedia, The Free Encyclopedia.

\bibitem[200]{key200} Jitao David Zhang (2015). 
\newblock KEGGgraph: Application Examples
\newblock R package version 1.28.0.

\bibitem[201]{key201}Csardi G, Nepusz T (2006).
\newblock The igraph software package for complex network research,
\newblock  InterJournal, Complex Systems 1695. 2006. http://igraph.org

\bibitem[202]{key202}Soetaert K. (2009).  
\newblock rootSolve: Nonlinear root finding, equilibrium and steady-state analysis of ordinary differential equations.  
\newblock R-package version 1.6.

\bibitem[203]{key203}Karline Soetaert, Thomas Petzoldt, R. Woodrow Setzer (2010). 
\newblock Solving Differential Equations in R: Package deSolve. 
\newblock Journal of Statistical Software, 33(9), 1--25. URL http://www.jstatsoft.org/v33/i09/ DOI 10.18637/jss.v033.i09

\bibitem[300]{key300}C. Stark, et aI., (2006)
\newblock "BioGRIO: a general repository for interaction datasets, " 
\newblock Nucleic Acids Res., vol. 34, iss. suppl I, pp. D535-539, January 2006. Available: http://thebiogrid.org. 

\bibitem[1000]{key1000}R Core Team (2015). 
\newblock R: A language and environment for statistical computing. R Foundation for Statistical Computing, Vienna, Austria.
\newblock URL https://www.R-project.org/.

\end{thebibliography}
\end{document}
