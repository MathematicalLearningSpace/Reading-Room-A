%-------------------------------------------------------------------------%
%--------------------Classroom Lecture Model Series-----------------------%
%-------------------------------------------------------------------------%

\begin{document}
\twocolumn
\scriptsize
\begin{frontmatter}
		\title{DRAFT Sample: A Molecular Machine Learning Algorithm of Multi-Protein Complexes in The Digestion System}
		\author{\corref{cor1}\fnref{fn1}}
		\cortext[cor1]{Corresponding author}
		\address{The Mathematical Learning Space}
		\ead{http://mathlearningspace.weebly.com}	
\end{frontmatter}	

Introduction:
\begin{enumerate}
\item Objective 1:
\item Objective 2:
\item Objective 3:
\end{enumerate}
Conclusion:

Keywords: Molecular Machines
Vocabulary Words:

\section{Introduction}

\centering
\begin{table}[H]\footnotesize
	\caption{}
	\begin{tabular}{rp{1cm}p{2cm}p{3cm}p{1cm}}
		\hline
		ID & A & B & C & Reference \\
		\hline
		\hline
	\end{tabular}
\end{table}
\raggedright


\subsection{Plan of the Article}

\begin{enumerate}
\end{enumerate}

\section{Biology Review Topics: Molecular Machines}

\centering	
\begin{table}[H]\tiny
	\caption{}	
	\begin{tabular}{rp{1cm}|p{4cm}|l}
		\hline	
		PubmedID & Topic & Description & Citation \\
		\hline 
		\hline 
	\end{tabular}
\end{table}

\section{Mathematical Topics}

\centering
\begin{table}[H]\footnotesize
	\caption{}
	\begin{tabular}{rp{1cm}p{2cm}p{3cm}p{1cm}}
		\hline
		ID & A & B & C & Reference \\
		\hline
		\hline
	\end{tabular}
\end{table}
\raggedright


\section{Results}

\subsection{Tables}

\centering	
\begin{table}[H]\tiny
	\caption{}	
	\begin{tabular}{rp{1cm}|p{4cm}|l}
		\hline	
		PubmedID & Topic & Description & Citation \\
		\hline 
		\hline 
	\end{tabular}
\end{table}

\centering
\begin{table}[H]\footnotesize
	\caption{}
	\begin{tabular}{rp{1cm}p{2cm}p{3cm}p{1cm}}
		\hline
		ID & A & B & C & Reference \\
		\hline
		\hline
	\end{tabular}
\end{table}
\raggedright

\centering
\begin{table}[H]\footnotesize
	\caption{}
	\begin{tabular}{rp{1cm}p{2cm}p{3cm}p{1cm}}
		\hline
		ID & A & B & C & Reference \\
		\hline
		\hline
	\end{tabular}
\end{table}
\raggedright


\subsection{Figures}

\begin{figure}[H]
	\centering
	\begin{minipage}[b]{0.5\linewidth}
	%\includegraphics[scale=0.25]{Example_1_Figure_1.png}
	\end{minipage}\hfill
	\begin{minipage}[b]{0.5\linewidth}
	%\includegraphics[scale=0.25]{Example_1_Figure_2.png}
	\end{minipage}\hfill	
	\begin{minipage}[b]{0.5\linewidth}
	%\includegraphics[scale=0.25]{Example_1_Figure_3.png}
	\end{minipage}\hfill
	\begin{minipage}[b]{0.5\linewidth}
	%\includegraphics[scale=0.25]{Example_1_Figure_4.png}
	\end{minipage}\hfill
	\caption{1, 2, 3 and 4}
	\label{fig:Figure1}
\end{figure} 


\section{Conclusions}

\centering
\begin{table}[H]\footnotesize
	\caption{}
	\begin{tabular}{rp{1cm}p{2cm}p{3cm}p{1cm}}
		\hline
		ID & A & B & C & Reference \\
		\hline
		\hline
	\end{tabular}
\end{table}
\raggedright


\section{Topics From the Classroom}

\centering	
\begin{table}[H]\tiny
	\caption{}	
	\begin{tabular}{r|p{4cm}|l}
		\hline	
		Topic & Description & Comment\\
		\hline 
		\hline 
	\end{tabular}
\end{table}


\section{R Application Programming Interfaces (APIs)}




\bibliographystyle{plain}
\begin{thebibliography}{00}

\bibitem[1]{key1}Wikipedia contributors. 
\newblock Molecular machine. 
\newblock Wikipedia, The Free Encyclopedia. Wikipedia The Free Encyclopedia

\bibitem[2]{key2}Cao D, Xiao N, Xu Q, Chen AF (2015). 
\newblock Rcpi: R/Bioconductor package to generate various descriptors of proteins, compounds and their interactions. 
\newblock Bioinformatics, 31(2), 279–281. doi: 10.1093/bioinformatics/btu624. 

\bibtem[3]{key3}Sunghwan Kim, Paul A. Thiessen, Evan E. Bolton, Jie Chen, Gang Fu, Asta Gindulyte, Lianyi Han, Jane He, Siqian He, Benjamin A. Shoemaker, Jiyao Wang, Bo Yu, Jian Zhang, Stephen H. Bryant; 
\newblock PubChem Substance and Compound databases, 
\newblock Nucleic Acids Research, Volume 44, Issue D1, 4 January 2016, Pages D1202–D1213, https://doi.org/10.1093/nar/gkv951

\bibitem[1000]{key1000}R Core Team (2015). 
\newblock R: A language and environment for statistical computing. R Foundation for Statistical Computing, Vienna, Austria.
\newblock URL https://www.R-project.org/.

\end{thebibliography}
\end{document}
