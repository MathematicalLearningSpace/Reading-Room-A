%-------------------------------------------------------------------------%
%--------------------Classroom Lecture Model Series-----------------------%
%-------------------------------------------------------------------------%

\begin{document}
\twocolumn
\scriptsize
\begin{frontmatter}
		\title{}
		\author{\corref{cor1}\fnref{fn1}}
		\cortext[cor1]{Corresponding author}
		\address{The Mathematical Learning Space}
		\ead{http://mathlearningspace.weebly.com}	
\end{frontmatter}	

Introduction:
\begin{enumerate}
\item Objective 1:
\item Objective 2:
\item Objective 3:
\end{enumerate}
Conclusion:

Keywords: Complex macromolecular machines, DNA Repair, Nanotechnology, Supramolecular chemistry, Molecular machines, Nanomachines
Vocabulary Words:

\section{Introduction}

\subsection{Plan of the Article}


\section{Conclusions}


\section{R Application Programming Interfaces (APIs)}




\bibliographystyle{plain}
\begin{thebibliography}{00}

\bibitem[1]{key1}Wikipedia contributors. 
\newblock Molecular machine. 
\newblock Wikipedia, The Free Encyclopedia. Wikipedia The Free Encyclopedia

\bibitem[2]{key2}Cao D, Xiao N, Xu Q, Chen AF (2015). 
\newblock Rcpi: R/Bioconductor package to generate various descriptors of proteins, compounds and their interactions. 
\newblock Bioinformatics, 31(2), 279–281. doi: 10.1093/bioinformatics/btu624. 

\bibtem[3]{key3}Sunghwan Kim, Paul A. Thiessen, Evan E. Bolton, Jie Chen, Gang Fu, Asta Gindulyte, Lianyi Han, Jane He, Siqian He, Benjamin A. Shoemaker, Jiyao Wang, Bo Yu, Jian Zhang, Stephen H. Bryant; 
\newblock PubChem Substance and Compound databases, 
\newblock Nucleic Acids Research, Volume 44, Issue D1, 4 January 2016, Pages D1202–D1213, https://doi.org/10.1093/nar/gkv951

\end{thebibliography}
\end{document}
