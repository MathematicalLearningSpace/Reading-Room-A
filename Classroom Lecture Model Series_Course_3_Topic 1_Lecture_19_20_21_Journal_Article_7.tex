%-------------------------------------------------------------------------%
%--------------------Classroom Lecture Model Series-----------------------%
%-------------------------------------------------------------------------%

\begin{document}
\twocolumn
\scriptsize
\begin{frontmatter}
		\title{DRAFT Sample: A Mathematical Model of Stress Induced Upregulation in Protein Reponses}
		\author{\corref{cor1}\fnref{fn1}}
		\cortext[cor1]{Corresponding author}
		\address{The Mathematical Learning Space}
		\ead{http://mathlearningspace.weebly.com}	
\end{frontmatter}	

Introduction:
\begin{enumerate}
\item Objective 1:
\item Objective 2:
\item Objective 3:
\end{enumerate}
Conclusion:

Keywords: Chaperonin Models, Macro-Molecular Crowding, Heat Shock Proteins, Heptameric rings, Computational Geometry
Vocabulary Words:

\section{Introduction}

\centering
\begin{table}[H]\footnotesize
	\caption{}
	\begin{tabular}{rp{1cm}p{2cm}p{3cm}p{1cm}}
		\hline
		ID & A & B & C & Reference \\
		\hline
		\hline
	\end{tabular}
\end{table}
\raggedright


\subsection{Plan of the Article}

\begin{enumerate}
\end{enumerate}

\section{Mathematical Review: Topic Computational Geometry}

\centering	
\begin{table}[H]\tiny
	\caption{}	
	\begin{tabular}{rp{1cm}|p{4cm}|l}
		\hline	
		PubmedID & Topic & Description & Citation \\
		\hline 
		\hline 
	\end{tabular}
\end{table}

\section{Biology Review}

\subsection{Review Folding Time Scales}

\begin{enumerate}
\item 1 ns  -  side chain rotation
\item 10 ns  - Peptide bond dynamics
\item 100 ns - Hydrophobic collapse -alpha helix
\item 1 mu s  -Beta hairpins
\item 10 mu s- Topological reorganization
\end{enumerate}

\subsection{Heat Shock Proteins : HSP60}

\centering	
\begin{table}[H]\tiny
	\caption{}	
	\begin{tabular}{rp{1cm}|p{4cm}|l}
		\hline	
		PubmedID & Topic & Description & Citation \\
		\hline 
		\hline 
	\end{tabular}
\end{table}

\subsection{Equation System A}

\begin{align*} 
\tiny
\frac{d^{\alpha_1}X_1(t)}{dt^{\alpha_1}} = a_{11} *X_1(t - \tau_1) + \\
a_{12} *\frac{X_1(t)^{\beta_1}}{(1-X_1(t - \tau_1))^{\beta_1}} - a_{15}X_1(t) + \\
\epsilon_1(t) \\
\frac{d^{\alpha_1}X_2(t)}{dt^{\alpha_1}} = a_{21} *X_2(t - \tau_2) + \\
a_{22} *\frac{X_2(t)^{\beta_1}}{(1-X_2(t - \tau_2))^{\beta_1}} - a_{25}X_2(t) + \\
\epsilon_2(t) \\ \\
\frac{d^{\alpha_1}X_3(t)}{dt^{\alpha_1}} = a_{31} *X_3(t - \tau_3) + \\
a_{32} *\frac{X_3(t)^{\beta_1}}{(1-X_3(t - \tau_3))^{\beta_1}} - a_{35}X_3(t) + \\
\epsilon_3(t) \\ \\
\frac{d^{\alpha_1}X_4(t)}{dt^{\alpha_1}} = a_{41} *X_4(t - \tau_4) + \\
a_{42} *\frac{X_4(t)^{\beta_1}}{(1-X_4(t - \tau_4))^{\beta_1}} - a_{45}X_4(t) + \\
\epsilon_4(t) \\ \\
\frac{d^{\alpha_1}X_5(t)}{dt^{\alpha_1}} = a_{51} *X_5(t - \tau_5) + \\
a_{52} *\frac{X_5(t)^{\beta_1}}{(1-X_5(t - \tau_5))^{\beta_1}} - a_{55}X_5(t) + \\
\epsilon_5(t) \\
\end{align*}


\subsection{Chaperonin Design Models: Topic Monomers}

\centering	
\begin{table}[H]\tiny
	\caption{}	
	\begin{tabular}{rp{1cm}|p{4cm}|l}
		\hline	
		PubmedID & Topic & Description & Citation \\
		\hline 
		\hline 
	\end{tabular}
\end{table}

\subsection{Chaperonin Design Models: Topic heptamers}

\centering	
\begin{table}[H]\tiny
	\caption{}	
	\begin{tabular}{rp{1cm}|p{4cm}|l}
		\hline	
		PubmedID & Topic & Description & Citation \\
		\hline 
		\hline 
	\end{tabular}
\end{table}

\subsection{Chaperonin Design Models: Topic tetradeceamers}

\centering	
\begin{table}[H]\tiny
	\caption{}	
	\begin{tabular}{rp{1cm}|p{4cm}|l}
		\hline	
		PubmedID & Topic & Description & Citation \\
		\hline 
		\hline 
	\end{tabular}
\end{table}

\subsection{Domain Models: Apical}

\centering	
\begin{table}[H]\tiny
	\caption{}	
	\begin{tabular}{rp{1cm}|p{4cm}|l}
		\hline	
		PubmedID & Topic & Description & Citation \\
		\hline 
		\hline 
	\end{tabular}
\end{table}

\subsection{Domain Models: Equatorial}

\centering	
\begin{table}[H]\tiny
	\caption{}	
	\begin{tabular}{rp{1cm}|p{4cm}|l}
		\hline	
		PubmedID & Topic & Description & Citation \\
		\hline 
		\hline 
	\end{tabular}
\end{table}

\subsection{Domain Models: Intermediate}

\centering	
\begin{table}[H]\tiny
	\caption{}	
	\begin{tabular}{rp{1cm}|p{4cm}|l}
		\hline	
		PubmedID & Topic & Description & Citation \\
		\hline 
		\hline 
	\end{tabular}
\end{table}

\subsection{Equation System A}

\begin{align*} 
\tiny
\frac{d^{\alpha_1}X_1(t)}{dt^{\alpha_1}} = a_{11} *X_1(t - \tau_1) + \\
a_{12} *\frac{X_1(t)^{\beta_1}}{(1-X_1(t - \tau_1))^{\beta_1}} - a_{15}X_1(t) + \\
\epsilon_1(t) \\
\frac{d^{\alpha_1}X_2(t)}{dt^{\alpha_1}} = a_{21} *X_2(t - \tau_2) + \\
a_{22} *\frac{X_2(t)^{\beta_1}}{(1-X_2(t - \tau_2))^{\beta_1}} - a_{25}X_2(t) + \\
\epsilon_2(t) \\ \\
\frac{d^{\alpha_1}X_3(t)}{dt^{\alpha_1}} = a_{31} *X_3(t - \tau_3) + \\
a_{32} *\frac{X_3(t)^{\beta_1}}{(1-X_3(t - \tau_3))^{\beta_1}} - a_{35}X_3(t) + \\
\epsilon_3(t) \\ \\
\frac{d^{\alpha_1}X_4(t)}{dt^{\alpha_1}} = a_{41} *X_4(t - \tau_4) + \\
a_{42} *\frac{X_4(t)^{\beta_1}}{(1-X_4(t - \tau_4))^{\beta_1}} - a_{45}X_4(t) + \\
\epsilon_4(t) \\ \\
\frac{d^{\alpha_1}X_5(t)}{dt^{\alpha_1}} = a_{51} *X_5(t - \tau_5) + \\
a_{52} *\frac{X_5(t)^{\beta_1}}{(1-X_5(t - \tau_5))^{\beta_1}} - a_{55}X_5(t) + \\
\epsilon_5(t) \\
\end{align*}




\section{Results}

\centering
\begin{table}[H]\footnotesize
	\caption{}
	\begin{tabular}{rp{1cm}p{2cm}p{3cm}p{1cm}}
		\hline
		ID & A & B & C & Reference \\
		\hline
		\hline
	\end{tabular}
\end{table}
\raggedright


\subsection{Tables}

\centering	
\begin{table}[H]\tiny
	\caption{}	
	\begin{tabular}{rp{1cm}|p{4cm}|l}
		\hline	
		Model ID & Topic & Description & Results \\
		\hline 
		\hline 
	\end{tabular}
\end{table}

\centering
\begin{table}[H]\footnotesize
	\caption{}
	\begin{tabular}{rp{1cm}p{2cm}p{3cm}p{1cm}}
		\hline
		ID & A & B & C & Reference \\
		\hline
		\hline
	\end{tabular}
\end{table}
\raggedright

\centering
\begin{table}[H]\footnotesize
	\caption{}
	\begin{tabular}{rp{1cm}p{2cm}p{3cm}p{1cm}}
		\hline
		ID & A & B & C & Reference \\
		\hline
		\hline
	\end{tabular}
\end{table}
\raggedright


\subsection{Figures}

\begin{figure}[H]
	\centering
	\begin{minipage}[b]{0.5\linewidth}
	%\includegraphics[scale=0.25]{Example_1_Figure_1.png}
	\end{minipage}\hfill
	\begin{minipage}[b]{0.5\linewidth}
	%\includegraphics[scale=0.25]{Example_1_Figure_2.png}
	\end{minipage}\hfill	
	\begin{minipage}[b]{0.5\linewidth}
	%\includegraphics[scale=0.25]{Example_1_Figure_3.png}
	\end{minipage}\hfill
	\begin{minipage}[b]{0.5\linewidth}
	%\includegraphics[scale=0.25]{Example_1_Figure_4.png}
	\end{minipage}\hfill
	\caption{1, 2, 3 and 4}
	\label{fig:Figure1}
\end{figure} 


\section{Conclusions}


\section{Topics for the Classroom}

\centering	
\begin{table}[H]\tiny
	\caption{}	
	\begin{tabular}{r|p{4cm}|l}
		\hline	
		Topic ID & Description & Results \\
		\hline 
		\hline 
	\end{tabular}
\end{table}

\section{R Application Programming Interfaces (APIs)}


\bibliographystyle{plain}
\begin{thebibliography}{00}

\bibitem[1000]{key1000}R Core Team (2015). 
\newblock R: A language and environment for statistical computing. R Foundation for Statistical Computing, Vienna, Austria.
\newblock URL https://www.R-project.org/.

\end{thebibliography}
\end{document}
