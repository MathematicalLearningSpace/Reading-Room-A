%-------------------------------------------------------------------------%
%--------------------Classroom Lecture Model Series-----------------------%
%-------------------------------------------------------------------------%

\begin{document}
\twocolumn
\scriptsize
\begin{frontmatter}
		\title{DRAFT Sample: A Mathematical Model of Transcriptional, Translational, Protein Folding, and Post Translational Errors}
		\author{\corref{cor1}\fnref{fn1}}
		\cortext[cor1]{Corresponding author}
		\address{The Mathematical Learning Space}
		\ead{http://mathlearningspace.weebly.com}	
\end{frontmatter}	

Introduction:
\begin{enumerate}
\item Objective 1:
\item Objective 2:
\item Objective 3:
\end{enumerate}
Conclusion:

Keywords: Transcriptional Errors, Translational Errors, Protein Folding Errors, Post Translational Errors
Vocabulary Words: Splicing Errors Exon Skipping, Intron Failure, Nucleotide Insertion and Polymerase slippage

\section{Introduction}

\begin{table}[H]\centering
	\begin{tabular}{p{1cm}p{4cm}p{3cm}}
		Article ID & Summary & Comments\\
		\hline
		\hline
	\end{tabular}
\end{table}



\subsection{Biology Review:Translational Transcription Error Classification}

\begin{enuemrate}
\item TLE-Residue Insertion Error
\item TLE-tRNA misacylation
\item TLE-Premature termination mRNA
\item TLE-Read through
\item TLE- Frameshift
\end{enumerate}

\begin{table}[H]\centering
	\begin{tabular}{p{1cm}p{4cm}p{3cm}}
		Article ID & Summary & Comments\\
		\hline
		\hline
	\end{tabular}
\end{table}


\subsection{Biology Review:Protein Folding Errors}

\begin{enumerate}
\item kinetic trapping
\item spontaneous unfolding 
\item Disruption of folding 
\item Erroneous polypeptide
\end{enumerate}

\begin{table}[H]\centering
	\begin{tabular}{p{1cm}p{4cm}p{3cm}}
		Article ID & Summary & Comments\\
		\hline
		\hline
	\end{tabular}
\end{table}


\subsection{Biology Review:Post translation Modification}

\begin{enumerate}
\item incorrect proteolyic cleavage  
\item errors ubiquitylation
\item glyeosylation
\item phosphorylation 
\item other post translation functionalization
\end{enumerate}

\begin{table}[H]\centering
	\begin{tabular}{p{1cm}p{4cm}p{3cm}}
		Article ID & Summary & Comments\\
		\hline
		\hline
	\end{tabular}
\end{table}


\subsection{Plan of the Article}

\begin{enumerate}
\item Review of Mathematical Topics
\item Presentation of the Results
\end{enumerate}

\section{Mathematical Background}

\subsection{Math Review Topic: Graphical Models}

\centering	
\begin{table}[H]\tiny
	\caption{}	
	\begin{tabular}{r|p{4cm}|l}
		\hline	
		Topic & Description & Results \\
		\hline 
		\hline 
	\end{tabular}
\end{table}

\section{Results}

\centering
\begin{table}[H]\footnotesize
	\caption{}
	\begin{tabular}{rp{1cm}p{2cm}p{3cm}p{1cm}}
		\hline
		ID & A & B & C & Reference \\
		\hline
		\hline
	\end{tabular}
\end{table}
\raggedright


\subsection{Tables}

\centering	
\begin{table}[H]\tiny
	\caption{}	
	\begin{tabular}{r|p{4cm}|l}
		\hline	
		Models & Description & Results \\
		\hline 
		\hline 
	\end{tabular}
\end{table}

\centering
\begin{table}[H]\footnotesize
	\caption{}
	\begin{tabular}{rp{1cm}p{2cm}p{3cm}p{1cm}}
		\hline
		ID & A & B & C & Reference \\
		\hline
		\hline
	\end{tabular}
\end{table}
\raggedright

\centering
\begin{table}[H]\footnotesize
	\caption{}
	\begin{tabular}{rp{1cm}p{2cm}p{3cm}p{1cm}}
		\hline
		ID & A & B & C & Reference \\
		\hline
		\hline
	\end{tabular}
\end{table}
\raggedright


\subsection{Figures}

\begin{figure}[H]
	\centering
	\begin{minipage}[b]{0.5\linewidth}
	%\includegraphics[scale=0.25]{Example_1_Figure_1.png}
	\end{minipage}\hfill
	\begin{minipage}[b]{0.5\linewidth}
	%\includegraphics[scale=0.25]{Example_1_Figure_2.png}
	\end{minipage}\hfill	
	\begin{minipage}[b]{0.5\linewidth}
	%\includegraphics[scale=0.25]{Example_1_Figure_3.png}
	\end{minipage}\hfill
	\begin{minipage}[b]{0.5\linewidth}
	%\includegraphics[scale=0.25]{Example_1_Figure_4.png}
	\end{minipage}\hfill
	\caption{1, 2, 3 and 4}
	\label{fig:Figure1}
\end{figure} 


\section{Conclusions}

\begin{enuemrate}
\end{enumerate}


\section{Additional Topics in the Classroom}

\centering	
\begin{table}[H]\tiny
	\caption{}	
	\begin{tabular}{r|p{4cm}|l}
		\hline	
		Topic & Description & Comments \\
		\hline 
		\hline 
	\end{tabular}
\end{table}

\section{R Application Programming Interfaces (APIs)}


\bibliographystyle{plain}
\begin{thebibliography}{00}

\bibitem[1]{key1}Bintu, L.,Buchler, N. E.,Garcia, H. G.,Gerland, U.,Hwa, T.,Kondev, J.,Kuhlman, T.andPhillips, R.(2005a). 
\newblock Transcriptional regulation by the numbers:Applications. 
\newblock Curr. Opin. Genet. Dev. 15125–135.

\bibitem[2]{key2}Bintu, L.,Buchler, N. E.,Garcia, H. G.,Gerland, U.,Hwa, T.,Kondev, J.and Phillips, R.(2005b). 
\newblock Transcriptional regulation by the numbers: Models.
\newblock Curr. Opin. Genet. Dev.15116–124.

\bibitem[3]{key3}Alberts, B.,Johnson, A.,Lewis, J.,Raff, M.,Roberts, K. and Walter, P. (2007).
\newblock Molecular Biology of the Cell
\newblock , 5th ed. Garland, New York, NY

\bibitem[4]{key4}Ting Chen, Hongyu He, and George M. Church
\newblock Modeling Gene Expression with Differential Equations 1999

\bibitem[5]{key5}Prelich, G. (2012). 
\newblock Gene Overexpression: Uses, Mechanisms, and Interpretation. 
\newblock Genetics, 190(3), 841–854. http://doi.org/10.1534/genetics.111.136911

\bibitem[6]{key6} Giuseppe Jordão
\newblock Mathematical Models in Cancer Systems Biology
\newblock University of Porto Portugal 2017

\bibitem[1000]{key1000}R Core Team (2015). 
\newblock R: A language and environment for statistical computing. R Foundation for Statistical Computing, Vienna, Austria.
\newblock URL https://www.R-project.org/.

\subsection{Biology Review:Translational Transcription Error Classification}

\subsection{Biology Review:Protein Folding Errors}

\subsection{Biology Review:Post translation Modification}

\end{thebibliography}
\end{document}
