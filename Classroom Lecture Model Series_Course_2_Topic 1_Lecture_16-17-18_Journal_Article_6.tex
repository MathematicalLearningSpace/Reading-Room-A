%-------------------------------------------------------------------------%
%--------------------Classroom Lecture Model Series-----------------------%
%-------------------------------------------------------------------------%

%--------------------Work in Progress-------------------------------------%

\begin{document}
\twocolumn
\scriptsize
\begin{frontmatter}
		\title{}
		\author{\corref{cor1}\fnref{fn1}}
		\cortext[cor1]{Corresponding author}
		\address{The Mathematical Learning Space}
		\ead{http://mathlearningspace.weebly.com}	
\end{frontmatter}	

Introduction:
\begin{enumerate}
\item Objective 1:
\item Objective 2:
\item Objective 3:
\end{enumerate}
Conclusion:

Keywords:  Longevity,Immune System,Wnt,Notch,Hedgehog,TGF-Beta and VGF
Vocabulary Words:

\section{Introduction}

\begin{table}[H]\centering
	\begin{tabular}{p{1cm}p{4cm}p{3cm}}
	Article ID & Summary & Comments\\
		\hline
		\hline
	\end{tabular}
\end{table}

\subsection{Plan of the Article}

\begin{enumerate}
\end{enumerate}

\section{Biolog Review: Network A}

\centering	
\begin{table}[H]\tiny
	\caption{}	
	\begin{tabular}{rp{1cm}|p{4cm}|l}
		\hline	
		PubmedID & Topic & Description & Citation \\
		\hline 
		\hline 
	\end{tabular}
\end{table}


\section{Mathematical Background}

\subsection{Mathematical Model}

\subsubsection{Longevity} 

\subsubsection{Immune System} 

\subsubsection{Wnt} 

\subsubsection{Notch} 

\subsubsection{Hedgehog} 

\subsubsection{TGF-Beta} 

\subsubsection{VEGF}


\subsubsection{Initial Conditions and Noise Distributions}

\centering	
\begin{table}[H]\tiny
	\caption{}	
	\begin{tabular}{rp{1cm}|p{4cm}|l}
		\hline	
		Noise ID & Description & Properties & Results \\
		\hline 
		\hline 
	\end{tabular}
\end{table}

\subsubsection{Longevity} 

\subsubsection{Parameter Matrix}
\vspace{4pt}
\centering
\begin{table}[h]\footnotesize
	\caption{Parameter Description and Value}
	\begin{tabular}{rllp{2cm}l}
		\hline	
		Parameter & Value & Interval & Description & Reference \\
		\hline 
	\end{tabular}	
\end{table}


\subsubsection{Immune System}

\subsubsection{Parameter Matrix}
\vspace{4pt}
\centering
\begin{table}[h]\footnotesize
	\caption{Parameter Description and Value}
	\begin{tabular}{rllp{2cm}l}
		\hline	
		Parameter & Value & Interval & Description & Reference \\
		\hline 
	\end{tabular}	
\end{table}


\subsubsection{Wnt} 

\subsubsection{Parameter Matrix}
\vspace{4pt}
\centering
\begin{table}[h]\footnotesize
	\caption{Parameter Description and Value}
	\begin{tabular}{rllp{2cm}l}
		\hline	
		Parameter & Value & Interval & Description & Reference \\
		\hline 
	\end{tabular}	
\end{table}


\subsubsection{Notch} 

\subsubsection{Parameter Matrix}
\vspace{4pt}
\centering
\begin{table}[h]\footnotesize
	\caption{Parameter Description and Value}
	\begin{tabular}{rllp{2cm}l}
		\hline	
		Parameter & Value & Interval & Description & Reference \\
		\hline 
	\end{tabular}	
\end{table}


\subsubsection{Hedgehog} 

\subsubsection{Parameter Matrix}
\vspace{4pt}
\centering
\begin{table}[h]\footnotesize
	\caption{Parameter Description and Value}
	\begin{tabular}{rllp{2cm}l}
		\hline	
		Parameter & Value & Interval & Description & Reference \\
		\hline 
	\end{tabular}	
\end{table}


\subsubsection{TGF-Beta} 

\subsubsection{Parameter Matrix}
\vspace{4pt}
\centering
\begin{table}[h]\footnotesize
	\caption{Parameter Description and Value}
	\begin{tabular}{rllp{2cm}l}
		\hline	
		Parameter & Value & Interval & Description & Reference \\
		\hline 
	\end{tabular}	
\end{table}

\subsubsection{VEGF}

\subsubsection{Parameter Matrix}
\vspace{4pt}
\centering
\begin{table}[h]\footnotesize
	\caption{Parameter Description and Value}
	\begin{tabular}{rllp{2cm}l}
		\hline	
		Parameter & Value & Interval & Description & Reference \\
		\hline 
	\end{tabular}	
\end{table}


\section{Results}

\subsection{Tables}

\centering	
\begin{table}[H]\tiny
	\caption{}	
	\begin{tabular}{rp{1cm}|p{4cm}|l}
		\hline	
		Model ID & Description & Properties & Results \\
		\hline 
		\hline 
	\end{tabular}
\end{table}

\subsection{Figures}

\begin{figure}[H]
	\centering
	\begin{minipage}[b]{0.5\linewidth}
	%\includegraphics[scale=0.25]{Example_1_Figure_1.png}
	\end{minipage}\hfill
	\begin{minipage}[b]{0.5\linewidth}
	%\includegraphics[scale=0.25]{Example_1_Figure_2.png}
	\end{minipage}\hfill	
	\begin{minipage}[b]{0.5\linewidth}
	%\includegraphics[scale=0.25]{Example_1_Figure_3.png}
	\end{minipage}\hfill
	\begin{minipage}[b]{0.5\linewidth}
	%\includegraphics[scale=0.25]{Example_1_Figure_4.png}
	\end{minipage}\hfill
	\caption{1, 2, 3 and 4}
	\label{fig:Figure1}
\end{figure} 



\section{Conclusions}

\section{Topics for the Classroom}

\begin{enumerate}
\end{enumerate}


\section{R Application Programming Interfaces (APIs)}


\bibliographystyle{plain}
\begin{thebibliography}{00}

\bibitem[400]{key400} Kanehisa, Furumichi, M., Tanabe, M., Sato, Y., and Morishima, K.; 
\newblock KEGG: new perspectives on genomes, pathways, diseases and drugs. 
\newblock Nucleic Acids Res. 45, D353-D361 (2017).

\bibitem[401]{key401} Kanehisa, M., Sato, Y., Kawashima, M., Furumichi, M., and Tanabe, M.; 
\newblock KEGG as a reference resource for gene and protein annotation. 
\newblock Nucleic Acids Res. 44, D457-D462 (2016).

\bibitem[402]{key402} Kanehisa, M. and Goto, S.; 
\newblock KEGG: Kyoto Encyclopedia of Genes and Genomes. 
\newblock Nucleic Acids Res. 28, 27-30 (2000). 

\bibitem[500]{key500} Cantorna MT1, Zhu Y, Froicu M, Wittke A.
\newblock Vitamin D status, 1,25-dihydroxyvitamin D3, and the immune system.
\newblock Am J Clin Nutr. 2004 Dec;80(6 Suppl):1717S-20S. doi: 10.1093/ajcn/80.6.1717S.

\subsection{Longevity} 

\subsection{Immune System} 

\subsection{Wnt} 

\subsection{Notch} 

\subsection{Hedgehog} 

\subsection{TGF-Beta} 

\subsection{VEGF}


\end{thebibliography}
\end{document}
