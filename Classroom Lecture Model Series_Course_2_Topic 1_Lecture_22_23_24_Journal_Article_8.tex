%-------------------------------------------------------------------------%
%--------------------Classroom Lecture Model Series-----------------------%
%-------------------------------------------------------------------------%

%--------------------Work in Progress for the Classroom-------------------%

\begin{document}
\twocolumn
\scriptsize
\begin{frontmatter}
		\title{DRAFT Sample: A Mathematical Model of Heterodimer DNA helix Bending}
		\author{\corref{cor1}\fnref{fn1}}
		\cortext[cor1]{Corresponding author}
		\address{The Mathematical Learning Space}
		\ead{http://mathlearningspace.weebly.com}	
\end{frontmatter}	

Introduction:
\begin{enumerate}
\item Objective 1:
\item Objective 2:
\item Objective 3:
\end{enumerate}
Conclusion:

Keywords: Digestion I and II Sugar and Insulin
Vocabulary Words:

\section{Introduction}


\begin{figure}[H]
\begin{minipage}[b]{0.3\linewidth}
\includegraphics[scale=0.40]{Digestion.png} 
\end{minipage}\hfill
\caption{(a) Digestion}
\label{fig:Figure2}
\end{figure} 


\centering	
\begin{table}[H]\tiny
	\caption{Biological Processes in Digestion}	
	\begin{tabular}{rp{1cm}|p{4cm}|l}
		\hline	
		Biological Process & Topic & Description & Citation \\
		\hline 
		Salivary secretion & & \cite{key400} \\
		Gastric acid secretion & & \cite{key400} \\
		Pancreatic secretion & & \cite{key400} \\
		Bile secretion & & \cite{key400} \\
		\hline
		Carbohydrate digestion and absorption & & \cite{key400} \\
		Protein digestion and absorption & & \cite{key400} \\
		Cholesterol metabolism & & \cite{key400} \\
		\hline
		Fat digestion and absorption & & \cite{key400} \\
		Vitamin digestion and absorption & & \cite{key400} \\
		Mineral absorption & & \cite{key400} \\
		\hline 
	\end{tabular}
\end{table}


\subsection{Plan of the Article}

\begin{enumerate}
\end{enumerate}

\section{Topic Review Signal Network:}

\begin{table}[H]
\tiny
\begin{tabular}{p{1cm}p{1cm}p{3cm}p{1cm}p{1cm}} 
EdgeID & Gene 1 & Gene 2 & Biological Function & Comment \\
\hline
\hline
\end{tabular}
\caption{Signal Network}
\label{tab:Table2}
\end{table}

\begin{center}
	\begin{tikzpicture}
	[->,>=stealth',shorten >=2pt,node distance=3cm,
	thick,main node/.style={circle,draw,scale=0.25,transform canvas={scale=0.75},font=\sffamily\Small\bfseries},
	blacknode/.style={shape=circle, draw=black, line width=2},
	bluenode/.style={shape=circle, draw=blue, line width=2},
	greennode/.style={shape=circle, draw=green, line width=2},
	rednode/.style={shape=circle, draw=red, line width=2}
	]
%-------Legend ------------------------------------------------------------	
	\matrix [draw,below left] at (current bounding box.south) {
		\node [state,label=right:State] {}; A. \\
		\node [shapeSquare,label=right:Square- B.] {}; \\
		\node [shapeEllipse,label=right:Ellipse-C.] {}; \\
		\node [shapeTriangle,label=right:Triangle-D.] {}; \\
		\node [shapeHexagon,label=right:Hexagon-E.] {}; \\
	};
	\end{tikzpicture}
\end{center}

\section{Mathematical Background: Topic Review}

\centering
\begin{table}[H]\footnotesize
	\caption{}
	\begin{tabular}{rp{1cm}p{2cm}p{3cm}p{1cm}}
		\hline
		ID & A & B & C & Reference \\
		\hline
		\hline
	\end{tabular}
\end{table}
\raggedright


\subsection{Mathematical Model}

\centering
\begin{table}[H]\footnotesize
	\caption{}
	\begin{tabular}{rp{1cm}p{2cm}p{3cm}p{1cm}}
		\hline
		ID & A & B & C & Reference \\
		\hline
		\hline
	\end{tabular}
\end{table}
\raggedright


\subsubsection{Digestion I} 

\begin{figure}[H]
\begin{minipage}[b]{0.3\linewidth}
\includegraphics[scale=0.40]{Digestion_1.png} 
\end{minipage}\hfill
\caption{(a) Digestion_1}
\label{fig:Figure2}
\end{figure} 

\centering
\begin{table}[H]\footnotesize
	\caption{}
	\begin{tabular}{rp{1cm}p{2cm}p{3cm}p{1cm}}
		\hline
		ID & A & B & C & Reference \\
		\hline
		\hline
	\end{tabular}
\end{table}
\raggedright

\section{Algorithms: Digestion I}

\begin{algorithm}[H]
	\footnotesize
	\begin{algorithmic}[1]
	\State 
	\end{algorithmic}
	\caption{Digestion I}\label{Digestion_I_1}
\end{algorithm}

\subsubsection{Protein Interaction Diagram and Annotations}

\begin{tikzpicture}
	[->,>=stealth',shorten >=2pt,node distance=3cm,
	thick,main node/.style={circle,draw,scale=0.25,transform canvas={scale=0.75},font=\sffamily\Small\bfseries},
	blacknode/.style={shape=circle, draw=black, line width=2},
	bluenode/.style={shape=circle, draw=blue, line width=2},
	greennode/.style={shape=circle, draw=green, line width=2},
	rednode/.style={shape=circle, draw=red, line width=2}
	]
	%-------Legend ------------------------------------------------------------	
	\matrix [draw,below left] at (current bounding box.south) {
		\node [state,label=right:State] {}; Description. \\
		\node [shapeSquare,label=right:Square-.] {}; \\
		\node [shapeEllipse,label=right:Ellipse-.] {}; \\
		\node [shapeTriangle,label=right:Triangle-.] {}; \\
		\node [shapeHexagon,label=right:Hexagon-.] {}; \\
	};
\end{tikzpicture}

\centering	
\begin{table}[H]\tiny
\caption{Gene/Protein/Enzymes Description and References}	
\begin{tabular}{r|p{3cm}|l}
\hline	
Gene/Protein & Description & Reference \\
\hline 
\hline 
\end{tabular}
\end{table}


\subsubsection{Equation System A}

\begin{align*} 
\tiny
\frac{d^{\alpha_1}X_1(t)}{dt^{\alpha_1}} = a_{11} *X_1(t - \tau_1) + \\
a_{12} *\frac{X_1(t)^{\beta_1}}{(1-X_1(t - \tau_1))^{\beta_1}} - a_{15}X_1(t) + \\
\epsilon_1(t) \\
\frac{d^{\alpha_1}X_2(t)}{dt^{\alpha_1}} = a_{21} *X_2(t - \tau_2) + \\
a_{22} *\frac{X_2(t)^{\beta_1}}{(1-X_2(t - \tau_2))^{\beta_1}} - a_{25}X_2(t) + \\
\epsilon_2(t) \\ \\
\frac{d^{\alpha_1}X_3(t)}{dt^{\alpha_1}} = a_{31} *X_3(t - \tau_3) + \\
a_{32} *\frac{X_3(t)^{\beta_1}}{(1-X_3(t - \tau_3))^{\beta_1}} - a_{35}X_3(t) + \\
\epsilon_3(t) \\ \\
\frac{d^{\alpha_1}X_4(t)}{dt^{\alpha_1}} = a_{41} *X_4(t - \tau_4) + \\
a_{42} *\frac{X_4(t)^{\beta_1}}{(1-X_4(t - \tau_4))^{\beta_1}} - a_{45}X_4(t) + \\
\epsilon_4(t) \\ \\
\frac{d^{\alpha_1}X_5(t)}{dt^{\alpha_1}} = a_{51} *X_5(t - \tau_5) + \\
a_{52} *\frac{X_5(t)^{\beta_1}}{(1-X_5(t - \tau_5))^{\beta_1}} - a_{55}X_5(t) + \\
\epsilon_5(t) \\
\end{align*}

\subsubsection{Parameter Table}

\begin{table}[h]\footnotesize
	\caption{Parameter Description and Value}
	\begin{tabular}{rllp{2cm}l}
		\hline	
		Parameter & Value & Interval & Description & Reference \\
		\hline 
		a11 & 0 & [0,1] & Equation 1 & \cite{key1}
		a12 & 0 & [0,1] & Equation 1 & \cite{key1}
		a13 & 0 & [0,1] & Equation 1 & \cite{key1}
		a14 & 0 & [0,1] & Equation 1 & \cite{key1}
		a15 & 0 & [0,1] & Equation 1 & \cite{key1}
		\hline
		a21 & 0 & [0,1] & Equation 2 & \cite{key1}
		a22 & 0 & [0,1] & Equation 2 & \cite{key1}
		a23 & 0 & [0,1] & Equation 2 & \cite{key1}
		a24 & 0 & [0,1] & Equation 2 & \cite{key1}
		a25 & 0 & [0,1] & Equation 2 & \cite{key1}
		\hline
		a31 & 0 & [0,1] & Equation 3 & \cite{key1}
		a32 & 0 & [0,1] & Equation 3 & \cite{key1}
		a33 & 0 & [0,1] & Equation 3 & \cite{key1}
		a34 & 0 & [0,1] & Equation 3 & \cite{key1}
		a35 & 0 & [0,1] & Equation 3 & \cite{key1}
		\hline
		a41 & 0 & [0,1] & Equation 4 & \cite{key1}
		a42 & 0 & [0,1] & Equation 4 & \cite{key1}
		a43 & 0 & [0,1] & Equation 4 & \cite{key1}
		a44 & 0 & [0,1] & Equation 4 & \cite{key1}
		a45 & 0 & [0,1] & Equation 4 & \cite{key1}
		\hline
		a51 & 0 & [0,1] & Equation 5 & \cite{key1}
		a52 & 0 & [0,1] & Equation 5 & \cite{key1}
		a53 & 0 & [0,1] & Equation 5 & \cite{key1}
		a54 & 0 & [0,1] & Equation 5 & \cite{key1}
		a55 & 0 & [0,1] & Equation 5 & \cite{key1}
		\hline
		$\tau_1$ & 1 & [1,2] & Equation 1 & \cite{key1}
		$\tau_2$ & 1 & [1,2] & Equation 2 & \cite{key1}
		$\tau_3$ & 1 & [1,2] & Equation 3 & \cite{key1}
		$\tau_4$ & 1 & [1,2] & Equation 4 & \cite{key1}
		$\tau_5$ & 1 & [1,2] & Equation 5 & \cite{key1}
		\hline
		$\alpha_1$ & 1 & (0,2] & Equation 1 & \cite{key1}
		$\alpha_2$ & 1 & (0,2] & Equation 2 & \cite{key1}
		$\alpha_3$ & 1 & (0,2] & Equation 3 & \cite{key1}
		$\alpha_4$ & 1 & (0,2] & Equation 4 & \cite{key1}
		$\alpha_5$ & 1 & (0,2] & Equation 5 & \cite{key1}
		\hline
		$\beta_1$ & 1 & (0,2] & Equation 1 & \cite{key1}
		$\beta_2$ & 1 & (0,2] & Equation 2 & \cite{key1}
		$\beta_3$ & 1 & (0,2] & Equation 3 & \cite{key1}
		$\beta_4$ & 1 & (0,2] & Equation 4 & \cite{key1}
		$\beta_5$ & 1 & (0,2] & Equation 5 & \cite{key1}
	\end{tabular}	
\end{table}

\subsubsection{Digestion II} 

\begin{figure}[H]
\begin{minipage}[b]{0.3\linewidth}
\includegraphics[scale=0.40]{Digestion_2.png} 
\end{minipage}\hfill
\caption{(a) Digestion 2}
\label{fig:Figure2}
\end{figure} 

\centering
\begin{table}[H]\footnotesize
	\caption{}
	\begin{tabular}{rp{1cm}p{2cm}p{3cm}p{1cm}}
		\hline
		ID & A & B & C & Reference \\
		\hline
		\hline
	\end{tabular}
\end{table}
\raggedright

\section{Algorithms: Digestion II}

\begin{algorithm}[H]
	\footnotesize
	\begin{algorithmic}[1]
	\State 
	\end{algorithmic}
	\caption{Digestion II}\label{Digestion_II_1}
\end{algorithm}

\subsubsection{Protein Interaction Diagram and Annotations}

\begin{tikzpicture}
	[->,>=stealth',shorten >=2pt,node distance=3cm,
	thick,main node/.style={circle,draw,scale=0.25,transform canvas={scale=0.75},font=\sffamily\Small\bfseries},
	blacknode/.style={shape=circle, draw=black, line width=2},
	bluenode/.style={shape=circle, draw=blue, line width=2},
	greennode/.style={shape=circle, draw=green, line width=2},
	rednode/.style={shape=circle, draw=red, line width=2}
	]
	%-------Legend ------------------------------------------------------------	
	\matrix [draw,below left] at (current bounding box.south) {
		\node [state,label=right:State] {}; Description. \\
		\node [shapeSquare,label=right:Square-.] {}; \\
		\node [shapeEllipse,label=right:Ellipse-.] {}; \\
		\node [shapeTriangle,label=right:Triangle-.] {}; \\
		\node [shapeHexagon,label=right:Hexagon-.] {}; \\
	};
\end{tikzpicture}

\centering	
\begin{table}[H]\tiny
\caption{Gene/Protein/Enzymes Description and References}	
\begin{tabular}{r|p{3cm}|l}
\hline	
Gene/Protein & Description & Reference \\
\hline 
\hline 
\end{tabular}
\end{table}

\subsubsection{Equation System A}

\begin{align*} 
\tiny
\frac{d^{\alpha_1}X_1(t)}{dt^{\alpha_1}} = a_{11} *X_1(t - \tau_1) + \\
a_{12} *\frac{X_1(t)^{\beta_1}}{(1-X_1(t - \tau_1))^{\beta_1}} - a_{15}X_1(t) + \\
\epsilon_1(t) \\
\frac{d^{\alpha_1}X_2(t)}{dt^{\alpha_1}} = a_{21} *X_2(t - \tau_2) + \\
a_{22} *\frac{X_2(t)^{\beta_1}}{(1-X_2(t - \tau_2))^{\beta_1}} - a_{25}X_2(t) + \\
\epsilon_2(t) \\ \\
\frac{d^{\alpha_1}X_3(t)}{dt^{\alpha_1}} = a_{31} *X_3(t - \tau_3) + \\
a_{32} *\frac{X_3(t)^{\beta_1}}{(1-X_3(t - \tau_3))^{\beta_1}} - a_{35}X_3(t) + \\
\epsilon_3(t) \\ \\
\frac{d^{\alpha_1}X_4(t)}{dt^{\alpha_1}} = a_{41} *X_4(t - \tau_4) + \\
a_{42} *\frac{X_4(t)^{\beta_1}}{(1-X_4(t - \tau_4))^{\beta_1}} - a_{45}X_4(t) + \\
\epsilon_4(t) \\ \\
\frac{d^{\alpha_1}X_5(t)}{dt^{\alpha_1}} = a_{51} *X_5(t - \tau_5) + \\
a_{52} *\frac{X_5(t)^{\beta_1}}{(1-X_5(t - \tau_5))^{\beta_1}} - a_{55}X_5(t) + \\
\epsilon_5(t) \\
\end{align*}

\subsubsection{Parameter Table}

\begin{table}[h]\footnotesize
	\caption{Parameter Description and Value}
	\begin{tabular}{rllp{2cm}l}
		\hline	
		Parameter & Value & Interval & Description & Reference \\
		\hline 
		a11 & 0 & [0,1] & Equation 1 & \cite{key1}
		a12 & 0 & [0,1] & Equation 1 & \cite{key1}
		a13 & 0 & [0,1] & Equation 1 & \cite{key1}
		a14 & 0 & [0,1] & Equation 1 & \cite{key1}
		a15 & 0 & [0,1] & Equation 1 & \cite{key1}
		\hline
		a21 & 0 & [0,1] & Equation 2 & \cite{key1}
		a22 & 0 & [0,1] & Equation 2 & \cite{key1}
		a23 & 0 & [0,1] & Equation 2 & \cite{key1}
		a24 & 0 & [0,1] & Equation 2 & \cite{key1}
		a25 & 0 & [0,1] & Equation 2 & \cite{key1}
		\hline
		a31 & 0 & [0,1] & Equation 3 & \cite{key1}
		a32 & 0 & [0,1] & Equation 3 & \cite{key1}
		a33 & 0 & [0,1] & Equation 3 & \cite{key1}
		a34 & 0 & [0,1] & Equation 3 & \cite{key1}
		a35 & 0 & [0,1] & Equation 3 & \cite{key1}
		\hline
		a41 & 0 & [0,1] & Equation 4 & \cite{key1}
		a42 & 0 & [0,1] & Equation 4 & \cite{key1}
		a43 & 0 & [0,1] & Equation 4 & \cite{key1}
		a44 & 0 & [0,1] & Equation 4 & \cite{key1}
		a45 & 0 & [0,1] & Equation 4 & \cite{key1}
		\hline
		a51 & 0 & [0,1] & Equation 5 & \cite{key1}
		a52 & 0 & [0,1] & Equation 5 & \cite{key1}
		a53 & 0 & [0,1] & Equation 5 & \cite{key1}
		a54 & 0 & [0,1] & Equation 5 & \cite{key1}
		a55 & 0 & [0,1] & Equation 5 & \cite{key1}
		\hline
		$\tau_1$ & 1 & [1,2] & Equation 1 & \cite{key1}
		$\tau_2$ & 1 & [1,2] & Equation 2 & \cite{key1}
		$\tau_3$ & 1 & [1,2] & Equation 3 & \cite{key1}
		$\tau_4$ & 1 & [1,2] & Equation 4 & \cite{key1}
		$\tau_5$ & 1 & [1,2] & Equation 5 & \cite{key1}
		\hline
		$\alpha_1$ & 1 & (0,2] & Equation 1 & \cite{key1}
		$\alpha_2$ & 1 & (0,2] & Equation 2 & \cite{key1}
		$\alpha_3$ & 1 & (0,2] & Equation 3 & \cite{key1}
		$\alpha_4$ & 1 & (0,2] & Equation 4 & \cite{key1}
		$\alpha_5$ & 1 & (0,2] & Equation 5 & \cite{key1}
		\hline
		$\beta_1$ & 1 & (0,2] & Equation 1 & \cite{key1}
		$\beta_2$ & 1 & (0,2] & Equation 2 & \cite{key1}
		$\beta_3$ & 1 & (0,2] & Equation 3 & \cite{key1}
		$\beta_4$ & 1 & (0,2] & Equation 4 & \cite{key1}
		$\beta_5$ & 1 & (0,2] & Equation 5 & \cite{key1}
	\end{tabular}	
\end{table}

\subsubsection{Sugar Model} 

\begin{figure}[H]
\begin{minipage}[b]{0.3\linewidth}
\includegraphics[scale=0.40]{Sugar.png} 
\end{minipage}\hfill
\caption{(a) Sugar}
\label{fig:Figure2}
\end{figure} 

\centering
\begin{table}[H]\footnotesize
	\caption{}
	\begin{tabular}{rp{1cm}p{2cm}p{3cm}p{1cm}}
		\hline
		ID & A & B & C & Reference \\
		\hline
		\hline
	\end{tabular}
\end{table}
\raggedright

\subsubsection{Protein Interaction Diagram and Annotations}

\begin{tikzpicture}
	[->,>=stealth',shorten >=2pt,node distance=3cm,
	thick,main node/.style={circle,draw,scale=0.25,transform canvas={scale=0.75},font=\sffamily\Small\bfseries},
	blacknode/.style={shape=circle, draw=black, line width=2},
	bluenode/.style={shape=circle, draw=blue, line width=2},
	greennode/.style={shape=circle, draw=green, line width=2},
	rednode/.style={shape=circle, draw=red, line width=2}
	]
	%-------Legend ------------------------------------------------------------	
	\matrix [draw,below left] at (current bounding box.south) {
		\node [state,label=right:State] {}; Description. \\
		\node [shapeSquare,label=right:Square-.] {}; \\
		\node [shapeEllipse,label=right:Ellipse-.] {}; \\
		\node [shapeTriangle,label=right:Triangle-.] {}; \\
		\node [shapeHexagon,label=right:Hexagon-.] {}; \\
	};
\end{tikzpicture}

\centering	
\begin{table}[H]\tiny
\caption{Gene/Protein/Enzymes Description and References}	
\begin{tabular}{r|p{3cm}|l}
\hline	
Gene/Protein & Description & Reference \\
\hline 
\hline 
\end{tabular}
\end{table}

\subsubsection{Equation System A}

\begin{align*} 
\tiny
\frac{d^{\alpha_1}X_1(t)}{dt^{\alpha_1}} = a_{11} *X_1(t - \tau_1) + \\
a_{12} *\frac{X_1(t)^{\beta_1}}{(1-X_1(t - \tau_1))^{\beta_1}} - a_{15}X_1(t) + \\
\epsilon_1(t) \\
\frac{d^{\alpha_1}X_2(t)}{dt^{\alpha_1}} = a_{21} *X_2(t - \tau_2) + \\
a_{22} *\frac{X_2(t)^{\beta_1}}{(1-X_2(t - \tau_2))^{\beta_1}} - a_{25}X_2(t) + \\
\epsilon_2(t) \\ \\
\frac{d^{\alpha_1}X_3(t)}{dt^{\alpha_1}} = a_{31} *X_3(t - \tau_3) + \\
a_{32} *\frac{X_3(t)^{\beta_1}}{(1-X_3(t - \tau_3))^{\beta_1}} - a_{35}X_3(t) + \\
\epsilon_3(t) \\ \\
\frac{d^{\alpha_1}X_4(t)}{dt^{\alpha_1}} = a_{41} *X_4(t - \tau_4) + \\
a_{42} *\frac{X_4(t)^{\beta_1}}{(1-X_4(t - \tau_4))^{\beta_1}} - a_{45}X_4(t) + \\
\epsilon_4(t) \\ \\
\frac{d^{\alpha_1}X_5(t)}{dt^{\alpha_1}} = a_{51} *X_5(t - \tau_5) + \\
a_{52} *\frac{X_5(t)^{\beta_1}}{(1-X_5(t - \tau_5))^{\beta_1}} - a_{55}X_5(t) + \\
\epsilon_5(t) \\
\end{align*}

\subsubsection{Parameter Table}

\begin{table}[h]\footnotesize
	\caption{Parameter Description and Value}
	\begin{tabular}{rllp{2cm}l}
		\hline	
		Parameter & Value & Interval & Description & Reference \\
		\hline 
		a11 & 0 & [0,1] & Equation 1 & \cite{key1}
		a12 & 0 & [0,1] & Equation 1 & \cite{key1}
		a13 & 0 & [0,1] & Equation 1 & \cite{key1}
		a14 & 0 & [0,1] & Equation 1 & \cite{key1}
		a15 & 0 & [0,1] & Equation 1 & \cite{key1}
		\hline
		a21 & 0 & [0,1] & Equation 2 & \cite{key1}
		a22 & 0 & [0,1] & Equation 2 & \cite{key1}
		a23 & 0 & [0,1] & Equation 2 & \cite{key1}
		a24 & 0 & [0,1] & Equation 2 & \cite{key1}
		a25 & 0 & [0,1] & Equation 2 & \cite{key1}
		\hline
		a31 & 0 & [0,1] & Equation 3 & \cite{key1}
		a32 & 0 & [0,1] & Equation 3 & \cite{key1}
		a33 & 0 & [0,1] & Equation 3 & \cite{key1}
		a34 & 0 & [0,1] & Equation 3 & \cite{key1}
		a35 & 0 & [0,1] & Equation 3 & \cite{key1}
		\hline
		a41 & 0 & [0,1] & Equation 4 & \cite{key1}
		a42 & 0 & [0,1] & Equation 4 & \cite{key1}
		a43 & 0 & [0,1] & Equation 4 & \cite{key1}
		a44 & 0 & [0,1] & Equation 4 & \cite{key1}
		a45 & 0 & [0,1] & Equation 4 & \cite{key1}
		\hline
		a51 & 0 & [0,1] & Equation 5 & \cite{key1}
		a52 & 0 & [0,1] & Equation 5 & \cite{key1}
		a53 & 0 & [0,1] & Equation 5 & \cite{key1}
		a54 & 0 & [0,1] & Equation 5 & \cite{key1}
		a55 & 0 & [0,1] & Equation 5 & \cite{key1}
		\hline
		$\tau_1$ & 1 & [1,2] & Equation 1 & \cite{key1}
		$\tau_2$ & 1 & [1,2] & Equation 2 & \cite{key1}
		$\tau_3$ & 1 & [1,2] & Equation 3 & \cite{key1}
		$\tau_4$ & 1 & [1,2] & Equation 4 & \cite{key1}
		$\tau_5$ & 1 & [1,2] & Equation 5 & \cite{key1}
		\hline
		$\alpha_1$ & 1 & (0,2] & Equation 1 & \cite{key1}
		$\alpha_2$ & 1 & (0,2] & Equation 2 & \cite{key1}
		$\alpha_3$ & 1 & (0,2] & Equation 3 & \cite{key1}
		$\alpha_4$ & 1 & (0,2] & Equation 4 & \cite{key1}
		$\alpha_5$ & 1 & (0,2] & Equation 5 & \cite{key1}
		\hline
		$\beta_1$ & 1 & (0,2] & Equation 1 & \cite{key1}
		$\beta_2$ & 1 & (0,2] & Equation 2 & \cite{key1}
		$\beta_3$ & 1 & (0,2] & Equation 3 & \cite{key1}
		$\beta_4$ & 1 & (0,2] & Equation 4 & \cite{key1}
		$\beta_5$ & 1 & (0,2] & Equation 5 & \cite{key1}
	\end{tabular}	
\end{table}


\subsubsection{Insulin Model}

\begin{figure}[H]
\begin{minipage}[b]{0.3\linewidth}
\includegraphics[scale=0.40]{Insulin.png} 
\end{minipage}\hfill
\caption{(a) Insulin}
\label{fig:Figure2}
\end{figure} 

\centering
\begin{table}[H]\footnotesize
	\caption{}
	\begin{tabular}{rp{1cm}p{2cm}p{3cm}p{1cm}}
		\hline
		ID & A & B & C & Reference \\
		\hline
		\hline
	\end{tabular}
\end{table}
\raggedright

\subsubsection{Protein Interaction Diagram and Annotations}

\begin{tikzpicture}
	[->,>=stealth',shorten >=2pt,node distance=3cm,
	thick,main node/.style={circle,draw,scale=0.25,transform canvas={scale=0.75},font=\sffamily\Small\bfseries},
	blacknode/.style={shape=circle, draw=black, line width=2},
	bluenode/.style={shape=circle, draw=blue, line width=2},
	greennode/.style={shape=circle, draw=green, line width=2},
	rednode/.style={shape=circle, draw=red, line width=2}
	]
	%-------Legend ------------------------------------------------------------	
	\matrix [draw,below left] at (current bounding box.south) {
		\node [state,label=right:State] {}; Description. \\
		\node [shapeSquare,label=right:Square-.] {}; \\
		\node [shapeEllipse,label=right:Ellipse-.] {}; \\
		\node [shapeTriangle,label=right:Triangle-.] {}; \\
		\node [shapeHexagon,label=right:Hexagon-.] {}; \\
	};
\end{tikzpicture}

\centering	
\begin{table}[H]\tiny
\caption{Gene/Protein/Enzymes Description and References}	
\begin{tabular}{r|p{3cm}|l}
\hline	
Gene/Protein & Description & Reference \\
\hline 
\hline 
\end{tabular}
\end{table}

\subsubsection{Equation System A}

\begin{align*} 
\tiny
\frac{d^{\alpha_1}X_1(t)}{dt^{\alpha_1}} = a_{11} *X_1(t - \tau_1) + \\
a_{12} *\frac{X_1(t)^{\beta_1}}{(1-X_1(t - \tau_1))^{\beta_1}} - a_{15}X_1(t) + \\
\epsilon_1(t) \\
\frac{d^{\alpha_1}X_2(t)}{dt^{\alpha_1}} = a_{21} *X_2(t - \tau_2) + \\
a_{22} *\frac{X_2(t)^{\beta_1}}{(1-X_2(t - \tau_2))^{\beta_1}} - a_{25}X_2(t) + \\
\epsilon_2(t) \\ \\
\frac{d^{\alpha_1}X_3(t)}{dt^{\alpha_1}} = a_{31} *X_3(t - \tau_3) + \\
a_{32} *\frac{X_3(t)^{\beta_1}}{(1-X_3(t - \tau_3))^{\beta_1}} - a_{35}X_3(t) + \\
\epsilon_3(t) \\ \\
\frac{d^{\alpha_1}X_4(t)}{dt^{\alpha_1}} = a_{41} *X_4(t - \tau_4) + \\
a_{42} *\frac{X_4(t)^{\beta_1}}{(1-X_4(t - \tau_4))^{\beta_1}} - a_{45}X_4(t) + \\
\epsilon_4(t) \\ \\
\frac{d^{\alpha_1}X_5(t)}{dt^{\alpha_1}} = a_{51} *X_5(t - \tau_5) + \\
a_{52} *\frac{X_5(t)^{\beta_1}}{(1-X_5(t - \tau_5))^{\beta_1}} - a_{55}X_5(t) + \\
\epsilon_5(t) \\
\end{align*}

\subsubsection{Parameter Table}

\begin{table}[h]\footnotesize
	\caption{Parameter Description and Value}
	\begin{tabular}{rllp{2cm}l}
		\hline	
		Parameter & Value & Interval & Description & Reference \\
		\hline 
		a11 & 0 & [0,1] & Equation 1 & \cite{key1}
		a12 & 0 & [0,1] & Equation 1 & \cite{key1}
		a13 & 0 & [0,1] & Equation 1 & \cite{key1}
		a14 & 0 & [0,1] & Equation 1 & \cite{key1}
		a15 & 0 & [0,1] & Equation 1 & \cite{key1}
		\hline
		a21 & 0 & [0,1] & Equation 2 & \cite{key1}
		a22 & 0 & [0,1] & Equation 2 & \cite{key1}
		a23 & 0 & [0,1] & Equation 2 & \cite{key1}
		a24 & 0 & [0,1] & Equation 2 & \cite{key1}
		a25 & 0 & [0,1] & Equation 2 & \cite{key1}
		\hline
		a31 & 0 & [0,1] & Equation 3 & \cite{key1}
		a32 & 0 & [0,1] & Equation 3 & \cite{key1}
		a33 & 0 & [0,1] & Equation 3 & \cite{key1}
		a34 & 0 & [0,1] & Equation 3 & \cite{key1}
		a35 & 0 & [0,1] & Equation 3 & \cite{key1}
		\hline
		a41 & 0 & [0,1] & Equation 4 & \cite{key1}
		a42 & 0 & [0,1] & Equation 4 & \cite{key1}
		a43 & 0 & [0,1] & Equation 4 & \cite{key1}
		a44 & 0 & [0,1] & Equation 4 & \cite{key1}
		a45 & 0 & [0,1] & Equation 4 & \cite{key1}
		\hline
		a51 & 0 & [0,1] & Equation 5 & \cite{key1}
		a52 & 0 & [0,1] & Equation 5 & \cite{key1}
		a53 & 0 & [0,1] & Equation 5 & \cite{key1}
		a54 & 0 & [0,1] & Equation 5 & \cite{key1}
		a55 & 0 & [0,1] & Equation 5 & \cite{key1}
		\hline
		$\tau_1$ & 1 & [1,2] & Equation 1 & \cite{key1}
		$\tau_2$ & 1 & [1,2] & Equation 2 & \cite{key1}
		$\tau_3$ & 1 & [1,2] & Equation 3 & \cite{key1}
		$\tau_4$ & 1 & [1,2] & Equation 4 & \cite{key1}
		$\tau_5$ & 1 & [1,2] & Equation 5 & \cite{key1}
		\hline
		$\alpha_1$ & 1 & (0,2] & Equation 1 & \cite{key1}
		$\alpha_2$ & 1 & (0,2] & Equation 2 & \cite{key1}
		$\alpha_3$ & 1 & (0,2] & Equation 3 & \cite{key1}
		$\alpha_4$ & 1 & (0,2] & Equation 4 & \cite{key1}
		$\alpha_5$ & 1 & (0,2] & Equation 5 & \cite{key1}
		\hline
		$\beta_1$ & 1 & (0,2] & Equation 1 & \cite{key1}
		$\beta_2$ & 1 & (0,2] & Equation 2 & \cite{key1}
		$\beta_3$ & 1 & (0,2] & Equation 3 & \cite{key1}
		$\beta_4$ & 1 & (0,2] & Equation 4 & \cite{key1}
		$\beta_5$ & 1 & (0,2] & Equation 5 & \cite{key1}
	\end{tabular}	
\end{table}



\subsubsection{Initial Conditions and Noise Distributions}


\section{Results}

\subsection{Tables}

\centering
\begin{table}[H]\footnotesize
	\caption{}
	\begin{tabular}{rp{1cm}p{2cm}p{3cm}p{1cm}}
		\hline
		ID & A & B & C & Reference \\
		\hline
		\hline
	\end{tabular}
\end{table}
\raggedright

\centering
\begin{table}[H]\footnotesize
	\caption{}
	\begin{tabular}{rp{1cm}p{2cm}p{3cm}p{1cm}}
		\hline
		ID & A & B & C & Reference \\
		\hline
		\hline
	\end{tabular}
\end{table}
\raggedright


\centering
\begin{table}[H]\footnotesize
	\caption{}
	\begin{tabular}{rp{1cm}p{2cm}p{3cm}p{1cm}}
		\hline
		ID & A & B & C & Reference \\
		\hline
		\hline
	\end{tabular}
\end{table}
\raggedright


\subsection{Figures}

\begin{figure}[H]
	\centering
	\begin{minipage}[b]{0.5\linewidth}
	%\includegraphics[scale=0.25]{Example_1_Figure_1.png}
	\end{minipage}\hfill
	\begin{minipage}[b]{0.5\linewidth}
	%\includegraphics[scale=0.25]{Example_1_Figure_2.png}
	\end{minipage}\hfill	
	\begin{minipage}[b]{0.5\linewidth}
	%\includegraphics[scale=0.25]{Example_1_Figure_3.png}
	\end{minipage}\hfill
	\begin{minipage}[b]{0.5\linewidth}
	%\includegraphics[scale=0.25]{Example_1_Figure_4.png}
	\end{minipage}\hfill
	\caption{1, 2, 3 and 4}
	\label{fig:Figure1}
\end{figure} 

\section{Conclusions and Topics for the Classroom}

\begin{table}[H]\centering
	\begin{tabular}{p{1cm}p{4cm}p{3cm}}
		Topic ID & Summary & Comments\\
		\hline
		\hline
	\end{tabular}
\end{table}

\section{R Application Programming Interfaces (APIs)}




\bibliographystyle{plain}
\begin{thebibliography}{00}

\subsection{Vitamin Absorption}

\bibitem[1]{key1}\href{}Vitamin A, Retinoic Acid and Tamibarotene, A Front Toward Its Advances: A Review
\bibitem[1]{key1}\href{}Vitamin A and retinoid signaling: genomic and nongenomic effects
\bibitem[1]{key1}\href{}Gene Expression regulation by retinoic acid
\bibitem[1]{key1}\href{}Mathematical Modelling of the Vitamin C Clock Reaction
\bibitem[1]{key1}\href{}Advanced Theoretical Applications of Mathematical Modeling and Compartmental Analysis in Vitamin A and ProVitamin A Carotenoid Research: Validation of Prediction Methods
\bibitem[1]{key1}\href{}A Mathematical Model Bone Remodeling Process: Effects of Vitamin D and Time Delay
\bibitem[1]{key1}\href{}A Mathematical Mdoel of Bone Remodeling Process: Effect of Vitamin D
\bibitem[1]{key1}\href{}A Mathematical Model Gives Insights into the Effects of Vitamin B-6 Deficiency on 1 Carbon and Glutatathione Metabolism
\bibitem[1]{key1}\href{}A stochastic chemical dynamic approach to correlate autoimmunity and optimal vitamin-D range
\bibitem[1]{key1}\href{}A Model for Establishing  Upper Levels of Intake for Nutrients and Related Substances

\subsection{Differential Equations}

\bibitem[1]{key1}A New Approach and Solution Technique to Solve Time Fractional Nonlinear Reaction-Diffusion Equations
\bibitem[1]{key1}Stability Analysis of Fractional-Order Nonlinear Systems with Delay
\bibitem[1]{key1}Application of the Multistep Generalized Differential Transform Method to Solve a Time-Fractional Enzyme Kinetics
\bibitem[1]{key1}Wavelet Methods for Solving Fractional Order Differential Equations
\bibitem[1]{key1}Numerical Methods for Pricing American Options with Time-Fractional PDE Models
\bibitem[1]{key1}Application of Multistep Generalized Differential Transform Method for the Solutions of the Fractional-Order Chua System
\bibitem[1]{key1}Numerical Solution of Some Types of Fractional Optimal Control Problems
\bibitem[1]{key1}An Efficient Series Solution for Fractional Differential Equations
\bibitem[1]{key1}Approximate Analytical Solution for Nonlinear System of Fractional Differential Equations by BPs Operational Matrices
\bibitem[1]{key1}Numerical Solution for Complex Systems of Fractional Order
\bibitem[1]{key1}Stability Analysis of Fractional-Order Nonlinear Systems with Delay
\bibitem[1]{key1}Numerical Study for Time Delay Multistrain Tuberculosis Model of Fractional Order
\bibitem[1]{key1}A Numerical Method for Solving Fractional Differential Equations by Using Neural Network
\bibitem[1]{key1}Numerical Studies for Fractional-Order Logistic Differential Equation with Two Different Delays
\bibitem[1]{key1}Numerical Modeling of Fractional-Order Biological Systems
\bibitem[1]{key1}Numerical Solution of Some Types of Fractional Optimal Control Problems
\bibitem[1]{key1}A Numerical Method for Delayed Fractional-Order Differential Equations
\bibitem[1]{key1} New Insights into the Fractional Order Diffusion Equation Using Entropy and Kurtosis
\bibitem[1]{key1} DELAY DIFFERENTIAL EQUATIONS IN SINGLE SPECIES DYNAMICS
\bibitem[1]{key1} An Improved Artificial Bee Colony Algorithm Based on Elite Strategy and Dimension Learning
\bibitem[1]{key1}Operators of Fractional Calculus and Their Applications
\bibitem[1]{key1} Modelling Physiological and Pharmacological Control on Cell Proliferation to Optimise Cancer Treatments
\bibitem[1]{key1}Press, W. H., S. A. Teukolsky, W. T Vetterling, and B. P. Flannery (2007). 
\newblock Numerical Recipes: The Art of Numerical Computing. 
\newblock Third Edition, Cambridge University Press, New York.

\subsection{Digestion I}


\subsection{Digestion II} 

\subsection{Sugar} 

\subsection{Insulin}


\end{thebibliography}
\end{document}
