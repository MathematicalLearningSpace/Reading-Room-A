%-------------------------------------------------------------------------%
%--------------------Classroom Lecture Model Series-----------------------%
%-------------------------------------------------------------------------%

\begin{document}
\twocolumn
\scriptsize
\begin{frontmatter}
		\title{}
		\author{\corref{cor1}\fnref{fn1}}
		\cortext[cor1]{Corresponding author}
		\address{The Mathematical Learning Space}
		\ead{http://mathlearningspace.weebly.com}	
\end{frontmatter}	

Introduction:
\begin{enumerate}
\item Objective 1:
\item Objective 2:
\item Objective 3:
\end{enumerate}
Conclusion:

Keywords: 
Vocabulary Words:

\section{Introduction}

\subsection{Plan of the Article}

\begin{document}
\twocolumn
\scriptsize
\begin{frontmatter}
		\title{}
		\author{\corref{cor1}\fnref{fn1}}
		\cortext[cor1]{Corresponding author}
		\address{The Mathematical Learning Space}
		\ead{http://mathlearningspace.weebly.com}	
\end{frontmatter}	

Introduction:
\begin{enumerate}
\item Objective 1:
\item Objective 2:
\item Objective 3:
\end{enumerate}
Conclusion:

Keywords:
Vocabulary Words:

\section{Introduction}

\subsection{Plan of the Article}

\section{Topic Review Signal Network:}

\begin{table}[H]
\tiny
\begin{tabular}{p{1cm}p{1cm}p{3cm}p{1cm}p{1cm}} 
EdgeID & Gene 1 & Gene 2 & Biological Function & Comment \\
\hline
\hline
\end{tabular}
\caption{Signal Network}
\label{tab:Table2}
\end{table}

\begin{center}
	\begin{tikzpicture}
	[->,>=stealth',shorten >=2pt,node distance=3cm,
	thick,main node/.style={circle,draw,scale=0.25,transform canvas={scale=0.75},font=\sffamily\Small\bfseries},
	blacknode/.style={shape=circle, draw=black, line width=2},
	bluenode/.style={shape=circle, draw=blue, line width=2},
	greennode/.style={shape=circle, draw=green, line width=2},
	rednode/.style={shape=circle, draw=red, line width=2}
	]
%-------Legend ------------------------------------------------------------	
	\matrix [draw,below left] at (current bounding box.south) {
		\node [state,label=right:State] {}; A. \\
		\node [shapeSquare,label=right:Square- B.] {}; \\
		\node [shapeEllipse,label=right:Ellipse-C.] {}; \\
		\node [shapeTriangle,label=right:Triangle-D.] {}; \\
		\node [shapeHexagon,label=right:Hexagon-E.] {}; \\
	};
	\end{tikzpicture}
\end{center}

\section{Mathematical Background: Topic Review}

\section{Results}

\subsection{Tables}

\subsection{Figures}

\begin{figure}[H]
	\centering
	\begin{minipage}[b]{0.5\linewidth}
	%\includegraphics[scale=0.25]{Example_1_Figure_1.png}
	\end{minipage}\hfill
	\begin{minipage}[b]{0.5\linewidth}
	%\includegraphics[scale=0.25]{Example_1_Figure_2.png}
	\end{minipage}\hfill	
	\begin{minipage}[b]{0.5\linewidth}
	%\includegraphics[scale=0.25]{Example_1_Figure_3.png}
	\end{minipage}\hfill
	\begin{minipage}[b]{0.5\linewidth}
	%\includegraphics[scale=0.25]{Example_1_Figure_4.png}
	\end{minipage}\hfill
	\caption{1, 2, 3 and 4}
	\label{fig:Figure1}
\end{figure} 



\section{Conclusions}


\section{R Application Programming Interfaces (APIs)}




\bibliographystyle{plain}
\begin{thebibliography}{00}

\end{thebibliography}
\end{document}

\section{Conclusions}


\section{R Application Programming Interfaces (APIs)}




\bibliographystyle{plain}
\begin{thebibliography}{00}

\end{thebibliography}
\end{document}
