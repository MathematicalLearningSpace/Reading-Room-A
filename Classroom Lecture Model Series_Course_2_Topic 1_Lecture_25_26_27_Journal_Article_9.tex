%-------------------------------------------------------------------------%
%--------------------Classroom Lecture Model Series-----------------------%
%-------------------------------------------------------------------------%


\begin{document}
\twocolumn
\scriptsize
\begin{frontmatter}
		\title{DRAFT Sample: A Mathematical Model of the Regulation of Intracellular Signaling Cascades}
		\author{\corref{cor1}\fnref{fn1}}
		\cortext[cor1]{Corresponding author}
		\address{The Mathematical Learning Space}
		\ead{http://mathlearningspace.weebly.com}	
\end{frontmatter}	

Introduction:
\begin{enumerate}
\item Objective 1:
\item Objective 2:
\item Objective 3:
\end{enumerate}
Conclusion:

Keywords:  Metabolism, Glucose/Insulin, Pancreatic Cancer, pancreatic islets
Vocabulary Words:

\section{Introduction}

\begin{table}[H]\centering
	\begin{tabular}{p{1cm}p{4cm}p{3cm}}
		Article ID & Summary & Comments\\
		\hline
		\hline
	\end{tabular}
\end{table}

\subsection{Plan of the Article}

\begin{enumerate}
\end{enumerate}

\section{Biology Review-Metabolism}

\centering	
\begin{table}[H]\tiny
	\caption{}	
	\begin{tabular}{rp{1cm}|p{4cm}|l}
		\hline	
		PubmedID & Topic & Description & Citation \\
		\hline 
		\hline 
	\end{tabular}
\end{table}

\section{Mathematical Background}

\subsection{Review Topic :}

\begin{table}[H]\centering
	\begin{tabular}{p{1cm}p{4cm}p{3cm}}
		Article ID & Summary & Comments\\
		\hline
		\hline
	\end{tabular}
\end{table}

\subsection{Mathematical Model}

\centering
\begin{table}[H]\footnotesize
	\caption{}
	\begin{tabular}{rp{1cm}p{2cm}p{3cm}p{1cm}}
		\hline
		ID & A & B & C & Reference \\
		\hline
		\hline
	\end{tabular}
\end{table}
\raggedright


\subsubsection{Metabolism} 

\centering
\begin{table}[H]\footnotesize
	\caption{}
	\begin{tabular}{rp{1cm}p{2cm}p{3cm}p{1cm}}
		\hline
		ID & A & B & C & Reference \\
		\hline
		\hline
	\end{tabular}
\end{table}
\raggedright

\subsubsection{Equation System A}

\begin{align*} 
\tiny
\frac{d^{\alpha_1}X_1(t)}{dt^{\alpha_1}} = a_{11} *X_1(t - \tau_1) + \\
a_{12} *\frac{X_1(t)^{\beta_1}}{(1-X_1(t - \tau_1))^{\beta_1}} - a_{15}X_1(t) + \\
\epsilon_1(t) \\
\frac{d^{\alpha_1}X_2(t)}{dt^{\alpha_1}} = a_{21} *X_2(t - \tau_2) + \\
a_{22} *\frac{X_2(t)^{\beta_1}}{(1-X_2(t - \tau_2))^{\beta_1}} - a_{25}X_2(t) + \\
\epsilon_2(t) \\ \\
\frac{d^{\alpha_1}X_3(t)}{dt^{\alpha_1}} = a_{31} *X_3(t - \tau_3) + \\
a_{32} *\frac{X_3(t)^{\beta_1}}{(1-X_3(t - \tau_3))^{\beta_1}} - a_{35}X_3(t) + \\
\epsilon_3(t) \\ \\
\frac{d^{\alpha_1}X_4(t)}{dt^{\alpha_1}} = a_{41} *X_4(t - \tau_4) + \\
a_{42} *\frac{X_4(t)^{\beta_1}}{(1-X_4(t - \tau_4))^{\beta_1}} - a_{45}X_4(t) + \\
\epsilon_4(t) \\ \\
\frac{d^{\alpha_1}X_5(t)}{dt^{\alpha_1}} = a_{51} *X_5(t - \tau_5) + \\
a_{52} *\frac{X_5(t)^{\beta_1}}{(1-X_5(t - \tau_5))^{\beta_1}} - a_{55}X_5(t) + \\
\epsilon_5(t) \\
\end{align*}


\subsubsection{Obesity}

\centering
\begin{table}[H]\footnotesize
	\caption{}
	\begin{tabular}{rp{1cm}p{2cm}p{3cm}p{1cm}}
		\hline
		ID & A & B & C & Reference \\
		\hline
		\hline
	\end{tabular}
\end{table}
\raggedright

\subsubsection{Equation System A}

\begin{align*} 
\tiny
\frac{d^{\alpha_1}X_1(t)}{dt^{\alpha_1}} = a_{11} *X_1(t - \tau_1) + \\
a_{12} *\frac{X_1(t)^{\beta_1}}{(1-X_1(t - \tau_1))^{\beta_1}} - a_{15}X_1(t) + \\
\epsilon_1(t) \\
\frac{d^{\alpha_1}X_2(t)}{dt^{\alpha_1}} = a_{21} *X_2(t - \tau_2) + \\
a_{22} *\frac{X_2(t)^{\beta_1}}{(1-X_2(t - \tau_2))^{\beta_1}} - a_{25}X_2(t) + \\
\epsilon_2(t) \\ \\
\frac{d^{\alpha_1}X_3(t)}{dt^{\alpha_1}} = a_{31} *X_3(t - \tau_3) + \\
a_{32} *\frac{X_3(t)^{\beta_1}}{(1-X_3(t - \tau_3))^{\beta_1}} - a_{35}X_3(t) + \\
\epsilon_3(t) \\ \\
\frac{d^{\alpha_1}X_4(t)}{dt^{\alpha_1}} = a_{41} *X_4(t - \tau_4) + \\
a_{42} *\frac{X_4(t)^{\beta_1}}{(1-X_4(t - \tau_4))^{\beta_1}} - a_{45}X_4(t) + \\
\epsilon_4(t) \\ \\
\frac{d^{\alpha_1}X_5(t)}{dt^{\alpha_1}} = a_{51} *X_5(t - \tau_5) + \\
a_{52} *\frac{X_5(t)^{\beta_1}}{(1-X_5(t - \tau_5))^{\beta_1}} - a_{55}X_5(t) + \\
\epsilon_5(t) \\
\end{align*}


\subsubsection{Pancreatic Signalling I}

\centering
\begin{table}[H]\footnotesize
	\caption{}
	\begin{tabular}{rp{1cm}p{2cm}p{3cm}p{1cm}}
		\hline
		ID & A & B & C & Reference \\
		\hline
		\hline
	\end{tabular}
\end{table}
\raggedright

\subsubsection{Equation System A}

\begin{align*} 
\tiny
\frac{d^{\alpha_1}X_1(t)}{dt^{\alpha_1}} = a_{11} *X_1(t - \tau_1) + \\
a_{12} *\frac{X_1(t)^{\beta_1}}{(1-X_1(t - \tau_1))^{\beta_1}} - a_{15}X_1(t) + \\
\epsilon_1(t) \\
\frac{d^{\alpha_1}X_2(t)}{dt^{\alpha_1}} = a_{21} *X_2(t - \tau_2) + \\
a_{22} *\frac{X_2(t)^{\beta_1}}{(1-X_2(t - \tau_2))^{\beta_1}} - a_{25}X_2(t) + \\
\epsilon_2(t) \\ \\
\frac{d^{\alpha_1}X_3(t)}{dt^{\alpha_1}} = a_{31} *X_3(t - \tau_3) + \\
a_{32} *\frac{X_3(t)^{\beta_1}}{(1-X_3(t - \tau_3))^{\beta_1}} - a_{35}X_3(t) + \\
\epsilon_3(t) \\ \\
\frac{d^{\alpha_1}X_4(t)}{dt^{\alpha_1}} = a_{41} *X_4(t - \tau_4) + \\
a_{42} *\frac{X_4(t)^{\beta_1}}{(1-X_4(t - \tau_4))^{\beta_1}} - a_{45}X_4(t) + \\
\epsilon_4(t) \\ \\
\frac{d^{\alpha_1}X_5(t)}{dt^{\alpha_1}} = a_{51} *X_5(t - \tau_5) + \\
a_{52} *\frac{X_5(t)^{\beta_1}}{(1-X_5(t - \tau_5))^{\beta_1}} - a_{55}X_5(t) + \\
\epsilon_5(t) \\
\end{align*}


\subsubsection{Initial Conditions and Noise Distributions}

\subsubsection{Metabolism} 

\subsubsection{Parameter Matrix}
\vspace{4pt}
\centering
\begin{table}[h]\footnotesize
	\caption{Parameter Description and Value}
	\begin{tabular}{rllp{2cm}l}
		\hline	
		Parameter & Value & Interval & Description & Reference \\
		\hline 
	\end{tabular}	
\end{table}

\subsubsection{Obesity}

\subsubsection{Parameter Matrix}
\vspace{4pt}
\centering
\begin{table}[h]\footnotesize
	\caption{Parameter Description and Value}
	\begin{tabular}{rllp{2cm}l}
		\hline	
		Parameter & Value & Interval & Description & Reference \\
		\hline 
	\end{tabular}	
\end{table}

\subsubsection{Pancreatic Signalling I}

\subsubsection{Parameter Matrix}
\vspace{4pt}
\centering
\begin{table}[h]\footnotesize
	\caption{Parameter Description and Value}
	\begin{tabular}{rllp{2cm}l}
		\hline	
		Parameter & Value & Interval & Description & Reference \\
		\hline 
	\end{tabular}	
\end{table}


\section{Results}

\centering
\begin{table}[H]\footnotesize
	\caption{}
	\begin{tabular}{rp{1cm}p{2cm}p{3cm}p{1cm}}
		\hline
		ID & A & B & C & Reference \\
		\hline
		\hline
	\end{tabular}
\end{table}
\raggedright


\subsection{Tables}

\centering	
\begin{table}[H]\tiny
	\caption{}	
	\begin{tabular}{rp{1cm}|p{4cm}|l}
		\hline	
		Model ID & Description & Results & Comment \\
		\hline 
		\hline 
	\end{tabular}
\end{table}

\subsection{Figures}

\begin{figure}[H]
	\centering
	\begin{minipage}[b]{0.5\linewidth}
	%\includegraphics[scale=0.25]{Example_1_Figure_1.png}
	\end{minipage}\hfill
	\begin{minipage}[b]{0.5\linewidth}
	%\includegraphics[scale=0.25]{Example_1_Figure_2.png}
	\end{minipage}\hfill	
	\begin{minipage}[b]{0.5\linewidth}
	%\includegraphics[scale=0.25]{Example_1_Figure_3.png}
	\end{minipage}\hfill
	\begin{minipage}[b]{0.5\linewidth}
	%\includegraphics[scale=0.25]{Example_1_Figure_4.png}
	\end{minipage}\hfill
	\caption{1, 2, 3 and 4}
	\label{fig:Figure1}
\end{figure} 


\section{Conclusions}

\begin{table}[H]\centering
	\begin{tabular}{p{1cm}p{4cm}p{3cm}}
		Article ID & Summary & Comments\\
		\hline
		\hline
	\end{tabular}
\end{table}


\section{R Application Programming Interfaces (APIs)}

\bibliographystyle{plain}
\begin{thebibliography}{00}

\bibitem[1]{key1}Wikipedia contributors. 
\newblock Molecular machine. 
\newblock Wikipedia, The Free Encyclopedia. Wikipedia The Free Encyclopedia

\bibitem[2]{key2}Cao D, Xiao N, Xu Q, Chen AF (2015). 
\newblock Rcpi: R/Bioconductor package to generate various descriptors of proteins, compounds and their interactions. 
\newblock Bioinformatics, 31(2), 279–281. doi: 10.1093/bioinformatics/btu624. 

\bibtem[3]{key3}Sunghwan Kim, Paul A. Thiessen, Evan E. Bolton, Jie Chen, Gang Fu, Asta Gindulyte, Lianyi Han, Jane He, Siqian He, Benjamin A. Shoemaker, Jiyao Wang, Bo Yu, Jian Zhang, Stephen H. Bryant; 
\newblock PubChem Substance and Compound databases, 
\newblock Nucleic Acids Research, Volume 44, Issue D1, 4 January 2016, Pages D1202–D1213, https://doi.org/10.1093/nar/gkv951

\bibitem[4]{key4}Wenrui Hao  Hannah M. Komarb, Phil A. Hart, Darwin L. Conwell, Gregory B. Lesinskid, and Avner Friedman
\newblock Mathematical model of chronic pancreatitis

\bibitem[5]{key5}Daniel  Auverkamp
\newblock Mathematical  Models  for  Pancreatic  Cancer  and Stellate  Cells

\bibitem[300]{key300}Jung Min Han, Vipul Periwal
\newblock A mathematical model of calcium dynamics: Obesity and mitochondria-associated ER membranes

\bibitem[1000]{key1000}R Core Team (2015). 
\newblock R: A language and environment for statistical computing. R Foundation for Statistical Computing, Vienna, Austria.
\newblock URL https://www.R-project.org/.

\subsection{Genomics}

\bibitem[400]{key400} Kanehisa, Furumichi, M., Tanabe, M., Sato, Y., and Morishima, K.; 
\newblock KEGG: new perspectives on genomes, pathways, diseases and drugs. 
\newblock Nucleic Acids Res. 45, D353-D361 (2017).

\bibitem[401]{key401} Kanehisa, M., Sato, Y., Kawashima, M., Furumichi, M., and Tanabe, M.; 
\newblock KEGG as a reference resource for gene and protein annotation. 
\newblock Nucleic Acids Res. 44, D457-D462 (2016).

\bibitem[402]{key402} Kanehisa, M. and Goto, S.; 
\newblock KEGG: Kyoto Encyclopedia of Genes and Genomes. 
\newblock Nucleic Acids Res. 28, 27-30 (2000). 

\bibitem[403]{key403} Rouillard AD, Gundersen GW, Fernandez NF, Wang Z, Monteiro CD, McDermott MG, Ma'ayan A. 
\newblock The harmonizome: a collection of processed datasets gathered to serve and mine knowledge about genes and proteins. 
\newblock Database (Oxford). 2016 Jul 3;2016. pii: baw100. 

\subsection{Metabolism} 
\subsection{Obesity} 
\subsection{Pancreatic Signalling I}


\end{thebibliography}
\end{document}
