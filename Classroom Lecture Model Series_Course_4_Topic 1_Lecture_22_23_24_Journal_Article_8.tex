%-------------------------------------------------------------------------%
%--------------------Classroom Lecture Model Series-----------------------%
%-------------------------------------------------------------------------%
\begin{document}
\twocolumn
\scriptsize
\begin{frontmatter}
		\title{}
		\author{\corref{cor1}\fnref{fn1}}
		\cortext[cor1]{Corresponding author}
		\address{The Mathematical Learning Space}
		\ead{http://mathlearningspace.weebly.com}	
\end{frontmatter}	

Introduction:
\begin{enumerate}
\item Objective 1:
\item Objective 2:
\item Objective 3:
\end{enumerate}
Conclusion:

keywords: Markov Models, Semi-Markov Models, Music Machine Learning, Classical Music, Smooth Jazz Music

\section{Introduction}

\subsection{Plan of the Article}

\begin{enumerate}
\item Review of Music Theory Perspectives of Classical and Smooth Jazz Music Compositions
\item Review of Graphical Models for Adjacency Matrix Representations for Feature Relationships
\item Results of Markov Models Specification and Evaluation for both Samples of Classical and Smooth Jazz Compositions of $10^N$ measures
\item Review of the Properties from the Models
\item Review of Additional Topics for the Classroom
\end{enumerate}

\section{Music Theory Review}

\subsection{Topic A: Classical Music Compositions}

\centering	
\begin{table}[H]\tiny
	\caption{}	
	\begin{tabular}{r|p{4cm}|l}
		\hline	
		Topic & Description & Reference \\
		\hline 
		\hline 
	\end{tabular}
\end{table}

\subsection{Topic B: Smooth Jazz Music Compositions}

\centering	
\begin{table}[H]\tiny
	\caption{}	
	\begin{tabular}{r|p{4cm}|l}
		\hline	
		Topic & Description & Reference \\
		\hline 
		\hline 
	\end{tabular}
\end{table}

\section{Mathmatical Background}

\subsection{Markov Models}

\subsection{Semi-Markov Models}


\section{Results}

\subsection{Data}

Table 1 provides the themes and instrument collection for each of the compositions.
	
\begin{table}[H]
\caption{Composition Collection}	
\begin{tabular}{p{1cm}p{4cm}p{2cm}p{1cm}p{1cm}}
\hline
ID & Name & Theme Pattern & Instruments & \\
\hline 
1 & Composition I &  &  & \\
2 & Composition II &  &  & \\
3 & Composition III &  & \\
4 & Composition IV & & \\
5 & Composition V & & & \\
\hline 
6 & Composition VI &  &  & \\
7 & Composition VII &  &  & \\
8 & Composition VIII &  & \\
9 & Composition IX & & \\
10 & Composition X & & & \\
\end{tabular}
\end{table}

\subsection{Feature Design Matrix}

\subsubsection{Power Spectrum}

\subsubsection{Acoustic Indicies}

\subsubsection{Entropy}

\centering	
\begin{table}[H]\tiny
	\caption{}	
	\begin{tabular}{p{1cm}p{1cm}p{1cm}p{1cm}p{1cm}}
		\hline	
		Model & F1 & F2 & F3 & F4 \\
		\hline 
		\hline 
	\end{tabular}
\end{table}

\subsection{Tables}

\centering	
\begin{table}[H]\tiny
	\caption{}	
	\begin{tabular}{r|p{4cm}|l}
		\hline	
		Model & Description & Results \\
		\hline 
		\hline 
	\end{tabular}
\end{table}

\subsection{Figures}

\begin{figure}[H]
	\centering
	\begin{minipage}[b]{0.5\linewidth}
	%\includegraphics[scale=0.25]{Example_1_Figure_1.png}
	\end{minipage}\hfill
	\begin{minipage}[b]{0.5\linewidth}
	%\includegraphics[scale=0.25]{Example_1_Figure_2.png}
	\end{minipage}\hfill	
	\begin{minipage}[b]{0.5\linewidth}
	%\includegraphics[scale=0.25]{Example_1_Figure_3.png}
	\end{minipage}\hfill
	\begin{minipage}[b]{0.5\linewidth}
	%\includegraphics[scale=0.25]{Example_1_Figure_4.png}
	\end{minipage}\hfill
	\caption{1, 2, 3 and 4}
	\label{fig:Figure1}
\end{figure} 

\subsubsection{Graph Models}

\centering	
\begin{table}[H]\tiny
	\caption{}	
	\begin{tabular}{r|p{4cm}|l}
		\hline	
		Model & Description & Results \\
		\hline 
		\hline 
	\end{tabular}
\end{table}

\begin{figure}[H]
	\centering
	\begin{minipage}[b]{0.5\linewidth}
	%\includegraphics[scale=0.25]{Example_2_Figure_1.png}
	\end{minipage}\hfill
	\begin{minipage}[b]{0.5\linewidth}
	%\includegraphics[scale=0.25]{Example_2_Figure_2.png}
	\end{minipage}\hfill	
	\begin{minipage}[b]{0.5\linewidth}
	%\includegraphics[scale=0.25]{Example_2_Figure_3.png}
	\end{minipage}\hfill
	\begin{minipage}[b]{0.5\linewidth}
	%\includegraphics[scale=0.25]{Example_2_Figure_4.png}
	\end{minipage}\hfill
	\caption{1, 2, 3 and 4}
	\label{fig:Figure1}
\end{figure} 

\section{Additional Topics for the Classroom}

\begin{enumerate}
\end{enumerate}


\bibliographystyle{plain}
\begin{thebibliography}{00}
\bibitem[1]{key100}Sueur J., Aubin T., Simonis C. (2008). 
\newblock Seewave: a free modular tool for sound analysis and synthesis. 
\newblock Bioacoustics, 18: 213-226.

\bibitem[2]{key101}Uwe Ligges, Sebastian Krey, Olaf Mersmann, and Sarah Schnackenberg (2016). 
\newblock tuneR: Analysis of music. 
\newblock URL: http://r-forge.r-project.org/projects/tuner/.

\bibitem[3]{key102}A. Anikin (2017). 
\newblock soundgen: Parametric Voice Synthesis. 
\newblock R package version 1.1.1.

\bibitem[4]{key103}Pieretti N, Farina A, Morri FD (2011) 
\newblock A new methodology to infer the singing activity of an avian community: the Acoustic Complexity Index (ACI). 
\newblock Ecological Indicators, 11, 868-873.
\end{thebibliography}

\end{document}
