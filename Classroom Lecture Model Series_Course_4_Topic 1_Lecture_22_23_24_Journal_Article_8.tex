%-------------------------------------------------------------------------%
%--------------------Classroom Lecture Model Series-----------------------%
%-------------------------------------------------------------------------%
\begin{document}
\twocolumn
\scriptsize
\begin{frontmatter}
		\title{}
		\author{\corref{cor1}\fnref{fn1}}
		\cortext[cor1]{Corresponding author}
		\address{The Mathematical Learning Space}
		\ead{http://mathlearningspace.weebly.com}	
\end{frontmatter}	

Introduction:
\begin{enumerate}
\item Objective 1:
\item Objective 2:
\item Objective 3:
\end{enumerate}
Conclusion:

keywords: Markov Models, Semi-Markov Models, Music Machine Learning, Classical Music, Smooth Jazz Music

\section{Introduction}

\subsection{Plan of the Article}

\begin{enumerate}
\item Review of Music Theory Perspectives of Classical and Smooth Jazz Music Compositions
\item Review of Graphical Models for Adjacency Matrix Representations for Feature Relationships
\item Results of Markov Models Specification and Evaluation for both Samples of Classical and Smooth Jazz Compositions of $10^N$ measures
\item Review of the Properties from the Models
\item Review of Additional Topics for the Classroom
\end{enumerate}

\section{Music Theory Review}

\subsection{Topic A: Classical Music Compositions}

\subsection{Topic B: Smooth Jazz Music Compositions}

\section{Mathmatical Background}

\subsection{Markov Models}

\subsection{Semi-Markov Models}


\section{Results}

\subsection{Feature Design Matrix}

\subsection{Tables}

\subsection{Figures}

\subsubsection{Graph Models}

\section{Additional Topics for the Classroom}

\begin{enumerate}
\end{enumerate}


\bibliographystyle{plain}
\begin{thebibliography}{00}

\end{thebibliography}

\end{document}
