%-------------------------------------------------------------------------%
%--------------------Classroom Lecture Model Series-----------------------%
%-------------------------------------------------------------------------%

%--------------------Work in Progress for the Classroom-------------------%
\begin{document}
\twocolumn
\scriptsize
\begin{frontmatter}
		\title{}
		\author{\corref{cor1}\fnref{fn1}}
		\cortext[cor1]{Corresponding author}
		\address{The Mathematical Learning Space}
		\ead{http://mathlearningspace.weebly.com}	
\end{frontmatter}	

Introduction:
\begin{enumerate}
\item Objective 1:
\item Objective 2:
\item Objective 3:
\end{enumerate}
Conclusion:

Keywords: Cell Cycle
Vocabulary Words:

\section{Introduction}

\begin{table}[H]\centering
	\begin{tabular}{p{1cm}p{4cm}p{3cm}}
		Article ID & Summary & Comments\\
		\hline
		\hline
	\end{tabular}
\end{table}

\subsection{Plan of the Article}

\begin{enumerate}
\end{enumerate}

\section{Topic Review Signal Network:}

\centering	
\begin{table}[H]\tiny
	\caption{}	
	\begin{tabular}{rp{1cm}|p{4cm}|l}
		\hline	
		PubmedID & Topic & Description & Citation \\
		\hline 
		\hline 
	\end{tabular}
\end{table}


\begin{table}[H]
\tiny
\begin{tabular}{p{1cm}p{1cm}p{3cm}p{1cm}p{1cm}} 
EdgeID & Gene 1 & Gene 2 & Biological Function & Comment \\
\hline
\hline
\end{tabular}
\caption{Signal Network}
\label{tab:Table2}
\end{table}

\begin{center}
	\begin{tikzpicture}
	[->,>=stealth',shorten >=2pt,node distance=3cm,
	thick,main node/.style={circle,draw,scale=0.25,transform canvas={scale=0.75},font=\sffamily\Small\bfseries},
	blacknode/.style={shape=circle, draw=black, line width=2},
	bluenode/.style={shape=circle, draw=blue, line width=2},
	greennode/.style={shape=circle, draw=green, line width=2},
	rednode/.style={shape=circle, draw=red, line width=2}
	]
%-------Legend ------------------------------------------------------------	
	\matrix [draw,below left] at (current bounding box.south) {
		\node [state,label=right:State] {}; A. \\
		\node [shapeSquare,label=right:Square- B.] {}; \\
		\node [shapeEllipse,label=right:Ellipse-C.] {}; \\
		\node [shapeTriangle,label=right:Triangle-D.] {}; \\
		\node [shapeHexagon,label=right:Hexagon-E.] {}; \\
	};
	\end{tikzpicture}
\end{center}

\section{Mathematical Background}

\subsection{Mathematical Model}

\subsubsection{Fox}

\begin{equation}
\end{equation}

\subsubsection{Cell cycle} 

\begin{equation}
\end{equation}

\subsubsection{mTOR}

\begin{equation}
\end{equation}

\subsubsection{Initial Conditions and Noise Distributions}

\centering	
\begin{table}[H]\tiny
	\caption{}	
	\begin{tabular}{rp{1cm}|p{4cm}|l}
		\hline	
		Noise ID & Description & Properties & Results \\
		\hline 
		\hline 
	\end{tabular}
\end{table}

\subsubsection{Fox}

\subsubsection{Parameter Matrix}
\vspace{4pt}
\centering
\begin{table}[h]\footnotesize
	\caption{Parameter Description and Value}
	\begin{tabular}{rllp{2cm}l}
		\hline	
		Parameter & Value & Interval & Description & Reference \\
		\hline 
	\end{tabular}	
\end{table}

\subsubsection{Cell cycle} 

\centering	
\begin{table}[H]\tiny
	\caption{}	
	\begin{tabular}{rp{1cm}|p{4cm}|l}
		\hline	
		Model ID & Description & Properties & Results \\
		\hline 
		\hline 
	\end{tabular}
\end{table}

\subsubsection{mTOR}

\section{Conclusions and Topics for Discussion}

\begin{table}[H]\centering
	\begin{tabular}{p{1cm}p{4cm}p{3cm}}
		Article ID & Summary & Comments\\
		\hline
		\hline
	\end{tabular}
\end{table}

\section{R Application Programming Interfaces (APIs)}




\bibliographystyle{plain}
\begin{thebibliography}{00}
\bibitem[1]{key1} Kevin J. Kesselera, Michael L. Blinovb, Timothy C. Elstonc, William K. Kaufmanna, and Dennis A. Simpsona,
\newblock A predictive mathematical model of the DNA damage G2 checkpoint
\newblock J Theor Biol. 2013 March 7; 320: . doi:10.1016/j.jtbi.2012.12.011.

\bibitem[2]{key2}Li H1, Zhang XP, Liu F. (2013). 
\newblock Bifurcation in Cell Cycle Dynamics Regulated by P53

\bibitem[3]{key3}Li H1, Zhang XP, Liu F. (2013). 
\newblock Coordination between p21 and DDB2 in the cellular response to UV radiation 
\newblock PLoS One. https://www.ncbi.nlm.nih.gov/pubmed/24260342

\bibitem[4]{key4} D.A.J. van Zwieten1, J.E. Rooda1,D.Armbruster and J.D. Nagy
\newblock Simulating feedback and reversibility in substrate-enzyme reactions
\newblock Eur. Phys. J. B 84, 673–684 (2011) DOI: 10.1140/epjb/e2011-10911-x

\bibitem[5]{key5} Pilwon Kima and Chang Hyeong Leeb
\newblock A probability generating function method for stochastic reaction networks
\newblock THE JOURNAL OF CHEMICAL PHYSICS 136, 234108 (2012)

\bibitem[6]{key6}Todd L Parsons1 and Tim Rogers
\newblock Dimension reduction for stochastic dynamical systems forced onto a manifold by large drift: a constructive approach with examples from theoretical biology
\newblock J. Phys. A: Math. Theor. 50 (2017) https://doi.org/10.1088/1751-8121/aa86c7

\bibitem[400]{key400} Kanehisa, Furumichi, M., Tanabe, M., Sato, Y., and Morishima, K.; 
\newblock KEGG: new perspectives on genomes, pathways, diseases and drugs. 
\newblock Nucleic Acids Res. 45, D353-D361 (2017).

\bibitem[401]{key401} Kanehisa, M., Sato, Y., Kawashima, M., Furumichi, M., and Tanabe, M.; 
\newblock KEGG as a reference resource for gene and protein annotation. 
\newblock Nucleic Acids Res. 44, D457-D462 (2016).

\bibitem[402]{key402} Kanehisa, M. and Goto, S.; 
\newblock KEGG: Kyoto Encyclopedia of Genes and Genomes. 
\newblock Nucleic Acids Res. 28, 27-30 (2000). 

\bibitem[403]{key403} Rouillard AD, Gundersen GW, Fernandez NF, Wang Z, Monteiro CD, McDermott MG, Ma'ayan A. 
\newblock The harmonizome: a collection of processed datasets gathered to serve and mine knowledge about genes and proteins. 
\newblock Database (Oxford). 2016 Jul 3;2016. pii: baw100. 

\subsection{Fox} 
\subsection{Cell cycle} 
\subsection{mTOR}

\end{thebibliography}
\end{document}
