%-------------------------------------------------------------------------%
%--------------------Classroom Lecture Model Series-----------------------%
%-------------------------------------------------------------------------%

\begin{document}
\twocolumn
\scriptsize
\begin{frontmatter}
		\title{}
		\author{\corref{cor1}\fnref{fn1}}
		\cortext[cor1]{Corresponding author}
		\address{The Mathematical Learning Space}
		\ead{http://mathlearningspace.weebly.com}	
\end{frontmatter}	

Introduction:
\begin{enumerate}
\item Objective 1:
\item Objective 2:
\item Objective 3:
\end{enumerate}
Conclusion:

Keywords:  Molecular Machine Learning, Shikimate Pathway, Mathematical chemistry, QSAR, Topological index,Docking (molecular),
Vocabulary Words:

\section{Introduction}


\subsection{Plan of the Article}

\section{Topic Review: Mathenmatical Chemistry}


\section{Conclusions}


\section{R Application Programming Interfaces (APIs)}




\bibliographystyle{plain}
\begin{thebibliography}{00}

\bibitem[1]{key1}N. Trinajstić, I. Gutman, 
\newblock Mathematical Chemistry, 
\newblock Croatica Chemica Acta, 75(2002), pp. 329–356.

\bibitem[2]{key2}A. T. Balaban, 
\newblock Reflections about Mathematical Chemistry, 
\newblock Foundations of Chemistry, 7(2005), pp. 289–306.

\bibitem[3]{key3}G. Restrepo, J. L. Villaveces, 
\newblock Mathematical Thinking in Chemistry, HYLE, 18(2012), pp. 3–22.
\newblock Advances in Mathematical Chemistry and Applications. Volume 2. Basak S. C., Restrepo G., Villaveces J. L. (Bentham Science eBooks, 2015)

\end{thebibliography}
\end{document}
