%-------------------------------------------------------------------------%
%--------------------Classroom Lecture Model Series-----------------------%
%-------------------------------------------------------------------------%

%---------------------Work in Progress for the Classrooom-----------------%

\begin{document}
\twocolumn
\scriptsize
\begin{frontmatter}
		\title{DRAFT Sample: A Mathematical Model of Mitotic Biological Processes in a Gene Ontology}
		\author{\corref{cor1}\fnref{fn1}}
		\cortext[cor1]{Corresponding author}
		\address{The Mathematical Learning Space}
		\ead{http://mathlearningspace.weebly.com}	
\end{frontmatter}	

Introduction:
\begin{enumerate}
\item Objective 1:
\item Objective 2:
\item Objective 3:
\end{enumerate}
Conclusion:

Keywords:  Molecular Machine Learning, Shikimate Pathway, Mathematical chemistry, QSAR, Topological index,Docking (molecular), secondary  metabolites
Vocabulary Words: phenylpropanoids, phytochemicals

\section{Introduction}

\begin{table}[H]\centering
	\begin{tabular}{p{1cm}p{4cm}p{3cm}}
		Article ID & Summary & Comments\\
		\hline
		\hline
	\end{tabular}
\end{table}


\subsection{Plan of the Article}

\begin{enumerate}
\end{enumerate}

\section{Topic Review: Mathematical Chemistry}

\centering	
\begin{table}[H]\tiny
	\caption{}	
	\begin{tabular}{p{1cm}p{1cm}|p{4cm}|l}
		\hline	
		PubMed ID & Topic & Description & Results \\
		\hline 
		\hline 
	\end{tabular}
\end{table}

\subsection{Phenylpropanoid Pathway}

\begin{table}[H]\centering
	\begin{tabular}{p{1cm}p{4cm}p{3cm}}
		Article ID & Summary & Comments\\
		\hline
		\hline
	\end{tabular}
\end{table}

\subsection{Plant BioChemistry and Phytochemistry Background}
\centering	
\begin{table}[H]\tiny
	\caption{}	
	\begin{tabular}{p{1cm}p{1cm}|p{4cm}|l}
		\hline	
		PubChem ID & Topic & Description & Results \\
		\hline 
		\hline 
	\end{tabular}
\end{table}

\section{Mathematical Background}

\subsection{Equation System A}

\begin{align*} 
\tiny
\frac{d^{\alpha_1}X_1(t)}{dt^{\alpha_1}} = a_{11} *X_1(t - \tau_1) + \\
a_{12} *\frac{X_1(t)^{\beta_1}}{(1-X_1(t - \tau_1))^{\beta_1}} - a_{15}X_1(t) + \\
\epsilon_1(t) \\
\frac{d^{\alpha_1}X_2(t)}{dt^{\alpha_1}} = a_{21} *X_2(t - \tau_2) + \\
a_{22} *\frac{X_2(t)^{\beta_1}}{(1-X_2(t - \tau_2))^{\beta_1}} - a_{25}X_2(t) + \\
\epsilon_2(t) \\ \\
\frac{d^{\alpha_1}X_3(t)}{dt^{\alpha_1}} = a_{31} *X_3(t - \tau_3) + \\
a_{32} *\frac{X_3(t)^{\beta_1}}{(1-X_3(t - \tau_3))^{\beta_1}} - a_{35}X_3(t) + \\
\epsilon_3(t) \\ \\
\frac{d^{\alpha_1}X_4(t)}{dt^{\alpha_1}} = a_{41} *X_4(t - \tau_4) + \\
a_{42} *\frac{X_4(t)^{\beta_1}}{(1-X_4(t - \tau_4))^{\beta_1}} - a_{45}X_4(t) + \\
\epsilon_4(t) \\ \\
\frac{d^{\alpha_1}X_5(t)}{dt^{\alpha_1}} = a_{51} *X_5(t - \tau_5) + \\
a_{52} *\frac{X_5(t)^{\beta_1}}{(1-X_5(t - \tau_5))^{\beta_1}} - a_{55}X_5(t) + \\
\epsilon_5(t) \\
\end{align*}

\centering
\begin{table}[H]\footnotesize
	\caption{}
	\begin{tabular}{rp{1cm}p{2cm}p{3cm}p{1cm}}
		\hline
		ID & A & B & C & Reference \\
		\hline
		\hline
	\end{tabular}
\end{table}
\raggedright


\section{Results}



\subsection{Tables}
\centering
\begin{table}[H]\footnotesize
	\caption{}
	\begin{tabular}{rp{1cm}p{2cm}p{3cm}p{1cm}}
		\hline
		ID & A & B & C & Reference \\
		\hline
		\hline
	\end{tabular}
\end{table}
\raggedright

\centering
\begin{table}[H]\footnotesize
	\caption{}
	\begin{tabular}{rp{1cm}p{2cm}p{3cm}p{1cm}}
		\hline
		ID & A & B & C & Reference \\
		\hline
		\hline
	\end{tabular}
\end{table}
\raggedright


\subsection{Figures}


\begin{figure}[H]
	\centering
	\begin{minipage}[b]{0.5\linewidth}
	%\includegraphics[scale=0.25]{Example_1_Figure_1.png}
	\end{minipage}\hfill
	\begin{minipage}[b]{0.5\linewidth}
	%\includegraphics[scale=0.25]{Example_1_Figure_2.png}
	\end{minipage}\hfill	
	\begin{minipage}[b]{0.5\linewidth}
	%\includegraphics[scale=0.25]{Example_1_Figure_3.png}
	\end{minipage}\hfill
	\begin{minipage}[b]{0.5\linewidth}
	%\includegraphics[scale=0.25]{Example_1_Figure_4.png}
	\end{minipage}\hfill
	\caption{1, 2, 3 and 4}
	\label{fig:Figure1}
\end{figure} 



\section{Conclusions and Discussion Topics}

\begin{table}[H]\centering
	\begin{tabular}{p{1cm}p{4cm}p{3cm}}
		Topic ID & Summary & Comments\\
		\hline
		\hline
	\end{tabular}
\end{table}




\section{R Application Programming Interfaces (APIs)}




\bibliographystyle{plain}
\begin{thebibliography}{00}

\bibitem[1]{key1}N. Trinajstić, I. Gutman, 
\newblock Mathematical Chemistry, 
\newblock Croatica Chemica Acta, 75(2002), pp. 329–356.

\bibitem[2]{key2}A. T. Balaban, 
\newblock Reflections about Mathematical Chemistry, 
\newblock Foundations of Chemistry, 7(2005), pp. 289–306.

\bibitem[3]{key3}G. Restrepo, J. L. Villaveces, 
\newblock Mathematical Thinking in Chemistry, HYLE, 18(2012), pp. 3–22.
\newblock Advances in Mathematical Chemistry and Applications. Volume 2. Basak S. C., Restrepo G., Villaveces J. L. (Bentham Science eBooks, 2015)

\bibitem[100]{key100}Brian Maitner (2018). 
\newblock BIEN: Tools for Accessing the Botanical Information and Ecology
\newblock Network Database. R package version 1.2.3. https://CRAN.R-project.org/package=BIEN

\bibitem[101]{key101} Paradis E., Claude J. and Strimmer K. 2004. 
\newblock APE: analyses of phylogenetics and evolution in R language. 
\newblock Bioinformatics 20: 289-290.

\bibitem[301]{key301} Klaus M. Herrmann 
\newblock The Shikimate Pathway as an  Entry to Aromatic Secondary Metabolism' 
\newblock Plant Physiol. (1995) 107: 7-12

\bibitem[302]{key302} Wikipedia contributors. 
\newblock Shikimate pathway 
\newblock Wikipedia, The Free Encyclopedia. Wikipedia. 

\bibitem[303]{key303}Wikipedia contributors. 
\newblock "Watercress." 
\newblock Wikipedia, The Free Encyclopedia. 

\bibitem[304]{key304} Voutsina et al.
\newblock Characterization of the watercress(Nasturtium officinaleR. Br.; Brassicaceae) transcriptome using RNASeq and
identification of candidate genes for important phytonutrient traits linked to human health
\newblock BMC Genomics (2016) 17:378 

\bibitem[401]{key401} 
\newblock Functional Foods 
\newblock Health and Disease 2012, 2(11):460-48

\bibitem[1000]{key1000}R Core Team (2015). 
\newblock R: A language and environment for statistical computing. R Foundation for Statistical Computing, Vienna, Austria.
\newblock URL https://www.R-project.org/.

\end{thebibliography}
\end{document}
