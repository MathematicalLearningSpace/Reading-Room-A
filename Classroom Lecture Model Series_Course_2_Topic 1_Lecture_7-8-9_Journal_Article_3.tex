%-------------------------------------------------------------------------%
%--------------------Classroom Lecture Model Series-----------------------%
%-------------------------------------------------------------------------%

%---------------------Work in Progress for the Classrooom-----------------%

\begin{document}
\twocolumn
\scriptsize
\begin{frontmatter}
		\title{DRAFT Sample: A Mathematical Model of Mitotic Biological Processes in a Gene Ontology}
		\author{\corref{cor1}\fnref{fn1}}
		\cortext[cor1]{Corresponding author}
		\address{The Mathematical Learning Space}
		\ead{http://mathlearningspace.weebly.com}	
\end{frontmatter}	

Introduction:
\begin{enumerate}
\item Objective 1:
\item Objective 2:
\item Objective 3:
\end{enumerate}
Conclusion:

Keywords:  Molecular Machine Learning, Shikimate Pathway, Mathematical chemistry, QSAR, Topological index,Docking (molecular), secondary  metabolites
Vocabulary Words: phenylpropanoids, phytochemicals

\section{Introduction}

\begin{table}[H]\centering
	\begin{tabular}{p{1cm}p{4cm}p{3cm}}
		Article ID & Summary & Comments\\
		\hline
		\hline
	\end{tabular}
\end{table}


\subsection{Plan of the Article}

\begin{enumerate}
\end{enumerate}

\section{Topic Review: Mathematical Chemistry}

\centering	
\begin{table}[H]\tiny
	\caption{}	
	\begin{tabular}{p{1cm}p{1cm}|p{4cm}|l}
		\hline	
		PubMed ID & Topic & Description & Results \\
		\hline 
		\hline 
	\end{tabular}
\end{table}

\subsection{Phenylpropanoid Pathway}

\begin{table}[H]\centering
	\begin{tabular}{p{1cm}p{4cm}p{3cm}}
		Article ID & Summary & Comments\\
		\hline
		\hline
	\end{tabular}
\end{table}

\subsection{Plant BioChemistry and Phytochemistry Background}
\centering	
\begin{table}[H]\tiny
	\caption{}	
	\begin{tabular}{p{1cm}p{1cm}|p{4cm}|l}
		\hline	
		PubChem ID & Topic & Description & Results \\
		\hline 
		\hline 
	\end{tabular}
\end{table}

\section{Mathematical Background}

\subsection{Equation System A}

\begin{align*} 
\tiny
\frac{d^{\alpha_1}X_1(t)}{dt^{\alpha_1}} = a_{11} *X_1(t - \tau_1) + \\
a_{12} *\frac{X_1(t)^{\beta_1}}{(1-X_1(t - \tau_1))^{\beta_1}} - a_{15}X_1(t) + \\
\epsilon_1(t) \\
\frac{d^{\alpha_1}X_2(t)}{dt^{\alpha_1}} = a_{21} *X_2(t - \tau_2) + \\
a_{22} *\frac{X_2(t)^{\beta_1}}{(1-X_2(t - \tau_2))^{\beta_1}} - a_{25}X_2(t) + \\
\epsilon_2(t) \\ \\
\frac{d^{\alpha_1}X_3(t)}{dt^{\alpha_1}} = a_{31} *X_3(t - \tau_3) + \\
a_{32} *\frac{X_3(t)^{\beta_1}}{(1-X_3(t - \tau_3))^{\beta_1}} - a_{35}X_3(t) + \\
\epsilon_3(t) \\ \\
\frac{d^{\alpha_1}X_4(t)}{dt^{\alpha_1}} = a_{41} *X_4(t - \tau_4) + \\
a_{42} *\frac{X_4(t)^{\beta_1}}{(1-X_4(t - \tau_4))^{\beta_1}} - a_{45}X_4(t) + \\
\epsilon_4(t) \\ \\
\frac{d^{\alpha_1}X_5(t)}{dt^{\alpha_1}} = a_{51} *X_5(t - \tau_5) + \\
a_{52} *\frac{X_5(t)^{\beta_1}}{(1-X_5(t - \tau_5))^{\beta_1}} - a_{55}X_5(t) + \\
\epsilon_5(t) \\
\end{align*}

\subsubsection{Parameter Table}

\begin{table}[h]\footnotesize
	\caption{Parameter Description and Value}
	\begin{tabular}{rllp{2cm}l}
		\hline	
		Parameter & Value & Interval & Description & Reference \\
		\hline 
		a11 & 0 & [0,1] & Equation 1 & \cite{key1}
		a12 & 0 & [0,1] & Equation 1 & \cite{key1}
		a13 & 0 & [0,1] & Equation 1 & \cite{key1}
		a14 & 0 & [0,1] & Equation 1 & \cite{key1}
		a15 & 0 & [0,1] & Equation 1 & \cite{key1}
		\hline
		a21 & 0 & [0,1] & Equation 2 & \cite{key1}
		a22 & 0 & [0,1] & Equation 2 & \cite{key1}
		a23 & 0 & [0,1] & Equation 2 & \cite{key1}
		a24 & 0 & [0,1] & Equation 2 & \cite{key1}
		a25 & 0 & [0,1] & Equation 2 & \cite{key1}
		\hline
		a31 & 0 & [0,1] & Equation 3 & \cite{key1}
		a32 & 0 & [0,1] & Equation 3 & \cite{key1}
		a33 & 0 & [0,1] & Equation 3 & \cite{key1}
		a34 & 0 & [0,1] & Equation 3 & \cite{key1}
		a35 & 0 & [0,1] & Equation 3 & \cite{key1}
		\hline
		a41 & 0 & [0,1] & Equation 4 & \cite{key1}
		a42 & 0 & [0,1] & Equation 4 & \cite{key1}
		a43 & 0 & [0,1] & Equation 4 & \cite{key1}
		a44 & 0 & [0,1] & Equation 4 & \cite{key1}
		a45 & 0 & [0,1] & Equation 4 & \cite{key1}
		\hline
		a51 & 0 & [0,1] & Equation 5 & \cite{key1}
		a52 & 0 & [0,1] & Equation 5 & \cite{key1}
		a53 & 0 & [0,1] & Equation 5 & \cite{key1}
		a54 & 0 & [0,1] & Equation 5 & \cite{key1}
		a55 & 0 & [0,1] & Equation 5 & \cite{key1}
		\hline
		$\tau_1$ & 1 & [1,2] & Equation 1 & \cite{key1}
		$\tau_2$ & 1 & [1,2] & Equation 2 & \cite{key1}
		$\tau_3$ & 1 & [1,2] & Equation 3 & \cite{key1}
		$\tau_4$ & 1 & [1,2] & Equation 4 & \cite{key1}
		$\tau_5$ & 1 & [1,2] & Equation 5 & \cite{key1}
		\hline
		$\alpha_1$ & 1 & (0,2] & Equation 1 & \cite{key1}
		$\alpha_2$ & 1 & (0,2] & Equation 2 & \cite{key1}
		$\alpha_3$ & 1 & (0,2] & Equation 3 & \cite{key1}
		$\alpha_4$ & 1 & (0,2] & Equation 4 & \cite{key1}
		$\alpha_5$ & 1 & (0,2] & Equation 5 & \cite{key1}
		\hline
		$\beta_1$ & 1 & (0,2] & Equation 1 & \cite{key1}
		$\beta_2$ & 1 & (0,2] & Equation 2 & \cite{key1}
		$\beta_3$ & 1 & (0,2] & Equation 3 & \cite{key1}
		$\beta_4$ & 1 & (0,2] & Equation 4 & \cite{key1}
		$\beta_5$ & 1 & (0,2] & Equation 5 & \cite{key1}
	\end{tabular}	
\end{table}




\centering
\begin{table}[H]\footnotesize
	\caption{}
	\begin{tabular}{rp{1cm}p{2cm}p{3cm}p{1cm}}
		\hline
		ID & A & B & C & Reference \\
		\hline
		\hline
	\end{tabular}
\end{table}
\raggedright


\section{Results}



\subsection{Tables}
\centering
\begin{table}[H]\footnotesize
	\caption{}
	\begin{tabular}{rp{1cm}p{2cm}p{3cm}p{1cm}}
		\hline
		ID & A & B & C & Reference \\
		\hline
		\hline
	\end{tabular}
\end{table}
\raggedright

\centering
\begin{table}[H]\footnotesize
	\caption{}
	\begin{tabular}{rp{1cm}p{2cm}p{3cm}p{1cm}}
		\hline
		ID & A & B & C & Reference \\
		\hline
		\hline
	\end{tabular}
\end{table}
\raggedright


\subsection{Figures}


\begin{figure}[H]
	\centering
	\begin{minipage}[b]{0.5\linewidth}
	%\includegraphics[scale=0.25]{Example_1_Figure_1.png}
	\end{minipage}\hfill
	\begin{minipage}[b]{0.5\linewidth}
	%\includegraphics[scale=0.25]{Example_1_Figure_2.png}
	\end{minipage}\hfill	
	\begin{minipage}[b]{0.5\linewidth}
	%\includegraphics[scale=0.25]{Example_1_Figure_3.png}
	\end{minipage}\hfill
	\begin{minipage}[b]{0.5\linewidth}
	%\includegraphics[scale=0.25]{Example_1_Figure_4.png}
	\end{minipage}\hfill
	\caption{1, 2, 3 and 4}
	\label{fig:Figure1}
\end{figure} 



\section{Conclusions and Discussion Topics}

\begin{table}[H]\centering
	\begin{tabular}{p{1cm}p{4cm}p{3cm}}
		Topic ID & Summary & Comments\\
		\hline
		\hline
	\end{tabular}
\end{table}




\section{R Application Programming Interfaces (APIs)}




\bibliographystyle{plain}
\begin{thebibliography}{00}

\bibitem[1]{key1}N. Trinajstić, I. Gutman, 
\newblock Mathematical Chemistry, 
\newblock Croatica Chemica Acta, 75(2002), pp. 329–356.

\bibitem[2]{key2}A. T. Balaban, 
\newblock Reflections about Mathematical Chemistry, 
\newblock Foundations of Chemistry, 7(2005), pp. 289–306.

\bibitem[3]{key3}G. Restrepo, J. L. Villaveces, 
\newblock Mathematical Thinking in Chemistry, HYLE, 18(2012), pp. 3–22.
\newblock Advances in Mathematical Chemistry and Applications. Volume 2. Basak S. C., Restrepo G., Villaveces J. L. (Bentham Science eBooks, 2015)

\subsection{Differential Equations}

\bibitem[1]{key1}A New Approach and Solution Technique to Solve Time Fractional Nonlinear Reaction-Diffusion Equations
\bibitem[1]{key1}Stability Analysis of Fractional-Order Nonlinear Systems with Delay
\bibitem[1]{key1}Application of the Multistep Generalized Differential Transform Method to Solve a Time-Fractional Enzyme Kinetics
\bibitem[1]{key1}Wavelet Methods for Solving Fractional Order Differential Equations
\bibitem[1]{key1}Numerical Methods for Pricing American Options with Time-Fractional PDE Models
\bibitem[1]{key1}Application of Multistep Generalized Differential Transform Method for the Solutions of the Fractional-Order Chua System
\bibitem[1]{key1}Numerical Solution of Some Types of Fractional Optimal Control Problems
\bibitem[1]{key1}An Efficient Series Solution for Fractional Differential Equations
\bibitem[1]{key1}Approximate Analytical Solution for Nonlinear System of Fractional Differential Equations by BPs Operational Matrices
\bibitem[1]{key1}Numerical Solution for Complex Systems of Fractional Order
\bibitem[1]{key1}Stability Analysis of Fractional-Order Nonlinear Systems with Delay
\bibitem[1]{key1}Numerical Study for Time Delay Multistrain Tuberculosis Model of Fractional Order
\bibitem[1]{key1}A Numerical Method for Solving Fractional Differential Equations by Using Neural Network
\bibitem[1]{key1}Numerical Studies for Fractional-Order Logistic Differential Equation with Two Different Delays
\bibitem[1]{key1}Numerical Modeling of Fractional-Order Biological Systems
\bibitem[1]{key1}Numerical Solution of Some Types of Fractional Optimal Control Problems
\bibitem[1]{key1}A Numerical Method for Delayed Fractional-Order Differential Equations
\bibitem[1]{key1} New Insights into the Fractional Order Diffusion Equation Using Entropy and Kurtosis
\bibitem[1]{key1} DELAY DIFFERENTIAL EQUATIONS IN SINGLE SPECIES DYNAMICS
\bibitem[1]{key1} An Improved Artificial Bee Colony Algorithm Based on Elite Strategy and Dimension Learning
\bibitem[1]{key1}Operators of Fractional Calculus and Their Applications
\bibitem[1]{key1} Modelling Physiological and Pharmacological Control on Cell Proliferation to Optimise Cancer Treatments

\bibitem[100]{key100}Brian Maitner (2018). 
\newblock BIEN: Tools for Accessing the Botanical Information and Ecology
\newblock Network Database. R package version 1.2.3. https://CRAN.R-project.org/package=BIEN

\bibitem[101]{key101} Paradis E., Claude J. and Strimmer K. 2004. 
\newblock APE: analyses of phylogenetics and evolution in R language. 
\newblock Bioinformatics 20: 289-290.

\bibitem[301]{key301} Klaus M. Herrmann 
\newblock The Shikimate Pathway as an  Entry to Aromatic Secondary Metabolism' 
\newblock Plant Physiol. (1995) 107: 7-12

\bibitem[302]{key302} Wikipedia contributors. 
\newblock Shikimate pathway 
\newblock Wikipedia, The Free Encyclopedia. Wikipedia. 

\bibitem[303]{key303}Wikipedia contributors. 
\newblock "Watercress." 
\newblock Wikipedia, The Free Encyclopedia. 

\bibitem[304]{key304} Voutsina et al.
\newblock Characterization of the watercress(Nasturtium officinaleR. Br.; Brassicaceae) transcriptome using RNASeq and
identification of candidate genes for important phytonutrient traits linked to human health
\newblock BMC Genomics (2016) 17:378 

\bibitem[401]{key401} 
\newblock Functional Foods 
\newblock Health and Disease 2012, 2(11):460-48

\bibitem[1000]{key1000}R Core Team (2015). 
\newblock R: A language and environment for statistical computing. R Foundation for Statistical Computing, Vienna, Austria.
\newblock URL https://www.R-project.org/.

\end{thebibliography}
\end{document}
