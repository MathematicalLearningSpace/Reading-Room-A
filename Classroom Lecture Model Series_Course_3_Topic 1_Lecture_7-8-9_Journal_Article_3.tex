%-------------------------------------------------------------------------%
%--------------------Classroom Lecture Model Series-----------------------%
%-------------------------------------------------------------------------%

%--------------------Work in Progress for the Classroom-------------------%

\begin{document}
\twocolumn
\scriptsize
\begin{frontmatter}
		\title{}
		\author{\corref{cor1}\fnref{fn1}}
		\cortext[cor1]{Corresponding author}
		\address{The Mathematical Learning Space}
		\ead{http://mathlearningspace.weebly.com}	
\end{frontmatter}	

Introduction:
\begin{enumerate}
\item Objective 1:
\item Objective 2:
\item Objective 3:
\end{enumerate}
Conclusion:

keywords: Protein Models, QSAR Models, Molecular Dynamics, Protein-Protein Interaction Models, Elastic Network Theory

\section{Introduction}

\begin{table}[H]\centering
	\begin{tabular}{p{1cm}p{4cm}p{3cm}}
		Article ID & Summary & Comments\\
		\hline
		\hline
	\end{tabular}
\end{table}

\subsection{Plan of the Article}

\begin{enumerate}
\end{enumerate}


\section{Biological Review}

\subsection{Topic A: Elastic Network Theory}

\centering	
\begin{table}[H]\tiny
	\caption{}	
	\begin{tabular}{p{1cm}p{1cm}|p{4cm}|l}
		\hline	
		PubMed ID & Topic & Description & Results \\
		\hline 
		\hline 
	\end{tabular}
\end{table}

\subsection{Topic B: Normal Modes}

\centering	
\begin{table}[H]\tiny
	\caption{}	
	\begin{tabular}{p{1cm}p{1cm}|p{4cm}|l}
		\hline	
		PubMed ID & Topic & Description & Results \\
		\hline 
		\hline 
	\end{tabular}
\end{table}

\subsection{Topic C:Quantitative Structure Activity Relationships Descriptors}

\centering	
\begin{table}[H]\tiny
	\caption{}	
	\begin{tabular}{p{1cm}p{1cm}|p{4cm}|l}
		\hline	
		PubMed ID & Topic & Description & Results \\
		\hline 
		\hline 
	\end{tabular}
\end{table}


\section{Mathematical Background: Hamiltoian Dynamics}

\centering	
\begin{table}[H]\tiny
	\caption{}	
	\begin{tabular}{p{1cm}p{1cm}|p{4cm}|l}
		\hline	
		Topic & Description & Results \\
		\hline 
		\hline 
	\end{tabular}
\end{table}

\section{Results}


\subsection{Tables}

\centering	
\begin{table}[H]\tiny
	\caption{}	
	\begin{tabular}{p{1cm}p{1cm}|p{4cm}|l}
		\hline	
		Model ID & Topic & Description & Results \\
		\hline 
		\hline 
	\end{tabular}
\end{table}


\subsection{Figures}

\begin{figure}[H]
	\centering
	\begin{minipage}[b]{0.5\linewidth}
	%\includegraphics[scale=0.25]{Example_1_Figure_1.png}
	\end{minipage}\hfill
	\begin{minipage}[b]{0.5\linewidth}
	%\includegraphics[scale=0.25]{Example_1_Figure_2.png}
	\end{minipage}\hfill	
	\begin{minipage}[b]{0.5\linewidth}
	%\includegraphics[scale=0.25]{Example_1_Figure_3.png}
	\end{minipage}\hfill
	\begin{minipage}[b]{0.5\linewidth}
	%\includegraphics[scale=0.25]{Example_1_Figure_4.png}
	\end{minipage}\hfill
	\caption{1, 2, 3 and 4}
	\label{fig:Figure1}
\end{figure} 


\section{Topics in the Classroom}

\begin{table}[H]\centering
	\begin{tabular}{p{1cm}p{4cm}p{3cm}}
		Article ID & Summary & Comments\\
		\hline
		\hline
	\end{tabular}
\end{table}

\bibliographystyle{plain}
\begin{thebibliography}{00}

\subsection{Normal Mode Analysis} 

\subsection{Compound Discovery} 

\subsection{QSAR Models}

\bibitem[1000]{key1000}R Core Team (2015). 
\newblock R: A language and environment for statistical computing. R Foundation for Statistical Computing, Vienna, Austria.
\newblock URL https://www.R-project.org/.

\end{thebibliography}

\end{document}
