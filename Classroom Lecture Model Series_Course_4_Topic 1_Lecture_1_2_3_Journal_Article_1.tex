%-------------------------------------------------------------------------%
%--------------------Classroom Lecture Model Series-----------------------%
%-------------------------------------------------------------------------%


\begin{document}
\twocolumn
\scriptsize
\begin{frontmatter}
		\title{DRAFT Sample: An Examination of Ratios and Spectral Properties for a Set of Non-Automated and Human Curated Genre Based Music Compositions}
		\author{\corref{cor1}\fnref{fn1}}
		\cortext[cor1]{Corresponding author}
		\address{The Mathematical Learning Space}
		\ead{http://mathlearningspace.weebly.com}	
\end{frontmatter}	

Introduction:
\begin{enumerate}
\item Objective 1:
\item Objective 2:
\item Objective 3:
\end{enumerate}
Conclusion:

keywords: Review of Musical Styles, Music Vocabulary Models, Music Notation

\section{Introduction}

\subsection{Plan of the Paper}

\begin{enumerate}
\item Music Theory  Review
\item A Review of Jazz Music
\item A Review of both Classical and Contemporary Classical Music
\item Mathematical Methods in Music Composition and Analysis
\item Review of Results
\item Topics for Additional Discussion in the Classroom
\end{enumerate}

\section{Music Theory Review}

\subsection{Music Model A}

\begin{table}[H]
	\caption{Composition Motifs in Album with Duration in Seconds and Acoustic Complexity}	
	\begin{tabular}{p{1cm}p{4cm}p{2cm}p{2cm}}
	\hline
	Track No & Name & Duration (seconds) & Acoustic Complexity\\
	\hline
	\hline 
	\end{tabular}
\end{table}

\subsection{Topic A: Jazz}

\centering	
\begin{table}[H]\tiny
	\caption{}	
	\begin{tabular}{r|p{4cm}|l}
		\hline	
		Topic & Description & Reference \\
		\hline 
		\hline 
	\end{tabular}
\end{table}

\subsection{Topic B: Classical and Contemporary Classical Music}

\centering	
\begin{table}[H]\tiny
	\caption{}	
	\begin{tabular}{r|p{4cm}|l}
		\hline	
		Topic & Description & Reference \\
		\hline 
		\hline 
	\end{tabular}
\end{table}

\section{Mathematical Background}

\subsection{Mathematical Methods in Music Composition and Analysis}

\subsection{Feature Matrix:Ratio Designs}

\begin{table}[H]
	\caption{Ratio Designs}	
	\begin{tabular}{p{3cm}p{4cm}p{1cm}}
	\hline
	Ratio Name & Description & \\
	\hline 
	\end{tabular}
\end{table}

\subsection{Spectral Analysis}


\section{Results}

\subsection{Data}

Table 1 provides the themes and instrument collection for each of the compositions.
	
\begin{table}[H]
\caption{Composition Collection}	
\begin{tabular}{p{1cm}p{4cm}p{2cm}p{1cm}p{1cm}}
\hline
ID & Name & Theme Pattern & Instruments & \\
\hline 
1 & Composition I &  &  & \\
2 & Composition II &  &  & \\
3 & Composition III &  & \\
4 & Composition IV & & \\
5 & Composition V & & & \\
\hline 
6 & Composition VI &  &  & \\
7 & Composition VII &  &  & \\
8 & Composition VIII &  & \\
9 & Composition IX & & \\
10 & Composition X & & & \\
\end{tabular}
\end{table}

\subsection{Tables}

\subsection{Music Track Summary}

\centering	
\begin{table}[H]\tiny
	\caption{}	
	\begin{tabular}{r|p{4cm}|l}
		\hline	
		Model & Description & Results \\
		\hline 
		\hline 
	\end{tabular}
\end{table}


\subsection{Figures}

\subsection{Group A}

\begin{figure}[H]
	\centering
	\begin{minipage}[b]{0.5\linewidth}
	%\includegraphics[scale=0.25]{Example_1_Figure_1.png}
	\end{minipage}\hfill
	\begin{minipage}[b]{0.5\linewidth}
	%\includegraphics[scale=0.25]{Example_1_Figure_2.png}
	\end{minipage}\hfill	
	\begin{minipage}[b]{0.5\linewidth}
	%\includegraphics[scale=0.25]{Example_1_Figure_3.png}
	\end{minipage}\hfill
	\begin{minipage}[b]{0.5\linewidth}
	%\includegraphics[scale=0.25]{Example_1_Figure_4.png}
	\end{minipage}\hfill
	\caption{1, 2, 3 and 4}
	\label{fig:Figure1}
\end{figure} 

\subsection{Group B}

Figures in group B form a 3 x 3 matrix for visual comparison and contrast with (a) top to bottom, (b) left to right and (c) diagonal from corner to corner.

\begin{figure}[H]
	\centering
	\begin{minipage}[b]{0.3\linewidth}
		%\includegraphics[scale=0.15]{Example_2_Figure_1.png}
	\end{minipage}\hfill
	\begin{minipage}[b]{0.3\linewidth}
		%\includegraphics[scale=0.15]{Example_2_Figure_2.png}
	\end{minipage}\hfill	
	\begin{minipage}[b]{0.3\linewidth}
		%\includegraphics[scale=0.15]{Example_2_Figure_3.png}
	\end{minipage}\hfill
	\begin{minipage}[b]{0.3\linewidth}
		%\includegraphics[scale=0.15]{Example_2_Figure_4.png}
	\end{minipage}\hfill
	\begin{minipage}[b]{0.3\linewidth}
		%\includegraphics[scale=0.15]{Example_2_Figure_5.png}
	\end{minipage}\hfill	
	\begin{minipage}[b]{0.3\linewidth}
		%\includegraphics[scale=0.15]{Example_2_Figure_6.png}
	\end{minipage}\hfill
	\begin{minipage}[b]{0.3\linewidth}
		%\includegraphics[scale=0.15]{Example_2_Figure_7.png}
	\end{minipage}\hfill
	\begin{minipage}[b]{0.3\linewidth}
		%\includegraphics[scale=0.15]{Example_2_Figure_8.png}
	\end{minipage}\hfill	
	\begin{minipage}[b]{0.3\linewidth}
		%\includegraphics[scale=0.15]{Example_2_Figure_9.png}
	\end{minipage}\hfill
	\caption{(a)}
	\label{fig:Figure1}
\end{figure} 


\section{Topics for the Classroom}

\begin{enumerate}

\end{enumerate}



\bibliographystyle{plain}
\begin{thebibliography}{00}

\bibitem[1]{key100}Sueur J., Aubin T., Simonis C. (2008). 
\newblock Seewave: a free modular tool for sound analysis and synthesis. 
\newblock Bioacoustics, 18: 213-226.

\bibitem[2]{key101}Uwe Ligges, Sebastian Krey, Olaf Mersmann, and Sarah Schnackenberg (2016). 
\newblock tuneR: Analysis of music. 
\newblock URL: http://r-forge.r-project.org/projects/tuner/.

\bibitem[3]{key102}A. Anikin (2017). 
\newblock soundgen: Parametric Voice Synthesis. 
\newblock R package version 1.1.1.

\bibitem[4]{key103}Pieretti N, Farina A, Morri FD (2011) 
\newblock A new methodology to infer the singing activity of an avian community: the Acoustic Complexity Index (ACI). 
\newblock Ecological Indicators, 11, 868-873.

\bibitem[5]{key104}Farina A, Pieretti N, Piccioli L (2011) 
\newblock The soundscape methodology for long-term bird monitoring: a Mediterranean Europe case-study. 
\newblock Ecological Informatics, 6, 354-363.

\end{thebibliography}

\end{document}
