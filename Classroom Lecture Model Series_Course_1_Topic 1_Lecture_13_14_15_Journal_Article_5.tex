%-------------------------------------------------------------------------%
%--------------------Classroom Lecture Model Series-----------------------%
%-------------------------------------------------------------------------%
\begin{document}
\twocolumn
\scriptsize
\begin{frontmatter}
		\title{DRAFT Sample: Minimal Spanning Trees and Multivariate Nonparametric Distributional Testing for Gastric Cancer Chemosensitivity}
		\author{\corref{cor1}\fnref{fn1}}
		\cortext[cor1]{Corresponding author}
		\address{The Mathematical Learning Space}
		\ead{http://mathlearningspace.weebly.com}	
\end{frontmatter}	

Introduction:
\begin{enumerate}
\item Objective 1:
\item Objective 2:
\item Objective 3:
\end{enumerate}
Conclusion:

Keywords: Semi-Markov Models, Graph Theory
Vocabulary Words:

\section{Introduction}

\begin{figure}[H]
\begin{minipage}[b]{0.3\linewidth}
\includegraphics[scale=0.40]{GastroIntestinal_Cancer.png} 
\end{minipage}\hfill
\caption{(a) GastroIntestinal Cancer}
\label{fig:Figure1}
\end{figure} 


\begin{table}[H]\centering
	\begin{tabular}{p{1cm}p{4cm}p{3cm}}
		Article ID & Summary & Comments\\
		\hline
		\hline
	\end{tabular}
\end{table}

\subsection{Plan of the Article}

\begin{enumerate}
\end{enumerate}

\section{Protein Interaction Diagram and Annotations}

\begin{tikzpicture}
	[->,>=stealth',shorten >=2pt,node distance=3cm,
	thick,main node/.style={circle,draw,scale=0.25,transform canvas={scale=0.75},font=\sffamily\Small\bfseries},
	blacknode/.style={shape=circle, draw=black, line width=2},
	bluenode/.style={shape=circle, draw=blue, line width=2},
	greennode/.style={shape=circle, draw=green, line width=2},
	rednode/.style={shape=circle, draw=red, line width=2}
	]
	%-------Legend ------------------------------------------------------------	
	\matrix [draw,below left] at (current bounding box.south) {
		\node [state,label=right:State] {}; Description. \\
		\node [shapeSquare,label=right:Square-.] {}; \\
		\node [shapeEllipse,label=right:Ellipse-.] {}; \\
		\node [shapeTriangle,label=right:Triangle-.] {}; \\
		\node [shapeHexagon,label=right:Hexagon-.] {}; \\
	};
\end{tikzpicture}

\centering	
\begin{table}[H]\tiny
\caption{Gene/Protein/Enzymes Description and References}	
\begin{tabular}{r|p{3cm}|l}
\hline	
Gene/Protein & Description & Reference \\
\hline 
\hline 
\end{tabular}
\end{table}


\section{Mathematical Review Topic: Semi-Markov Models}

\subsection{Definitions}

Let X be a N x N matrix with non-zero elements represented by 

\begin{equation}
X=
\begin{bmatrix}
a_{11} & a_{12} & ... & a_{1N} \\
a_{11} & a_{12} & ... & a_{2N} \\
.
.
.
a_{N1} & a_{N2} & ... & a_{NN} \\
\end{bmatrix}
\end{equation}

Here the representation of X is (a) a matrix of derivatives of k orders (b) a Markov Chain or (c) an adjacency matrix A where each entry $a_{i,j}$ is an edge weight with ith node to the jth node represented by the graph G with $G=(N,E,W)$ where N= nodes, E=edges and W=weights on the edge relationships.  Here the set of node relationships from the set $E={C_{1}, C_{2},...,C_{k}}$ where $C_{1}$ is no relationship,  $C_{2}$ is a directed relationship between node i and j with either left, right or bidirectional in the relationship.  These are represented by $C_{2}$, $C_{3}$ and $C_{4}$ considered to be a subset of E and represented in the adjacency matrix $A=T(X)$ where $T(*)$ is the transformation operator and 1,-1 represent the direction for $C_{2}$, $C_{3}$ and $C_{1}$ for no association between the ith and jth node. 

In the case of a matrix of derivatives with first and second order derivatives of (a) Jacobian and (b) Hessian  for the matrix, the entries for each $a_{i,j}$ can be obtained by (a) forward, (b) central and (c) and backward combinations of step size specified difference operators. 

The same matrix can be used for Feature Matrix Representation where each column is a scalar based ratio design on some interval I.

\centering
\begin{table}[H]\footnotesize
	\caption{}
	\begin{tabular}{rp{1cm}p{2cm}p{3cm}p{1cm}}
		\hline
		ID & A & B & C & Reference \\
		\hline
		\hline
	\end{tabular}
\end{table}
\raggedright

\section{Markov Chains}

Let X be a N x N fully symmetric matrix with non-zero elements represented by 

\begin{equation}
X=
\begin{bmatrix}
a_{11} & a_{12} & ... & a_{1N} \\
a_{11} & a_{12} & ... & a_{2N} \\
.
.
.
a_{N1} & a_{N2} & ... & a_{NN} \\
\end{bmatrix}
\end{equation}


\section{Hidden Markov Models}

Let X be a N x N fully symmetric matrix with non-zero elements represented by 

\begin{equation}
X=
\begin{bmatrix}
a_{11} & a_{12} & ... & a_{1N} \\
a_{11} & a_{12} & ... & a_{2N} \\
.
.
.
a_{N1} & a_{N2} & ... & a_{NN} \\
\end{bmatrix}
\end{equation}


\centering
\begin{table}[H]\footnotesize
	\caption{}
	\begin{tabular}{rp{1cm}p{2cm}p{3cm}p{1cm}}
		\hline
		ID & A & B & C & Reference \\
		\hline
		\hline
	\end{tabular}
\end{table}
\raggedright

\section{Algorithms: Viterbi}

\begin{algorithm}[H]
	\footnotesize
	\begin{algorithmic}[1]
	\State 
	\end{algorithmic}
	\caption{Viterbi}\label{Viterbi_1}
\end{algorithm}


\section{Training and Testing Hidden Markov Models}

\section{Results}

\begin{enumerate}
\end{enumerate}

\subsection{Tables}

\centering
\begin{table}[H]\footnotesize
	\caption{}
	\begin{tabular}{rp{1cm}p{2cm}p{3cm}p{1cm}}
		\hline
		ID & A & B & C & Reference \\
		\hline
		\hline
	\end{tabular}
\end{table}
\raggedright

\centering
\begin{table}[H]\footnotesize
	\caption{}
	\begin{tabular}{rp{1cm}p{2cm}p{3cm}p{1cm}}
		\hline
		ID & A & B & C & Reference \\
		\hline
		\hline
	\end{tabular}
\end{table}
\raggedright

\subsection{Figures}

\begin{figure}[H]
	\centering
	\begin{minipage}[b]{0.5\linewidth}
	%\includegraphics[scale=0.25]{Example_1_Figure_1.png}
	\end{minipage}\hfill
	\begin{minipage}[b]{0.5\linewidth}
	%\includegraphics[scale=0.25]{Example_1_Figure_2.png}
	\end{minipage}\hfill	
	\begin{minipage}[b]{0.5\linewidth}
	%\includegraphics[scale=0.25]{Example_1_Figure_3.png}
	\end{minipage}\hfill
	\begin{minipage}[b]{0.5\linewidth}
	%\includegraphics[scale=0.25]{Example_1_Figure_4.png}
	\end{minipage}\hfill
	\caption{1, 2, 3 and 4}
	\label{fig:Figure1}
\end{figure} 


\section{Comparision of Models}

\begin{table}[H]\centering
	\begin{tabular}{p{1cm}p{4cm}p{3cm}}
		Model ID & Results & Comments\\
		\hline
		\hline
	\end{tabular}
\end{table}

\section{Topics for the Classroom}

\begin{table}[H]\centering
	\begin{tabular}{p{1cm}p{4cm}p{3cm}}
		Article ID & Summary & Comments\\
		\hline
		\hline
	\end{tabular}
\end{table}

\bibliographystyle{plain}
\begin{thebibliography}{00}

\bibitem[1000]{key1000}R Core Team (2015). 
\newblock R: A language and environment for statistical computing. R Foundation for Statistical Computing, Vienna, Austria.
\newblock URL https://www.R-project.org/.

\subsection{Markov Models}

\bibitem[4101]{key4101} Scientific Software Development - Dr. Lin Himmelmann and www.linhi.com (2010). 
\newblock HMM: HMM - Hidden Markov Models. 
\newblock R package version 1.0. https://CRAN.R-project.org/package=HMM

\bibitem[4102]{key4102} Christopher H. Jackson (2011). 
\newblock Multi-State Models for Panel Data: The msm Package for R. 
\newblock Journal of Statistical Software,38(8), 1-29. URL http://www.jstatsoft.org/v38/i08/.

\bibitem[4103]{key4103} Castelo, R. and Roverato, A. 
\newblock A robust procedure for Gaussian graphical model search from microarray data with p larger than n. 
\newblock J Mach Learn Res, 7:2621-50, 2006.

\bibitem[4104]{key4104} Castelo, R. and Roverato, A. 
\newblock Reverse engineering molecular regulatory networks from microarray data with qp-graphs 
\newblock J Comput Biol, 16(2):213-27, 2009.

\bibitem[4105]{key4105} Tur, I., Roverato, A. and Castelo, R. 
\newblock Mapping eQTL networks with mixed graphical Markov models 
\newblock Genetics, 198(4):1377-93, 2014.

\bibitem[4106]{key4106} Agnieszka Krol, Philippe Saint-Pierre (2015). 
\newblock SemiMarkov: An R Package for Parametric Estimation in Multi-State Semi-Markov Models. 
\newblock Journal of Statistical Software, 66(6), 1-16. URL http://www.jstatsoft.org/v66/i06/.

\bibitem[4107]{key4107} Ingmar Visser, Maarten Speekenbrink (2010). 
\newblock depmixS4: An R Package for Hidden Markov Models. 
\newblock Journal of Statistical Software, 36(7), 1-21. URL http://www.jstatsoft.org/v36/i07/.

\bibitem[4108]{key4108} Oscar Rueda, Ramon Diaz-Uriarte. zlib from Jean-loup Gailly and Mark Adler (2015). 
\newblock mRJaCGH: Reversible Jump MCMC for the Analysis of CGH Arrays. 
\newblock R package version 2.0.4. https://CRAN.R-project.org/package=RJaCGH

\end{thebibliography}
\end{document}

